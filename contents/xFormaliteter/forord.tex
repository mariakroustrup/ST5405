\chapter*{Forord og læsevejledning}

\section*{Forord}
Dette projekt er udarbejdet af gruppe ST5405, 5. semesters studerende på ingeniøruddannelsen sundhedsteknologi på Aalborg Universitet. Projektet er udarbejdet i perioden 1. september til 19. december år 2016. Projektet er opstillet af Sten Rasmussen, som er overlæge på ortopædkirurgisk afdeling på Aalborg Universitetshospital, og omhandler risikovurdering ved ortopædkirurgi. På dette semester er der samarbejdet med ortopædkirurgisk afdeling på Aalborg Universitetshospital. I projektet tages der udgangspunkt i en klinisk teknologi herunder en prædiktiv model til forudsigelse af indlæggelsesvarigheden for pateinter på ortopædkirurgisk afdelingen. 


Vi vil gerne takke vores hovedvejleder Pia B. Elberg, kliniske vejleder Sten Rasmussen samt bi-vejleder Christian Kruse for vejledning og feedback gennem hele projektperioden. Derudover vil vi give en særlig tak til ortopædkirurgisk afdeling på Aalborg Universitetshospital for samarbejdet. 


\section*{Læsevejledning}
Rapporten er udarbejdet efter den problembaserede AAU-model. Selve rapporten er inddelt i fire kapitler samt bilag. I første kapitel indeholder projektets indledning samt den initierende problemstilling der ligger til grund for problemanalysen, som fremgår af andet kapitel. Tredje kapitel beskriver problemløsningen, hvor det analyseres og vurderes brugen af prædiktiv modellering ift. Forudsigelse af indlæggelsesvarigheden. Det fjerde kapitel omhandler syntese, der indeholder en diskussion, konklusion samt perspektivering af projektet. Kapitlet efterfølges af litteraturliste samt bilag. 


Til håndtering af kilder anvendes Vancouver-metoden. De anvendte kilder nummereres i kantede parenteser. Er referencen placeret efter et punktum i en sætning, tilhører den hele afsnittet. Er referencen placeret før et punktum, tilhører den sætningen. Er der placeret ere referencer efter hinanden, betyder dette, at der er anvendt flere referencer til den pågældende sætning eller afsnit. Kilderne er angivet i litteraturlisten med eksempelvis forfatter, titel samt årstal. 


Anvendte forkortelser er skrevet ud første gang, hvorefter forkortelsen ses i en parentes efterfølgende og anvendes herefter fremadrettet i rapporten. 


Rapporten er udarbejdet i Latex, herudover anvendes MATLAB til databehandling samt visualisering af grafer og SPSS til beregning af statisk. 
