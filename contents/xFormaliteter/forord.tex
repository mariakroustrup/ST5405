\chapter*{Forord og læsevejledning}

\section*{Forord}
Dette projekt er udarbejdet af gruppe $16$gr$5405$, $5$. semesters studerende på ingeniøruddannelsen sundhedsteknologi på Aalborg Universitet. Projektet er udarbejdet i perioden $2$. september til $19$. december år $2016$. Projektforslaget er stillet af Sten Rasmussen, overlæge på ortopædkirurgisk afdeling på Aalborg Universitetshospital, og omhandler prædiktiv modellering til forudsigelse af indlæggelsesvarigheden for patienter mhp. at effektivisere planlægningen. 


Vi vil gerne takke hovedevejleder Pia B. Elberg, kliniske vejleder Sten Rasmussen samt kliniske bi-vejleder Christian Kruse for vejledning og feedback gennem hele projektperioden. Derudover vil vi give en særlig tak til ortopædkirurgisk afdeling på Aalborg Universitetshospital for samarbejdet. 


\section*{Læsevejledning}
Rapporten er inddelt i fem kapitler. Det første kapitel indeholder projektets indledning samt den initierende problemstilling, der ligger til grund for problemanalysen, som fremgår af andet kapitel. Metoden beskrives i tredje kapitel. Fjerde kapitel analyserer implementeringen af en prædiktiv model på ortopædkirurgisk afdeling ift. forudsigelse af indlæggelsesvarighed for patienter mhp. planlægning af disse. Det fjerde kapitel er syntese, der indeholder en diskussion, konklusion samt perspektivering af projektet. Kapitlerne efterfølges af bibliografi samt bilag. 


Til håndtering af kilder anvendes Vancouver-metoden. De anvendte kilder nummereres i kantede parenteser. Er referencen placeret efter et punktum i en sætning, tilhører den hele afsnittet. Er referencen placeret før et punktum, tilhører den sætningen. Er der placeret flere referencer efter hinanden, betyder dette, at der er anvendt flere referencer til den pågældende sætning eller afsnit. Reference til bilagene er ligeledes indsat i kantede parenteser og er placeret efter samme metode som kilder. 

Forkortelser er skrevet ud ved første anvendelse, hvorefter forkortelsen er skrevet i parentes. Denne forkortelse anvendes herefter fremadrettet i rapporten. 


Rapporten er udarbejdet i \LaTeX, herudover anvendes MATLAB $2016$b til databehandling samt visualisering af grafer. 
