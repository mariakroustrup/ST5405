\chapter*{Metode}
\section*{Metode for rapporten}
Denne rapport fokuserer på at analysere og vurdere, hvorvidt indlæggelsesvarigheden kan vurderes ud fra en prædiktiv model ved analyse af parametre der har indflydelse på indlæggelsesvarigheden. For at kunne belyse den initierende problemstilling, blev der i problemanalysen undersøgt, hvilke problemer det kan medføre, hvis kapacitetsudnyttelsen ikke udnyttes optimal på afdelingen samt omfanget af belægning belyst.


Analysen af problemet leder frem til en problemformulering. I problemløsningen analyseres, hvilke parametre der har indflydelse på indlæggelsesvarigheden samt, hvorvidt der findes en sammenhæng mellem parametrene. Efterfølgende vurderes og analyseres modeller inden for prædiktion af indlæggelsesvarigheden samt, hvilke overvejelser der skal vurderes inden påbegyndelse af en model. 


Problemløsningen er belyst ud fra problemformulering og vil blive diskuteret, konkluderet og perspektiveret i en syntese. Problemanalysen og problemløsning er understøttet med statistisk, litteratur og udarbejdet interviews med ortopædkirurgisk afdeling på Aalborg Universitetshospital.  /fxnote{Ved ikke om vi skal have selve problemstillingen og problemformuleringen inde i metoden når den bliver nævnt.}. 


\subsection*{Interview}
Relevant information og forståelse af ortopædkirurgisk afdeling på Aalborg Universitet er indsamlet ved kvalitative interviews med sygeplejersker og lægesekretær. Før interviewet udarbejdes spørgsmål som fremgår af bilag \ref{BilagA} og opstillet eksklusionskriterier for interviewet. 


Personalet må mindst have arbejdet på afdelingen 1 år, da der ikke ønskes en sammenligning med andre afdelinger. Derudover ønskes der en åben dialog, hvorfor personalet ikke må have kendskab til spørgsmålene inden interviewet. Interviewet er optaget og vil efterfølgende skriberes, hvor fyldeord som eksempelvis ‘Ja’ og ‘Øh’ er udeladt. Derudover kan der være byttet om på ordstillingen med henblik på tydeliggørelse af meningen. Interviewer er kortet ned til de spørgsmål som fremgår af bilag \ref{Bilag A} for at gøre sammenligningen mellem informanters svar tydeligere. \fxnote{Ved ikke lige hvordan jeg skulle skrive det her?? Vi tager den på mandag} 


\subsection*{Behandling af data}
Aalborg Universitetshospital har i et tidligere projekt indsamlet data fra 970 hospitalsindlæggelser på ortopædkirurgisk afdeling. Dette er indsamlet fra digitale patientjournaler i perioden 1. august til 31 oktober år 2014 \fxnote{Er i tvivl om vi skulle have det med digitale patientjournaler?}. Data fordeler sig på 78 forskellige parametre herunder demografiske og kliniske faktorer. I datasættet er flere datapunkter ikke udfyldt, hvorfor datasættet er behandlet. 
Datasættet er behandlet i MATLAB 2015b således udvalgte kolonner er samlet i et nyt sæt. Derudover er alle rækker med tomme eller korrupte celler blevet fjernet. Dette har medført, at datasættet er reduceret fra 970 til 472 datapunkter. I rapporten anvendes datasættet med 472 datapunkter til udarbejdelse af grafer og statistik. 
 




