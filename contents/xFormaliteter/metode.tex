\chapter{Metode}
%Dette projekt er udarbejdet mhp. at kunne analysere en prædiktiv model, der har til formål at forudsige indlæggelsesvarigheden for patienter på OA. For at belyse den initierende problemstilling, undersøges det i problemanalysen, hvordan afdelingen påvirkes af ubalance i kapacitetsudnyttelsen samt, hvilken betydning handleplanen 2017 vil have for afdelingen. Denne problemanalyse leder frem til en problemformulering, der herefter leder frem til en analyse af prædiktiv modellering. Dette vurderes ud fra en analyse af implementeringen samt betydningen af en prædiktiv model. Herefter vil problemanalysen, problemformuleringen samt analysen af prædiktiv model diskuteres, konkluderes og perspektiveres i en syntese. 

Dette kapitel omhandler metode anvendt i projektrapporten. Projektet tager udgangspunkt i litteratur samt interviews af personale fra afdelingen, statistik og data fra OA. 


\section{Litteratursøgning}
For at kunne effektivisere litteratursøgningen for projektet er der opstillet kriterier. Disse er bl.a. at litteratur som udgangspunkt ikke må være ældre end $20$ år og skal være af den nyeste udgave, samt have tilgængeligt DOI eller ISBN nummer. 
Det er så vidt muligt forsøgt at anvende peer-reviewede artikler, der eks. er fundet gennem primo, NCBI eller SpringerLink. Herudover anvendes PRI, sundhedsdatastyrelsen, samt informationspjecer fra OA, da der primært søges litteratur for det danske sundhedssystem med fokus på OA. Der er dog anvendt litteratur fra andre lande, primært tværnationale europæiske studier, hvor der hertil er taget forbehold for forskelle i sundhedsvæsner.


\section{Interview}
Der er foretaget kvalitative interviews af to sygeplejersker samt én lægesekretær fra OA for at besvare tvivlsspørgsmål omkring kapacitet, arbejdsgang samt planlægning af patienter. Interviewene er afholdt, da nogle spørgsmål ikke har været mulige at besvare gennem litteraturen. Der er forinden interviewene udarbejdet spørgsmål for således at sikre, at de uklare spørgsmål blev besvaret. Disse spørgsmål fremgår af bilag[\ref{bilagA}]. Der blev ligeledes opstillet nogle kriterier for interviews. Herunder ønskes det at afholde mindst tre interviews med personale fra OA for at undersøge, hvorvidt der er overensstemmelse mellem svarene. Det ønskes, at personalet har arbejdet på OA i mindst ét år for at sikre, at de har kendskab til planlægningen og arbejdsgangen på afdelingen. Herudover ønskes det ikke, at personalet kender spørgsmålene på forhånd, for således at undgå bias i besvarelsen. Det foretrækkes at have en åben dialog, da det hermed er muligt at stille opfølgende spørgsmål for at sikre forståelse under interviewene. Efterfølgende er interviewene blevet transskriberet, hvilket fremgår af bilag \ref{bilagO1}, bilag \ref{bilagO2} og bilag \ref{bilagsek}.


\section{Behandling af data}
Der er i et tidligere projekt indsamlet data fra $970$ hospitalsindlæggelser på OA. Dette er indsamlet fra Clinical Suite i perioden fra juli til og med oktober år $2014$. Datasættet indeholder forskellige parametre for patienterne. Disse er fordelt over $78$ parametre, herunder demografiske- og kliniske faktorer. De $78$ parametre er dog ikke udfyldt for hver enkelt patient, hvilket resulterer i flere tomme celler. Hermed er datasættet behandlet for dette i MATLAB $2016$b, hvor udvalgte kolonner er samlet i et nyt subsæt, hvorefter rækker med tomme celler er fjernet. Denne databehandling har medført, at data opsamlet fra $970$ hospitalsindlæggelser er reduceret til $472$. I rapporten anvendes datasættet med $472$ hospitalsindlæggelser på OA til udarbejdelse af grafer i MATLAB, medmindre andet er angivet.