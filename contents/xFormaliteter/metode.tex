\chapter{Metode}
Dette projekt er udarbejdet mhp. at kunne analysere en prædiktiv model, der har til formål at forudsige indlæggelsesvarigheden for patienter på OA. For at belyse den initierende problemstilling, undersøges det i problemanalysen, hvordan afdelingen påvirkes af ubalance i kapacitetsudnyttelsen samt, hvilken betydning handleplanen 2017 vil have for afdelingen. Denne problemanalyse leder frem til en problemformulering, der herefter leder frem til en analyse af prædiktiv modellering. Dette vurderes ud fra en analyse af implementeringen samt betydningen af en prædiktiv model. Herefter vil problemanalysen, problemformuleringen samt analysen af prædiktiv model diskuteres, konkluderes og perspektiveres i en syntese. Projektet tager udgangspunkt i litteratur samt interviews af personale fra afdelingen, statistik og data fra OA. 


\subsection{Litteratursøgning}
For at kunne præcisere litteratursøgningen for projektet er der opstillet kriterier. Det er så vidt muligt forsøgt at anvende peer-reviewed artikler, der eksempelvis er søgt gennem primo, NCBI eller SpringerLink for at sikre kvalitet. Herudover anvendes PRI, sundhedsdata styrelsen, Danmarks statistik samt informationspjecer fra OA, da der primært søges litteratur for det danske sundhedssystem med fokus på OA. 


\subsection{Interview}
Der er foretaget interviews af 2 sygeplejersker samt 1 lægesekretær fra OA for at besvare tvivlsspørgsmål omkring kapacitet, arbejdsgang samt planlægning. Disse spørgsmål har ikke været mulige at besvare ud fra litteraturen pga. mangel af dokumentation fra OA. Der opstilles spørgsmål som interviewene skal tage udgangspunkt i for at sikre kvaliteten. Disse spørgsmål fremgår bilag \ref{bilagA}. Hertil opstilles følgende kriterier 

%\begin{itemize}
%\item Vi ønsker at snakke med 3 sygeplejersker hver for sig
%\item Vi ønsker, at sygeplejerskerne skal have arbejdet på OA i mindst 1 år.
%\item Vi ønsker ikke, at personalet kender spørgsmålene på forhånd, da vi foretrækker en åben dialog.
%\item\ Interviewet skal bruges til at underbygge argumenter i rapporten, og personalet vil derfor muligvis blive citeret.
%\end{itemize}

\subsection{Behandling af data}
Aalborg Universitetshospital har i et tidligere projekt indsamlet data fra 970 hospitalsindlæggelser på OA. Dette er indsamlet fra Clinical Suite i perioden 1. august til 31 oktober år 2014. Datasættet beskriver forskellige parametre for patienterne. Disse er fordelt over 78 parametre, herunder demografiske- og kliniske faktorer. De 78 parametre er dog ikke udfyldt for hver enkelt patient, hvilket resulterer i flere tomme celler. Hermed er datasættet behandlet for dette i MATLAB 2015b, hvor tomme celler er fjernet og udvalgte kolonner er herefter samlet i et nyt sæt. Denne databehandling har medført, at data opsamlet fra 970 hospitalsindlæggelser er reduceret til 472. I rapporten anvendes datasættet med 472 hospitalsindlæggelser på OA til udarbejdelse af grafer i MATLAB. 
