\chapter*{Metode}
\section*{Metode for rapporten}
Denne rapport fokuserer på at analysere og vurdere, hvorvidt indlæggelsesvarigheden af patienter kan forudsiges ud fra en prædiktiv model ved analyse af parametre, der har indflydelse på indlæggelsesvarigheden. For at kunne belyse den initierende problemstilling, er det i problemanalysen undersøgt, hvilke problemer det kan medføre, hvis kapacitetsudnyttelsen ikke udnyttes optimalt på afdelingen. %samt omfanget af belægning belyst.

Analysen af problemet leder frem til en problemformulering og dertil et problemløsningsafsnit. I problemløsningen analyseres det hvilke parametre der har indflydelse på indlæggelsesvarigheden samt, hvorvidt der findes en sammenhæng mellem parametrene. Efterfølgende vurderes og analyseres modeller inden for prædiktion af indlæggelsesvarigheden samt, hvilke overvejelser der skal vurderes inden påbegyndelse af en model. 


Problemløsningen er belyst ud fra problemformulering og diskuteres, konkluderes og perspektiveres i en syntese. Problemanalysen og problemløsning er understøttet med statistik, litteratur og udarbejdede interviews med ortopædkirurgisk afdeling på Aalborg Universitetshospital.


\subsection*{Interview}
Relevant information og forståelse af ortopædkirurgisk afdeling på Aalborg Universitet er indsamlet ved kvalitative interviews med sygeplejersker og lægesekretær. Interviewene er udarbejdet fra spørgsmål som fremgår af bilag \ref{bilagA}. For at sikre kvaliteten af interviewene er der forinden opstillet eksklusions kriterier for disse. Personalet skal som minimum have arbejdet på afdelingen 1 år, da der ikke ønskes en sammenligning med andre afdelinger. Derudover ønskes der en åben dialog, hvorfor personalet ikke må have kendskab til spørgsmålene inden interviewet. Interviewet er optaget og vil efterfølgende transskriberes, hvor fyldeord som eksempelvis ‘Øh’ er udeladt. Derudover kan der være byttet om på ordstillingen med henblik på at tydeliggøre meningen. Interviewet er kortet ned til de spørgsmål, som fremgår af bilag \ref{bilagA} for at gøre sammenligningen mellem informanters svar tydeligere.


\subsection*{Behandling af data}
Aalborg Universitetshospital har i et tidligere projekt indsamlet data fra 970 hospitalsindlæggelser på ortopædkirurgisk afdeling. Dette er indsamlet fra Clinical Suite i perioden 1. august til 31 oktober år 2014. Data fordeler sig på 78 forskellige parametre herunder demografiske og kliniske faktorer. I datasættet er flere datapunkter ikke udfyldt, hvorfor datasættet er behandlet. 
Datasættet er behandlet i MATLAB 2015b således udvalgte kolonner er samlet i et nyt sæt. Derudover er alle rækker med tomme eller korrupte celler blevet fjernet. Dette har medført, at datasættet er reduceret fra 970 til 472 datapunkter. I rapporten anvendes datasættet med 472 datapunkter til udarbejdelse af grafer og statistik. 
 




