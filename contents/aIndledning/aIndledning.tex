\chapter{Indledning} \label{indl}
Flere danske hospitalsafdelinger oplever en ubalance i kapacitetsudnyttelsen, hvilket kan betyde, at der er flere patienter end der er kapacitet til. Dette medfører mangel på sengepladser, personale og rum.\cite{Company2013} Overlæge Sten Rasmussen på ortopædkirurgisk afdeling på Aalborg Universitetshospital (OA) udtaler, at omfanget af kapacitetsmangel opleves stigende og tiltagende for hver gang det opstår[\ref{sten}]. I år $2017$ forekommer en ny budgetfordeling, der skal medføre en hurtigere udredning af patienter samt en reduktion i den gennemsnitlige ventetid\cite{Budget2016}. 
Foruden budgetfordelingen $2017$ forventes det, at procentdelen af danskere over $65$ år vil stige fra $29~\%$ til $34~\%$ inden år $2025$, hvilket angiveligt vil øge antallet af fremtidige patienter. En stigning i antallet af fremtidige patienter samt den nye budgetfordeling vil dermed kunne skabe udfordringer ift. planlægning af patienterne.

Perioder, hvor afdelingerne oplever kapacitetsmangel, kan medføre overarbejde for personalet, hvor sygeplejersker ofte skal varetage flere patienter\cite{Danske2015}[\ref{bilagO1}]. Hertil mener hver anden regionalt ansat sygeplejerske på tværs af regionerne, at den travle arbejdsdag påvirker patientsikkerheden\cite{Kjeldsen2015}. Et studie påviser dertil, at en $10~\%$ overskridelse af den normerede belægning øger mortalitetsraten på afdelingen med $1,2~\%$. Dog kan andre parametre end belægning være årsag til den forøgede mortalitet. Foruden personalets øgede risiko for at begå fejl ift. behandlingen af patienter forekommer der ligeledes kapacitetsmangel, som medfører, at patienter overflyttes til uhensigtsmæssige områder som f.eks. gangarealer og fyldte stuer\cite{Madsen2014}. Dette kan forårsage, at patienter såvel som pårørende oplever et skærpet privatliv\cite{Heidmann2014}.

Foruden kapacitetsmangel opstår der perioder, hvor OA ikke modtager patienter nok ift., hvad der er kapacitet til. Dette kan skyldes, at patienterne ikke er planlagt mhp. kapacitet. Det kan derved tyde på, at en effektivisering af planlægningen af patienter kan hjælpe afdelingen med at opretholde en balance i kapacitetsudnyttelsen. 

\section{Initierende problemstilling}
\textit{Hvordan påvirkes ortopædkirurgisk afdeling på Aalborg Universitetshospital af ubalance i kapacitetsudnyttelse, og hvilke konsekvenser medfører dette?}




%\chapter{Indledning}   
%Flere danske hospitalsafdelinger oplever i perioder at have flere patienter end der er kapacitet til. Dette medfører, at der sker en ubalance i kapacitetsudnyttelsen, da der forekommer mangel på sengepladser, personale og rum.
%I budgetfordelingen for ortopædkirurgisk afdeling på Aalborg Universitetshospital, udarbejdet for år 2017 indgår det, at ventetiden på en operation, for elektive patienter, skal reduceres fra 57 dage til 50 dage\cite{Budget2016}. Dermed forventes det, at presset for at få opereret elektive patienter øges. \fxnote{måske tilføje noget ift. det Sten siger pr. mail}
%
%
%Derudover forventes det, at procentdelen af danskere over 65 år vil stige fra $29~\%$ til $34~\%$ og dermed også antallet af fremtidige patienter\cite{RegionNord2016}. En stigning i antallet af patienter vil i takt med kortere ventetid på behandling skabe en udfordring ift. planlægning af de elektive indlæggelser. Dette kan medføre, at der i perioder opstår en ubalance mellem patienter og personale, hvilket fører til udfordringer ift. at varetage ekstra patienter og dermed forringede arbejdsvilkår for personalet. 
%
%
%Hertil mener hver anden regionalt ansat sygeplejerske på tværs af regionerne, at den travle arbejdsdag påvirker patientsikkerheden\cite{Kjeldsen2015}. Et studie påviser, at ved blot én ekstra indlagt patient i 30 dage pr. sygeplejerske øges mortalitetsraten for patienten med $7~\%$\cite{Aiken2002}. Foruden personalets øgede risiko for at begå fejl ift. behandlingen af patienter forekommer der ligeledes kapacitetsmangel, som medfører, at patienter overflyttes til uhensigtsmæssige områder som f.eks. gangarealer og fyldte stuer\cite{Madsen2014}. Dette kan forårsage, at patienter såvel som pårørende oplever et skærpet privatliv\cite{Heidmann2014}. 
%
%
%På ortopædkirurgisk afdeling på Aalborg Universitetshospital opleves ligeledes en ubalance i kapacitetsudnyttelse. Over en 35 måneders periode forekommer der hhv. en belægning over og under $100~\%$\cite{SDS2015}. Dette betyder, at ressourcerne ikke udnyttes optimalt, hvortil afdelingen oplever perioder med mangel på personale og perioder med for meget personale ift. indlagte patienter. Hvis planlægning af patienter struktureres vil dette kunne hjælpe med at balancere kapacitetsudnyttelsen. Hensigten med dette vil være, at gøre ortopædkirurgisk afdeling på Aalborg Universitetshospital bedre rustet til at imødekomme budgetfordelingen år 2017.
%
%
%\section{Initierende problemstilling}
%\textit{Hvordan påvirkes ortopædkirurgisk afdeling på Aalborg Universitetshospital af ubalance i kapacitetsudnyttelse og, hvordan vil budgetfordelingen år 2017 samt den øgede patientbyrde påvirke afdelingen?}






%Flere danske hospitalsafdelinger oplever i perioder at have flere patienter end der er kapacitet til. Dette medfører at der sker en ubalance i kapacitetsudnyttelsen, da der kan være mangel på sengepladser, personale  og rum.\cite{Company2013}.  I budgetfordelingen for Aalborg Universitetshospital i år 2017 indgår det, at ventetiden på en operation for elektive patienter skal reduceres fra 57 dage til 50 dage\cite{Budget2016}. Dette forventes at medføre, at det daglige antal elektive patienter, der indlægges, vil skabe en reducering i antallet af ledige sengepladser til akutte patienter. 
%
%
%
%Overskridelsen af kapaciteten resulterer bl.a. i, at personalet får mindre tid pr. indlagt patient, hvilket kan medføre gener for både personalet og patienter.\cite{Kjeldsen2015}
%
%Et estimat fra 2016 indikerer, at procentdelen af danskere over 65 år vil stige fra $29~\%$ til $34~\%$ og dermed også antallet af patienter\cite{RegionNord2016}. En stigning i antallet af patienter vil i takt med kortere ventetid på behandling skabe et aktuelt problem ift. planlægning af indlæggelser samt kapacitetudnyttelse på ortopædkirurgisk afdeling. Ifølge en undersøgelse fra Dansk Sygeplejeråd, mener hver anden regionalt ansat sygeplejerske på tværs af regionerne, at den travle arbejdsdag påvirker patientsikkerheden\cite{Kjeldsen2015}. Foruden personalets øgede risiko for at begå fejl ift. behandlingen af patienter, kan der ligeledes opstå en sundhedsrisiko ved kapacitetsmangel. Manglen på fysisk kapacitet kan give anledning til at overflytte patienter til uhensigtsmæssige områder som f.eks. hosptialsgange\cite{Madsen2014}. Dermed er der opstået et aktuelt problem som vedrører kapacitetsmangel, og konsekvenserne af dette problem bør undersøges nærmere.\fxnote{Denne sætning skal omformuleres så det ikke lyder som et aktuelt problem.} Ved at udnytte kapaciteten på afdelingen opnås der mere sundhed for pengene\cite{Company2013}. På baggrund af dette opstilles følgende initierende problem: \fxnote{Mere fokus på ortopædkirurgisk afdeling. Hvilken sundhedsrisiko er der ved kapacitetsmangel (fejl ved længere arbejdsdage + mortalitetsrate). Vi skal have skrevet noget om belægning i indledningen, så det passer til den initierende problemstilling}
%
%
%
%\textit{Hvordan påvirkes ortopædkirurgisk afdeling på Aalborg Universitetshospital af ændringerne vedrørende kapacitetsudnyttelse og hvor udbredte er belægningsrelaterede problemer på afdelingen?} \fxnote{skriv så det fremgår at det er på daglig basis og ikke har noget med handleplanen at gøre.}


% Der er i dag overbelægning på flere afdelinger på de danske hospitaler, hvoraf nogle afdelingerne berøres i hele og flere måneder ad gangen. \cite{2015} Overbelægning resulterer i, at sundhedspersonalet får mindre tid pr. indlagt patient. Ifølge en undersøgelse fra Dansk Sygeplejeråd, mener hver anden regionalt ansat sygeplejerske på tværs af regionerne, at den travle arbejdsdag går ud over patienternes sikkerhed \cite{Kjeldsen2015}. Et studie påviser, at ved blot én ekstra indlagt patient pr. sygeplejerske øges mortalitetsraten med $7~\%$ indenfor en indlæggelse på 30 dage \fxnote{Har stadig lidt svært ved om man kan forstå sætningen på den rigtige måde.} \cite{Aiken2014}. 

% Foruden sundhedspersonalets øgede risiko for at begå fejl ift. behandlingen af patienter, er der ligeledes en sikkerhedsrisiko forbundet ved overbelægning på hospitalerne. De ekstra patienter, der ligger på stuerne, gangene og vaskerummene, pga. overbelægning, er en større udfordring ved evakuering under brand. Pladsmangel, som medfører, at patienterne opholder sig i vaskerummene og på gangene, bevirker desuden til, at patienterne oplever et skærpet privatliv. \cite{Madsen2014}

% Aalborg universitetshospital har i et tidligere projekt indsamlet data fra $1.000$ hospitalsindlæggelser. Disse data inkluderer blodprøveanalyser og knoglescanninger (DXA), hvilket formodes at kunne anvendes til udvikling af en prædiktiv model, der kan estimere indlæggelsesvarigheden blandt patienter. Denne rapport vil på baggrund af dette undersøge, hvorvidt det er muligt at forudsige indlæggelsesvarigheden ved brug af machine learning. \fxnote{mangler stadig noget om begrundelsen for valg af machine learning}



% \section{Initierende problemstilling} \fxnote{Dette er ikke den endelige problemstilling}
% Hvordan påvirkes personalet og patienterne af overbelægning på hospitaler, og hvilke konsekvenser har overbelægning i forhold til sikkerheden på afdelingerne?
% Hvilke typer af machine learning findes der, samt hvor benyttes det i dag?

