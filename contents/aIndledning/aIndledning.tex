\chapter{Indledning}
Der er i dag overbelægning på flere afdelinger på de danske hospitaler. Nogle af afdelingerne berøres i hele og flere måneder adgangen. \cite{2015} Dette resulterer i, at sundhedspersonalet får mindre tid pr. indlagt patient. Ifølge en undersøgelse fra Dansk Sygeplejeråd, mener hver anden regionalt ansat sygeplejerske på tværs af regionerne, at den travle arbejdsdag går ud over patienternes sikkerhed \cite{Kjeldsen2015}. Et studie påviser, at ved blot én ekstra indlagt patient pr. sygeplejerske øges mortalitetsraten med $7 \%$ indenfor en indlæggelse på 30 dage \fxnote{Har stadig lidt svært ved om man kan forstå sætningen på den rigtige måde.} \cite{Aiken2014}. 

Foruden sundhedspersonalets øgede risiko for at begå fejl ift. behandlingen af patienter, er der ligeledes en sikkerhedsrisiko forbundet ved overbelægning på hospitalerne. De ekstra patienter der er kommet på stuerne, gangene og vaskerummene, pga. overbelægning, er en større udfordring ved evakuering under brand. Pladsmangel, som medfører at patienterne opholder sig i vaskerummene og på gangene, bevirker desuden til at patienterne oplever et skærpet privatliv. \cite{Madsen2014}

Aalborg universitetshospital har i et tidligere projekt indsamlet data fra $1.000$ hospitalsindlæggelser. Disse data inkluderer blodprøveanalyser og knoglescanninger (DXA), hvilket formodes at kunne anvendes til udvikling af en prædiktiv model, der kan estimere indlæggelsesvarigheden blandt patienter. Denne rapport vil på baggrund af dette undersøge, hvorvidt det er muligt at forudsige indlæggelsesvarigheden ved brug af machine learning. \fxnote{mangler stadig noget om begrundelsen for valg af machine learning}



\section{Initierende problemstilling}
Hvordan påvirkes personalet og patienterne af overbelægning på hospitaler og hvilke konsekvenser har overbelægning i forhold til sikkerheden på afdelingerne?
Hvilke typer af machine learning findes der, hvor benyttes det i dag og hvordan relaterer det til problemstillingen.

