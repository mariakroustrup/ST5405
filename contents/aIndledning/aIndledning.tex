\chapter{Indledning}
Der ses i dag stor overbelægning på flere afdelinger på de danske hospitaler. Dette resulterer i, at sundhedspersonalet får mindre tid pr. indlagt patient. Ifølge en undersøgelse fra Dansk Sygeplejeråd, mener hver anden regionalt ansat sygeplejerske, at den travle arbejdsdag går ud over patienternes sikkerhed \citep{Kjeldsen2015}. Et studie påviser, at ved blot én ekstra indlagt patient pr. sygeplejerske øges mortalitetsraten med $7 \%$ indenfor 30 dage \fxnote{Har stadig lidt svært ved om man kan forstå sætningen på den rigtige måde.} \citep{Aiken2014}. 
Foruden sundhedspersonalets øget risko for at begå fejl i forhold til patienter, er der ligeledes en sikkerhedsrisiko forbundet ved overbelægning på hospitalerne. De ekstra patienter liggende på stuerne, gangene og vaskerummene er en større udfordring ved evakuering under brand. Patienterne oplever ydereligere skærpet privatliv ved overbelægning. \citep{Madsen2014} 
De danske hospitalsafdelinger oplever ikke overbelægning i samme grad. Dette viser sig ved, at flere afdelinger berøres af overbelægning svarende til en hel måned adgangen, mens andre ikke er berørt. \citep{2015}

