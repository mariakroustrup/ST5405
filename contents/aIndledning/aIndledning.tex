\chapter{Indledning}


Flere danske hospitalsafdelinger oplever i perioder at have flere patienter end der er kapacitet til, i form af mangel på sengepladser, personale eller rum\cite{Company2013}. Overskridelsen af kapaciteten resulterer bl.a. i, at personalet får mindre tid pr. indlagt patient, hvilket kan medføre gener for både personalet og patienter.\cite{Kjeldsen2015} I budgetfordelingen for Aalborg Universitetshospital i år 2017 indgår det, at ventetiden på en operation for elektive patienter skal reduceres fra 57 dage til 50 dage\cite{Budget2016}. Dette forventes at medføre, at det daglige antal elektive patienter, der indlægges, vil skabe en reducering i antallet af ledige sengepladser til akutte patienter. 

Et estimat fra 2016 indikerer, at procentdelen af danskere over 65 år vil stige fra $29~\%$ til $34~\%$ og dermed også antallet af patienter\cite{RegionNord2016}. En stigning i antallet af patienter vil i takt med kortere ventetid på behandling skabe et aktuelt problem ift. planlægning af indlæggelser samt kapacitetudnyttelse på ortopædkirurgisk afdeling. Ifølge en undersøgelse fra Dansk Sygeplejeråd, mener hver anden regionalt ansat sygeplejerske på tværs af regionerne, at den travle arbejdsdag påvirker patientsikkerheden\cite{Kjeldsen2015}. Foruden personalets øgede risiko for at begå fejl ift. behandlingen af patienter, kan der ligeledes opstå en sundhedsrisiko ved kapacitetsmangel. Manglen på fysisk kapacitet kan give anledning til at overflytte patienter til uhensigtsmæssige områder som f.eks. hosptialsgange\cite{Madsen2014}. Dermed er der opstået et aktuelt problem som vedrører kapacitetsmangel, og konsekvenserne af dette problem bør undersøges nærmere.\fxnote{Denne sætning skal omformuleres så det ikke lyder som et aktuelt problem.} Ved at udnytte kapaciteten på afdelingen opnås der mere sundhed for pengene\cite{Company2013}. På baggrund af dette opstilles følgende initierende problem: \fxnote{Mere fokus på ortopædkirurgisk afdeling. Hvilken sundhedsrisiko er der ved kapacitetsmangel (fejl ved længere arbejdsdage + mortalitetsrate). Vi skal have skrevet noget om belægning i indledningen, så det passer til den initierende problemstilling}



\textit{Hvordan påvirkes ortopædkirurgisk afdeling på Aalborg Universitetshospital af ændringerne vedrørende kapacitetsudnyttelse og hvor udbredte er belægningsrelaterede problemer på afdelingen?} \fxnote{skriv så det fremgår at det er på daglig basis og ikke har noget med handleplanen at gøre.}


% Der er i dag overbelægning på flere afdelinger på de danske hospitaler, hvoraf nogle afdelingerne berøres i hele og flere måneder ad gangen. \cite{2015} Overbelægning resulterer i, at sundhedspersonalet får mindre tid pr. indlagt patient. Ifølge en undersøgelse fra Dansk Sygeplejeråd, mener hver anden regionalt ansat sygeplejerske på tværs af regionerne, at den travle arbejdsdag går ud over patienternes sikkerhed \cite{Kjeldsen2015}. Et studie påviser, at ved blot én ekstra indlagt patient pr. sygeplejerske øges mortalitetsraten med $7~\%$ indenfor en indlæggelse på 30 dage \fxnote{Har stadig lidt svært ved om man kan forstå sætningen på den rigtige måde.} \cite{Aiken2014}. 

% Foruden sundhedspersonalets øgede risiko for at begå fejl ift. behandlingen af patienter, er der ligeledes en sikkerhedsrisiko forbundet ved overbelægning på hospitalerne. De ekstra patienter, der ligger på stuerne, gangene og vaskerummene, pga. overbelægning, er en større udfordring ved evakuering under brand. Pladsmangel, som medfører, at patienterne opholder sig i vaskerummene og på gangene, bevirker desuden til, at patienterne oplever et skærpet privatliv. \cite{Madsen2014}

% Aalborg universitetshospital har i et tidligere projekt indsamlet data fra $1.000$ hospitalsindlæggelser. Disse data inkluderer blodprøveanalyser og knoglescanninger (DXA), hvilket formodes at kunne anvendes til udvikling af en prædiktiv model, der kan estimere indlæggelsesvarigheden blandt patienter. Denne rapport vil på baggrund af dette undersøge, hvorvidt det er muligt at forudsige indlæggelsesvarigheden ved brug af machine learning. \fxnote{mangler stadig noget om begrundelsen for valg af machine learning}



% \section{Initierende problemstilling} \fxnote{Dette er ikke den endelige problemstilling}
% Hvordan påvirkes personalet og patienterne af overbelægning på hospitaler, og hvilke konsekvenser har overbelægning i forhold til sikkerheden på afdelingerne?
% Hvilke typer af machine learning findes der, samt hvor benyttes det i dag?

