\chapter{Indledning}
Der ses stor overbelægning på flere afdelinger på de danske hospitaler. I år 2015 har flere afdelinger i Danmark haft overbelægning en hel måned, mens andre afdelinger ikke oplever overbelægning. \cite{ROLF}\fxnote{Mendelay virker ikke....(sundhedsdatastyrelsen)} Ved overbelægning er der påvist 9 $%$ øget risiko for dødelighed sammenlignet med underbelægningen. Forholdet mellem patient og sygeplejersker formindsket ydereligere, da sygeplejersker får mindre tid pr. patient ved overbelægning.\cite{dodelighed2014} 



Overbelægning på sygehuse samt forlængede arbejdsdage viser at have en negativ indvirkning på sundhedspersonale. Dette medfører en forhøget risiko for at lave fejl fra sundhedspersonalets side. Fejlene opstår hovedsagligt, når personalet har arbejdsdage på mere end 12 timer.\citep{forogelse2004} 

Overbelægning af patienter resulterer i, at sunhedspersonalet har kortere tid til de enkelte patienter, hvilket ligeledes øger risikioen for fejl. Dette resulterer i genindlæggelse og flere sengedage for patienterne. Herudover skaber dette en forringet kvalitet i forhold til operationer og medicinering.
