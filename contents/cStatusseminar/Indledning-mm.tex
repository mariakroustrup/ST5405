\section{Baggrund for projektet}
Projektet er stillet af Sten Rasmussen fra ortop�dkirugisk afdeling og Christian Kruse fra endokrinologisk afdeling. Form�let er at udarbejde en pr�diktiv model til at forudsige indl�ggelsesvarigheden p� ortop�dkirugisk afdeling baseret p� et eksisterende datas�t med 1.000 hospitalsindl�ggelser. 

\section{Indledning-ish}
Flere danske hospitalsafdelinger oplever i perioder at have flere patienter end der er kapacitet til. Dette kaldes overbel�gning og kan v�re et problem, afh�ngigt af afdelingen og varigheden.\cite{SDS2015} Den h�je bel�gningsgrad resulterer bl.a. i, at sundhedspersonalet f�r mindre tid pr. indlagt patient, hvilket kan medf�re gener for b�de personale og patient.\cite{Kjeldsen2015}

% ======== Er ikke sikker p� om denne del skal med eller ej
If�lge en unders�gelse fra Dansk Sygeplejer�d, mener hver anden regionalt ansat sygeplejerske p� tv�rs af regionerne, at den travle arbejdsdag g�r ud over patienternes sikkerhed\cite{Kjeldsen2015}.
% ========
\fxnote{Det kunne nok v�re fint med nogle �konomiske konsekvenser, men jeg synes ikke vi har noget solidt endnu. Ellers ved jeg ikke hvor meget der skal med af konsekvenser, da det initierende problem l�gger op til at analysen skal unders�ge dette, s� det er lidt underligt at ridse dem alle op nu.. }

\subsection{Inititerende problem(er) - udkast - slet subsection efter brug}
\subsubsection{Generel og uden ordet overbel�gning:}
\textbf{Hvor udbredte er bel�gningsrelaterede problemer p� ortop�dkirugisk afdeling p� Aalborg universitetshospital og hvad er konsekvensen af en bel�gningsgrad over, s�vel som under, $100$\%}

\subsubsection{Specifik og med ordet overbel�gning:}
\textbf{Hvor stort et problem er overbel�gning p� ortop�dkirugisk afdeling p� Aalborg universitetshospital og hvordan p�virkes personalet af dette}