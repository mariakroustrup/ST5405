\subsection{Problemafgrænsning}
I takt med at sengepladserne er reduceret fra $1996$ til $2001$ har dette forårsaget at antallet af disponible senge også er reduceret. Dette kan få betydning af antallet af akutte indlæggelser, da det ikke er muligt at estimere omfanget af disse. Dette medfører til, at der skabes overbelægning i perioder, da ortopædkirurgisk afdeling på nogle tidspunkter oplever flere patienter end afdelingen er normeret til. Som det er beskrevet i afsnit \ref{...} opleves der en belægning på over $100~\%$ på et tidspunkt i hver måned fra januar år $2014$ til juni år $2015$ på ortopædkirurgisk afdelingen. Det er dog kun 2 ud af de 18 måneder der har en gennemsnitlig belægning på over $100~\%$. Overbelægning skaber konsekvenser og er omkostningsfuldt.

Ved overbelægning sker der en omstrukturering af arbejdet for personalet med henblik på at finde en løsning mellem ressourcer og krav, dette gøres for, at sikre patientens behov, kvalitetssikring og udnyttelse af kompetencer. De ekstra patienter som sundhedspersonalet skal varetage skærper risikoen for fejl. Derudover påvirkes personalet af de længere arbejdsdage der kan være forbundet med overbelægning, hvilket yderligere medvirker til forringet kvalitet i behandlingen. Patienterne flyttes til andre stuer og opholdsrum pga. pladsmangel, hvilket kan belaste deres fysiske og psykiske forhold. Mortalitetsraten for patienter øges med $1,2~\%$ ved overskridelse af sengebelægningskapaciteten på $10~\%$. Det er nødvendigvis ikke  overbelægning der er den primære årsag til dette, men den har en indflydelse. Overbelægning er omkostningsfuldt og skaber nogle juridiske overvejelser ift. tilkaldes af brandvagter for skærpet sikkerhed samt overenskomster for sundhedspersonalet ved overarbejde samt hvis de tilkaldes ekstraordinært.

Da det forventes at antallet af akutte indlæggelser stiger, vil dette medfører at der i flere måneder vil være en belægning på over $100~\%$. For at kunne tage højde for de ekstra akutte paitenter kan en løsning være at forudsige indlægsgeletiden for patienterne både akutte samt elektive patienter hvorved det muliggør en bedre planlægning af indkaldelse af elektive patienter.

\subsubsection{Problemformulering}
Hvilket potentiale har en prædiktiv model som f.eks. machine learning til at forudsige indlæggelsestiden for patienter på ortopædkirurgisk afdeling på Aalborg Universitetshospital med henblik på at mindske overbelægning? 