\section{Forudsigelse af indlæggelsesvarighed}
Der er forskellige parametre, der har betydning for indlæggelsesvarigheden af patienter. Dette kan eksempelvis være demografiske parametre som alder og køn, samt kliniske parametre som blodprøver, blodtab og operationstype. Da der kan opstå komplikationer under operationen, som kan ændre indlæggelsesvarigheden, vurderes parametrene præ- og postoperativt.

\subsection{Præoperativt}

Dette afsnit baseres primært ud fra interviewet på ortopædkirurgisk afdeling på Aalborg universitetshospital. 


Hvilke parametre før en operation kan have indflydelse på varigheden af indlæggelsen? Er der nogle parametre som lægerne vægter højere end andre i estimeringen? 


Derudover vil vi via. litteratur, grafer og statistiske analyser understøtte ovenstående. Vi har på nuværende tidspunkt fundet frem til, at operationstypen, patienttypen (akut/elektiv), alder,  kombination af parametre og komorbiditeter har indflydelse. Hertil har vi også fundet ud af parametre som det ikke er påvist har en indflydelse på indlæggelsesvarigheden.


I dette afsnit vil vi gerne analysere på, hvorledes faktorer såsom ledige sengepladser og kirurger kan have indflydelse på patienters indlæggelsesvarighed. Ved begrænset sengepladser og kirurger vil patienters operationer eventuelt udskydes. 

