\section{Forudsigelse af indlæggelsesvarigheden} \label{forudsig}
Ved ortopædkirurgiske operationer er der flere parametre, der kan have indflydelse på indlæggelsesvarigheden. Dette kan eksempelvis være demografiske parametre såsom alder samt køn og kliniske parametre som blodprøver, blodtab og operationstype. Da der kan opstå komplikationer hhv. før, under og efter en operation, som kan have indflydelse på indlæggelsesvarigheden opdeles parametrene i præ- og postoperativt. 



\subsection{Præoperativt} \label{praeop}
På ortopædkirurgisk afdeling estimeres indlæggelsesvarigheden ikke på nuværende tidspunkt, dog vurderes indlæggelsesforløbet ud fra personalets erfaringer \ref{bilagA}. Ud fra informationspjecer fra ortopædkirurgisk afdeling på Aalborg Universitetshospital samt studier, der har påvist at demografiske faktorer, livsstilsfaktorer samt kliniske faktorer, kan medføre komplikationer under en operation\cite{Nordjylland2014}. Disse kan angiveligt medføre forlænget indlæggelsesvarighed.


\subsubsection{Demografiske}
Livsstilsfaktorer som eksempelvis alder kan tyde på at have betydning for komplikationer under en operation. En faktor som lavt funktionsniveau og muskeltab, som ofte er aldersbetinget, kan herunder medføre komplikationer og derved en længere indlæggelsesvarighed, hvorimod patienter med højt funktionsniveau ofte har kortere indlæggelsesvarighed.\cite{Kehlet2001, Janssen2002} Af \figref{alderogindlaeggelse} fremgår sammenhængen mellem patienters alder og indlæggelsesvarighed. 


\begin{figure}[H]
	%\flushleft 
	\centering
	\includegraphics[scale=0.3]{figures/alderogindlaeg}
	%\flushleft
	\caption{\textit{Sammenhængen mellem alder og indlæggelsesvarighed for patienter på ortopædkirurgisk afdeling. Alderen er opdelt i intervaller af 10 år. Dette er over en periode på 3 måneder fra den 1. august til den 31. oktober år 2014.}}
	\label{alderogindlaeggelse}
\end{figure}


\noindent
På \figref{alderogindlaeggelse} ses indlæggelsesvarigheden ift. alder. Det ses herpå, at indlæggelsesvarigheden for ældre patienter er længere end for yngre patienter på ortopædkirurgisk afdeling.  at der er en bred aldersfordeling på ortopædkirurgisk afdeling. Patienter under $30$ år er i gennemsnit indlagt 5 dage, hvor patienter over $30$ år i gennemsnit er indlagt 10,9 dage. Det er uvist, hvorvidt dette skyldes lavt funktionsniveau eller andre faktorer. Eksempelvis kan der være forskel på, hvilke operationtyper, der foretages på ældre frem for yngre patienter, hvilket kan medvirke til variation i indlæggelsesvarigheden.

For at teste for om der er signifikant sammenhæng mellem alder og indlæggelsesvarighed, foretages en two sample t-test. Resultaterne af denne test fremgår af \tabref{alderogindlaegtab}.

\begin{table}[H]
\centering
\begin{tabular}{|c|c|c|c|}
\hline
\multicolumn{1}{|l|}{\textbf{Alder (år)}} & \multicolumn{1}{l|}{\textbf{Sub-sample}} & \multicolumn{1}{l|}{\textbf{Indlæggelsesgennemsnit}} & \multicolumn{1}{l|}{\textbf{P-værdi}} \\ \hline
0-10                                      & 46                                       & 4,67                                                 & 0,0021*                               \\ \hline
11-20                                     & 55                                       & 4,56                                                 & 4,9727                                \\ \hline
21-30                                     & 36                                       & 5,89                                                 & 0,0280*                               \\ \hline
31-40                                     & 17                                       & 13,71                                                & 0,0483*                               \\ \hline
41-50                                     & 34                                       & 9,00                                                 & 0,4470                                \\ \hline
51-60                                     & 52                                       & 11,65                                                & 0,0637                                \\ \hline
61-70                                     & 91                                       & 12,04                                                & 0,0105*                               \\ \hline
71-80                                     & 73                                       & 10,95                                                & 0,0948                                \\ \hline
81-90                                     & 62                                       & 9,47                                                 & 0,4362                            \\ \hline
91-100                                    & 20                                       & 9,50                                                 & 0,4574                                \\ \hline
\end{tabular}
\caption{\textit{Statistik over sammenhæng mellem indlæggelsesvarigheden og alder. P-værdier under et signifikantsniveau på 5\% er angivet med *.}}
\label{alderogindlaegtab}
\end{table}

\noindent
Af \tabref{alderogindlaegtab} ses p-værdier for sammenhængen mellem indlæggelsesvarigheden og alder. Det fremgår, at aldergrupperne 0 til 10, 21 til 30, 31 til 40 år samt 61 til 70 år har en statistisk evidens for at have indflydelse på indlæggelsesvarigheden. Patienter der er mellem 0 år og 10 år samt 21 år og 30 år ligger i gennemsnit kortere tid end den gennemsnitlige patient. Derudover er patienter mellem 31 år og 40 år samt 61 og 70 år er indlagt længere tid end gennemsnittet. Det tyder derved på, at der er en sammenhæng mellem alder og indlæggelsesvarighed inden for disse aldersgrupper.


\subsubsection{Livsstilfaktorer}
Udover demografiske faktorer kan livsstilsfaktorer, såsom vægt, have en indflydelse på operationer foretaget på ortopædkirurgisk afdeling, da overvægt giver større risiko for blodpropper\cite{Ermonds2004}. Ved overvægt anbefales det derved at tabe sig før en eventuel operation. Foruden mindsket risiko ved opstående komplikationer, kan smerter ligeledes reduceres ved vægttab.\cite{Nordjylland2014} På \figref{BMIogindlaeggelse} fremgår sammenhængen mellem body mass index (BMI) og indlæggelsesvarigheden.

\begin{figure}[H]
	%\flushleft 
	\centering
	\includegraphics[scale=0.35]{figures/BMIogindlaeg}
	%\flushleft
	\caption{\textit{Sammenhæng mellem BMI og indlæggelsesvarighed for patienter på ortopædkirurgisk afdeling. BMI er opdelt i 6 kategorier: Undervægtig, normalvægtig, overvægt, fedme I, fedme II og fedme III. Dette er over en periode på 3 måneder fra den 1. august til den 31. oktober år 2014.}}
	\label{BMIogindlaeggelse}
\end{figure}

\noindent
Af \figref{BMIogindlaeggelse} fremgår det, at størstedelen af patienterne på ortopædkirurgisk afdeling befinder sig inden for vægtklassen normal og overvægtig. I disse klasser er indlæggelsesvarigheden mere varierende end de resterende klasser, hvor patienterne ses indlagt fra 0 til 61 dage. Dette kan skyldes, at samplestørrelsen i de resterende klasser er mindre, hvorfor der ved en større sample kan være samme variation. 
Den gennemsnitlige indlæggelsesvarighed for normalvægt og overvægt er 11,3 dage, mens den for fedmeklasse I, II og III i gennemsnit er 11,7 dage. Det tyder derved på, at et højere BMI er betydende ift. en længere indlæggelsesvarighed. 


For at teste om BMI har indflydelse på indlæggelsesvarigheden udføres en two sample t-test. Resultaterne heraf fremgår af \tabref{BMIindlaegtab}.

\begin{table}[H]
\centering
\begin{tabular}{|c|c|c|c|}
\hline
\textbf{BMI}         & \textbf{Sub-sample} & \textbf{Indlæggelsesgennemsnit} & \textbf{P-værdi} \\ \hline
Undervægt \textless18.5  & 19                  & 11,37                           & 0,4953           \\ \hline
Normalvægt 18.5 - 25     & 77                  & 11,48                           & 0,4581           \\ \hline
Overvægt 25-30           & 47                  & 11,13                           & 0,4432           \\ \hline
Fedme I 30-35            & 15                  & 12,40                           & 0,3357           \\ \hline
Fedme II 35-40           & 6                   & 8,00                            & 0,1899           \\ \hline
Fedme III \textgreater40 & 5                   & 14,60                           & 0,2180           \\ \hline
\end{tabular}
\caption{\textit{Statistik over sammenhæng mellem BMI og indlæggelsesvarigheden.}}
\label{BMIindlaegtab}
\end{table}

\noindent
Det fremgår af \tabref{BMIindlaegtab}, at der ikke er statistisk sammenhæng mellem BMI og indlæggelsesvarighed. Dette betyder angiveligt, at BMI er en parameter, som ikke har indflydelse på indlæggelsesvarigheden. 

Andre livsstilsfaktorer som eksempelvis rygning og alkoholindtag kan have betydning for komplikationer under operationer. Rygning kan være medvirkende til at knogler og sår heler langsommere. Ved operationer, hvor der skal transplanteres knoglevæv, f.eks. rygoperationer, afhænger resultatet af operation af, at knoglevævet heler rigtigt. Det anbefales derfor at stoppe med at ryge 6 måneder før en operation, hvorfor en ikke ryger defineres som en person, der har været røgfri i 6 måneder før den planlagte operation.\cite{Nordjylland2014} Fordelingen af ikke rygere, tidligere rygere og aktive rygere samt sammenhængen mellem, hvilken betydning dette har for indlæggelsesvarigheden fremgår af \figref{rygningogindlaeggelse}.


\begin{figure}[H]
	%\flushleft 
	\centering
	\includegraphics[scale=0.35]{figures/rygerogindlaeg}
	%\flushleft
	\caption{\textit{Sammenhæng mellem rygning og indlæggelsesvarighed for patienter på ortopædkirurgisk afdeling. Rygning er opdelt efter ikke ryger, tidligere ryger og aktiv ryger. Tidligere ryger defineres som røgfri i et halvt år. Patienterne er registrerede i periode fra den 1. august til den 31. oktober år 2014.}}
	\label{rygningogindlaeggelse}
\end{figure}


\noindent
Det ses af \figref{rygningogindlaeggelse}, at størstedelen af patienterne som indlægges på ortopædkirurgisk afdeling i den tre måneders periode er ikke rygere. Derudover fremgår det, at variationen i indlæggelsesvarigheden er større for ikke rygere og eks rygere end for aktive rygere. Den gennemsnitlige indlæggelsesvarighed for ikke rygere er 8,4 dage, hvor den for tidligere rygere og aktive rygere er henholdsvis 12,8 og 11 dage. Det tyder derfor på, at rygning har en betydning for indlæggelsesvarigheden. 

Dertil undersøges en statistisk sammenhæng mellem indlæggelsesvarigheden og rygning, hvortil der udføres en two sample t-test, som fremgår af \tabref{rygerindlaegtab}.

\begin{table}[H]
\centering
\begin{tabular}{|c|c|c|c|}
\hline
\textbf{Ryger} & \textbf{Sub-sample} & \textbf{Indlæggelsesgennemsnit} & \textbf{P-værdi} \\ \hline
Ikke ryger        & 296                 & 8,3649                          & 0,0324*          \\ \hline
Tidligere ryger         & 84                  & 12,8095                         & 0,0072*          \\ \hline
Aktiv ryger             & 123                 & 11,0325                         & 0,1186           \\ \hline
\end{tabular}
\caption{\textit{Statistik over sammenhæng mellem rygning og indlæggelsesvarighed. Tidligere ryger defineres som røgfri i et halvt år. P-værdier under et signifikantsniveau på 5\% er angivet med *.}}
\label{rygerindlaegtab}
\end{table}

\noindent
På \tabref{rygerindlaegtab} ses det, at aktiv rygning ikke har en indflydelse på indlæggelsesvarigheden, mens tidligere rygere og ikke rygere har en statistisk sammenhæng med indlæggelsesvarigheden. Tidligere rygere er i gennemsnit indlagt længere tid end den gennemsnitlige patient, mens ikke rygere modsat er indlagt kortere tid end den gennemsnittelige patient. Dette viser, at indlæggelsesvarigheden angiveligvis er kortere for patienter, der ikke ryger. 


Som tidligere nævnt kan alkoholindtag ligeledes medføre øget risiko for komplikationer under operationer. Herunder kan der opstå øget blødning under operationer samt infektioner i såret.\cite{Nordjylland2014} Sammenhængen mellem alkoholindtag og indlæggelsesvarighed er illustreret på  \figref{alkohologindlaeggelse}. 


\begin{figure}[H]
	%\flushleft 
	\centering
	\includegraphics[scale=0.4]{figures/alkohologindlaeg}
	%\flushleft
	\caption{\textit{Sammenhæng mellem alkoholindtag og indlæggelsesvarighed for patienter på ortopædkirurgisk afdeling. Patienter er opdelt efter Sundhedsstyrelsens anbefaling af alkoholindtag, herunder inden for anbefalet og over anbefalet. Dette er over en periode på 3 måneder fra den 1. august til den 31. oktober år 2014.}}
	\label{alkohologindlaeggelse}
\end{figure}

\noindent
Det fremgår af \figref{alkohologindlaeggelse}, at størstedelen af patienter på ortopædkirurgisk afdeling befinder sig inden for anbefalingen af alkoholindtag. Sundhedsstyrelsen anbefaling er 7 genstande for kvinder og 14 genstande for mænd om ugen \cite{Sundhedsstyrelsen2016}.


Den gennemsnitlige indlæggelsesvarighed for patienter, der holder sig inden for det anbefalede grænse er 10 dage, mens den for patienter, der drikker over anbefalet er indlagt 8,4 dage. Det tyder derfor på, at patienter, der drikker inden for den anbefalede grænse er indlagt i længere tid end patienter, der drikker over det anbefalede alkoholindtag.
For at kunne bekræfte denne sammenhæng udføres en two sample t-test. Resultatet af denne test fremgår af \tabref{alkoholindlaegtab}


\begin{table}[H]
\centering
\begin{tabular}{|c|c|c|c|}
\hline
\textbf{Alkohol} & \textbf{Sub-sample} & \textbf{Indlæggelsesgennemsnit} & \textbf{P-værdi} \\ \hline
Under anbefalet   & 443                 & 10,01                           & 0,4154           \\ \hline
Over anbefalet    & 47                  & 8,45                            & 0,1882           \\ \hline
\end{tabular}
\caption{\textit{Statistik over sammenhæng mellem alkoholindtag og indlæggelsesvarigheden. Patienter er opdelt efter Sundhedsstyrelsens anbefaling af alkoholindtag, herunder inden for anbefalet og over anbefalet.}}
\label{alkoholindlaegtab}
\end{table}

\noindent
Det fremgår af \tabref{alkoholindlaegtab}, at der ingen statistisk sammenhæng er mellem alkoholindtag og indlæggelsesvarighed. Dette kan skyldes, at det er en subjektiv vurdering, hvorfor egen holdning kan have indflydelse på patientens svar. 

Det tyder på, at livsstilsfaktorer som komorbiditeter ligeledes kan have betydning for indlæggelsesvarigheden. Patienter med komorbiditeter som eksempelvis dårligt blodomløb og diabetes, kræver ofte særlig behandling, hvilket kan medføre længere indlæggelsesvarighed \ref{bilagA}. Diabetes har bl.a. indvirkning på heling af sår \ref{bilagA}. Af \figref{diabetesogindlaeggelse} fremgår sammenhængen mellem diabetes og indlæggelsesvarigheden. 

\begin{figure}[H]
	%\flushleft 
	\centering
	\includegraphics[scale=0.25]{figures/diabetesogindlaeg}
	%\flushleft
	\caption{\textit{Sammenhæng mellem diabetes og indlæggelsesvarighed for patienter på ortopædkirurgisk afdeling. Diabetes er opdelt i hendholdsvis ikke diabetes, type I og type II. Patienterne er registrerede i periode fra den 1. august til den 31. oktober år 2014.}}
	\label{diabetesogindlaeggelse}
\end{figure}

\noindent
På \figref{diabetesogindlaeggelse} ses det, at type I diabetikere har en længere indlæggelsesvarighed end type II diabetikere. Den gennemsnitlige indlæggelsesvarighed for ikke diabetikere er 7,8 dage, mens den for type I og type II diabetikere er henholdsvis 18,8 og 12,1 dage. Det tyder derfor på, at der er en sammenhæng mellem diabetes og indlæggelsesvarighed. 

Denne sammenhæng undersøges ved en two sample t-test. Resultatet heraf fremgår af \tabref{diabetesindlaegtab}

\begin{table}[H]
\centering
\begin{tabular}{|c|c|c|c|}
\hline
\textbf{Diabetes}  & \textbf{Sub-sample} & \textbf{Indlæggelsesgennemsnit} & \textbf{P-værdi} \\ \hline
Ikke diabetiker    & 415                 & 7,85                            & 0,0159*          \\ \hline
Type I diabetiker  & 57                  & 18,77                           & 5,2136           \\ \hline
Type II diabetiker & 14                  & 12,07                           & 0,1582           \\ \hline
\end{tabular}
\caption{\textit{Statistik over sammenhæng mellem indlæggelsesvarigheden og diabetes. Diabetes er opdelt i hendholdsvis ikke diabetes, type I og type II. P-værdier under et signifikantsniveau på 5\% er angivet med *.}}
\label{diabetesindlaegtab}
\end{table}

\noindent
Det fremgår af \tabref{diabetesindlaegtab}, at der er en statistisk sammenhæng mellem ikke diabetikere og indlæggelsesvarigheden. Modsat ses ingen statistik sammenhæng mellem diabetes og indlæggelsesvarighed. Ikke diabetikere er i gennemsnit indlagt kortere tid end den gennemsnitlige patient. Dertil ses en statistisk sammenhæng mellem ikke diabetikere og indlæggelsesvarigheden.

\subsubsection{Kliniske faktorer}
Udover livsstilsfaktorer tyder det på, at kliniske faktorer kan have en indflydelse på indlæggelsesvarigheden. Som beskrevet i \ref{kap_OA} udføres flere operationstyper med forskellig tidsvarighed på ortopædkirurgisk afdeling, hvorfor det forventes, at indlæggelsesvarigheden ligeledes kan være varierende. 


\begin{figure}[H]
	%\flushleft 
	\centering
	\includegraphics[scale=0.5]{figures/operaogindlaeg}
	%\flushleft
	\caption{\textit{Sammenhæng mellem operationstyper og indlæggelsesvarighed for patienter på ortopædkirurgisk afdeling. Operationstyper er opdelt i albue og underarm, hånd og håndled, ryg og hals, knæ og underben, ankel og fod samt hofte og lår. Operationerne er registrerede i periode fra den 1. august til den 31. oktober år 2014.}}
	\label{opvsindlaegtid}
\end{figure}


\noindent
På \figref{opvsindlaegtid} fremgår det, at indlæggelsesvarigheden er varierende for typen af operationer, hvoraf den gemmesnitlige indlæggesesvarighed varer fra 2,7 til 12,5 dage. Den gennemsnitlige indlæggelsesvarighed er for ryg og hals, hofte og lår, knæ og underben samt ankel og fod, henholdsvis 12,5, 10,8, 10,2 og 10,1 dage. Modsat er den gennemsnitlige indlæggelsesvarighed for hånd og håndled samt albue og underarm 3,8 og 2,7 dage. Hertil tyder det på, at indlæggelsesvarigheden er længere for operationer foretaget på underekstremiteter sammenlignet med overekstremiteter. Derudover ses det, at de fleste operationer, der foretages på ortopædkirurgisk afdeling er hofte og lår samt knæ og underben operationer. 

For at teste om der er signifikant sammenhæng mellem operationstypen og indlæggelsesvarighed, foretages en two sample t-test. Resultaterne af denne test fremgår af \tabref{optypeogindlaegtab}.

\begin{table}[H]
\centering
\begin{tabular}{|c|c|c|c|}
\hline
\textbf{Operationstype} & \textbf{Sub-sample} & \textbf{Indlæggelsesgennemsnit} & \textbf{P-værdi} \\ \hline
Hånd og håndled   & 34                  & 3,74                            & 5,9083           \\ \hline
Albue og underarm & 50                  & 2,72                            & 0,0011*          \\ \hline
Ankel og fod      & 49                  & 10,10                           & 0,2964           \\ \hline
Knæ og underben   & 100                 & 10,20                           & 0,2015           \\ \hline
Hofte og lår      & 157                 & 10,78                           & 0,0532 \\ \hline
Ryg og Hals       & 37                  & 12,46                           & 0,0378*          \\ \hline
\end{tabular}
\caption{\textit{Statistik over sammenhæng mellem indlæggelsesvarigheden og operationstype. Operationstyper er opdelt i albue og underarm, hånd og håndled, ryg og hals, knæ og underben, ankel og fod samt hofte og lår. P-værdier under et signifikantsniveau på 5\% er angivet med *.}}
\label{optypeogindlaegtab}
\end{table}

\noindent
Af \tabref{optypeogindlaegtab} fremgår det, at  operationer af albue og underarm samt ryg og hals har statistisk sammenhæng med indlæggelsesvarigheden. Patienter, der har fået foretaget en albue- og underarmsoperationer er statistisk indlagt kortere tid end den gennemsnitlige patient. Herimod patienter, der har fået foretaget  ryg- og halsoperationer er indlagt længere tid end den gennemsnitlige patient.


\subsubsection{Tilgængelige kirurger}
Foruden livsstilsfaktorer og operationstype kan tilgængeligheden af kirurger have indflydelse på indlæggelsesvarigheden for patienterne. Det kan i nogle tilfælde være nødvendigt at udsætte elektive patienters operationer, hvis et akut tilfælde opstår, hvorved kirurgen skal være til stede. Derved kan den estimerede indlæggelsesvarighed for patienterne forlænges. Ved en aflysning af elektive patienter vil vedkommende ofte få en ny tid hurtigst muligt. Den nye tid vil typisk være først på dagen, således risikoen for endnu en udsættelse mindskes.\ref{bilagA}