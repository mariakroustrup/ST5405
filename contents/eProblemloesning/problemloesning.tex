\chapter{Problemløsning}
På ortopædkirurgisk afdeling på Aalborg Universitetshospital ønskes der, på baggrund af afsnit \ref{kap}, en $100~\%$ kapacitetsudnyttelse med henblik på at udnytte de tilgængelige ressourcer optimalt. Kapacitetsudnyttelse er forholdet mellem aktivitet og kapacitet, hvoraf antal patienter er en betydende faktor ift. aktivitet. 
Det fremgår af afsnit \ref{omfang}, at belægningsgraden på ortopædkirurgisk afdelingen er varierende for hver måned. Ved en belægning over $100~\%$ vil afdelingen opleve kapacitetsmangel, hvilket vil medføre, at personalet skal yde mere end afdelingen har kapacitet til. Derimod vil en belægning under $100~\%$ forårsage, at der ikke er fuld udnyttelse af personalets arbejdskraft. Derved oplever ortopædkirurgisk afdeling en ubalance i kapacitetsudnyttelsen. 
For at opnå en kapacitetsudnyttelse på $100~\%$, skal en ligevægt mellem aktivitet og kapacitet forekomme. Denne ligevægt kan tilnærmes ved at justere på blot én af faktorerene under aktivitet eller kapacitet\cite{Bames2015}. Det kan herunder forsøges at effektivere planlægningen af patienter og dertil forsøge at estimere indlæggelsesvarigheden for de indlagte patienter. 
På baggrund af dette opstilles følgende hypotese:\\

\noindent
\textit{Indlæggelsesvarigheden for patienter kan benyttes til at effektivisere kapacitetsudnyttelsen.} 

\section{Forudsigelse af indlæggelsesvarigheden}
For at forbedre kapacitetsudnyttelsen undersøges det, hvorvidt indlæggelsesvarigheden kan forudsiges.
%Det fremgår af AFSNIT\fxnote{ref til afsnit}, at ortopædkirugisk afdeling på Aalborg Universitetshospital ikke fastholder en kapacitetsudnyttelse på $100$\%.\fxnote{snak sammen med Linette og Maria} 
Aalborg Universitetshospital har i et tidligere projekt indsamlet data fra $970$ hospitalsindlæggelser fra ortopædkirurgisk afdeling. Datasættet er indsamlet gennem digitale patientjournaler i perioden 1. august til 31. oktober år 2014. Disse data er fordelt på 78 forskellige parametre. %Flere af parametrene for patienterne er ukendte, hvorfor præprocessering af datasættet anses nødvendigt. Ud fra parametrene i datasættet vil indlæggelsesvarigheden forsøges estimeres. 

På baggrund af hypotesen kan en effektivisering af kapacitetsudnyttelse på ortopædkirurgisk afdeling på Aalborg Universitetshospital medvirke til en bedre planlægning. Dette opnås ved at mindske ubalancen mellem aktivitet og kapacitet. Ved denne balance opnås en bedre planlægning, der kan resultere i bedre strukturering af personalets arbejdsopgaver og bedre udnyttelse af disponible sengepladser. 





%Det forventes, at en forudsigelse af patienters indlæggelsesvarighed på ortopædkirurgisk afdeling på Aalborg Universitetshospital, vil kunne optimere planlægning af patienter. 
%Ved en optimering af planlægningen, vil det være muligt at tilnærme en kapacitetsudnyttelse 100\%. 
%Dermed vil ubalancen ift. aktivitet og kapacitet mindskes. En forudsigelse af indlæggelsesvarigheden kan skabe større overblik over patientindlæggelserne, hvorved der muligvis kan behandles flere patienter. Herudover vil en forudsigelse af indlæggelsesvarigheden sikre en optimal udnyttelse af disponible sengepladser. En bedre planlægning af patientindlæggelsen vil derfor kunne medvirke til balance i kapacitetsudnyttelsen. 
