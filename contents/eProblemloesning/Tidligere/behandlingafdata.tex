\subsection{Indsamling af data}
Den prædiktiv model skal designes specifikt til ortopædkirurgisk afdeling på Aalborg Universitetshospital, da der er bestemte parametre som f.eks. operationer, procedurer, udstyr, personale og medicinering. For at sikre, at indsamling af data er tilstrækkelig og relevant skal der opstilles stramme retningslinjer for, hvordan disse indsamles og noteres. På baggrund af afgrænsningen til supervised learning i afsnit \ref{prae_model}, anses det nødvendigt, at hver indsamlede parameter kan have indflydelse for den enkelte patients indlæggelsesforløb. Disse overvejelser øger sandsynligheden for at opnå en brugbar prædiktiv model. 


%\subsection{Vurdering af parametre}
%Antallet af parametre vurderes grundlæggende ud fra parametrenes betydning for bestemmelse af indlæggelsesvarigheden. Flere parametre og mængden af data øger databehandlingstiden samtidig med, at modellen optager mere hukommelse i den anvendte computer.\cite{Kuhn2013}
%%Hvis computerkraft og budget ikke er den afgørende begrænsning i konstruktionen af en prædiktiv model, er det fordelagtigt at anvende så mange parametre som muligt for således at opnå en bedre estimering af indlæggelsesvarigheden.


%Da parametre kan angives som objektive og subjektive er det nødvendigt at finde en kvantificerbar skala til notering af parametrene. Objektive parametre som køn, alder og blodtryk angives med en direkte værdi. Subjektive parametre, som fysisk aktivitet og livsstil eksempelvis rygning, der oplyses af patienten. 
%


\subsubsection{Eksklusionskriterier for data} 
For at udarbejde en prædiktiv model opstilles kriterier ift. dataindsamling ud fra formålet med modellen. Da nogle modeller er sensitive for strømlinet data er det vigtigt at bestemme, hvorvidt der skal opstilles eksklusionskriterier til dataindsamlingen eller om en forudbestemt prædiktiv model skal anvendes.\cite{Kuhn2013}
For at systemet kan sammenholde parametre er det nødvendigt, at data er indskrevet efter faste retningslinjer. Hvis der ikke opstilles faste retningslinjer for indskrevet data, kan dette have indflydelse på estimeringen af indlæggelsesvarigheden.\cite{Kuhn2013} \\

\noindent
\textbf{Kategorisering af data} \\
\noindent
Ved at kategorisere data, anses det muligt at lave en prædiktiv model med en mindre mængde data. Denne model er dog ikke så specifik som en model uden kategoriseret data, da en mulig variation i parametrene kan reduceres.\cite{Rowan2007} 
Derfor bør det overvejes, hvor stor datamængden skal være for at konstruere modellens træningssæt. Ved ikke at kategorisere data, bliver modellen mere specifik ift. hver enkelt patient, men kræver dertil også en større database for at lave en funktionel model. Dette kan f.eks. inkludere meget specifikke kirurgiske indgreb eller sjældne komorbiditeter. 
Et eksempel på kategorisering af data kan være alder, hvor denne kan inddeles i aldersgrupper. En gruppe kan eksempelvis være 20-29 år hvilket kræver en mindre database, end aldersinddelingen 20-25 år, der udspænder et mindre interval.\cite{Rowan2007}  

\subsubsection{Opdatering af model}
En vigtig del af en prædiktiv model er, at denne kan tilpasse ændringer i parametres vægtning løbende ved ny data.\cite{Kuhn2013} Ved ændring i medicinering eller procedure af kirurgiske indgreb kan prædiktering af det gældende indgreb give misvisende estimateringer af indlæggelsesvarigheden. Derfor kan modellen ved transitioner mellem procedureændringer blive invalideret og dermed skal den prædiktive model have inkorporeret mere data før denne kan forudsige indlæggelsesvarigheden. 
Hvis et datapunkt ikke kan kategoriseres i systemet, kan det være nødsaget at ekskludere denne data fra indskrivelse i databasen.


\subsubsection{Præprocessering} \label{praeproc}
Hvis en datamængde allerede er tilgængelig, eksempelvis indsamlet til et andet formål, kan det være nødvendigt at anvende præprocessering på dataen. Dette gøres for at tilpasse den allerede indsamlede data til det nye formål.
Da flere parametre i datasættet fra ortopædkirurgisk afdeling på Aalborg Universitetshospital mangler, anses det nødvendigt at foretage præprocessering. Præprocessering foregår manuelt før en prædiktiv modellering kan foretages.
Der findes flere metoder, der kan anvendes for at kompensere for de manglende parametre. Kompenseringen kan forekomme ved kassering af værdier, tilegne manglende værdier, reducering af værdier og imputering. Imputering opdeles i tre underkategorier: Prædiktiv værdi imputation, Distribution-baserede imputation og Unik-værdi imputation. Prædiktiv værdi imputation erstatter den manglende værdi med estimerede værdier. Distribution-baserede imputation vægter værdien af manglende data mindre end det resterende, således disse får en mindre betydning under generering. Unik-værdi imputation erstatter den manglede værdi med en vilkårlig værdi fra samme parameter.\cite{Saar2007} 

Manglende data kan have flere årsager. Én årsag kan være manglende indsamlig af data grundet fejl. Manglende data kan også opstå som følge af andre faktorer. F.eks. kan udskrivelses tidspunkt være en manglende parameter hvis den pågældende person er død under indlæggelsen. Ligeledes kan manglende data være grundet censur, hvilket giver udtryk for en skjult ekstra parameter, eller som følge af en allerede kendt parameter.\cite{Kuhn2013} Dette fænomen er kendt som informative missingness. Da informative missingness betyder at manglende data er opstået som følge af en bestemt årsag, kan det være relevant ikke at imputere dataen, men derimod beholde den pågældende data. Derfor bør årsagen til hver manglende parameter overvejes, inden det besluttes hvorvidt der anvendes imputering, informative missingness eller en kombination af disse.