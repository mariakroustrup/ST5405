\chapter{Problemløsning}
På ortopædkirurgisk afdeling på Aalborg Universitetshospital ønskes der, på baggrund af afsnit \ref{kap}, en $100~\%$ kapacitetsudnyttelse med henblik på at udnytte de tilgængelige ressourcer optimalt. Kapacitetsudnyttelse er forholdet mellem aktivitet og kapacitet, hvoraf antal patienter er en betydende faktor ift. aktivitet. 
Det fremgår af afsnit \ref{omfang}, at belægningsgraden på ortopædkirurgisk afdelingen er varierende for hver måned. Ved en belægning over $100~\%$ vil afdelingen opleve kapacitetsmangel, hvilket vil medføre, at personalet skal yde mere end afdelingen har kapacitet til. Derimod vil afdelingen ved en belægning under $100~\%$ forårsage, at der ikke er fuld udnyttelse af personalets arbejdskraft. Derved oplever ortopædkirurgisk afdeling en ubalance i kapacitetsudnyttelsen. 
For at opnå en kapacitetsudnyttelse på $100~\%$, skal en ligevægt mellem aktivitet og kapacitet forekomme. Denne ligevægt kan tilnærmes ved at justere på en af faktorerene under aktivitet og kapacitet\cite{Bames2015}. Det kan herunder forsøges at effektivere planlægningen af patienter og dertil forsøge at estimere indlæggelsesvarigheden for de indlagte patienter. 
På baggrund af dette opstilles følgende hypotese:\\

\noindent
%\begin{itemize}
%\item 
Indlæggelsesvarigheden for patienter kan benyttes til at effektivisere kapacitetsudnyttelsen. 
%\end{itemize}

\section{Forudsigelse af indlæggelsesvarigheden}
For at forbedre kapacitetsudnyttelsen på ortopædkirurgisk afdelingen undersøges det, hvorvidt indlæggelsesvarigheden kan forudsiges.
%Det fremgår af AFSNIT\fxnote{ref til afsnit}, at ortopædkirugisk afdeling på Aalborg Universitetshospital ikke fastholder en kapacitetsudnyttelse på $100$\%.\fxnote{snak sammen med Linette og Maria} 
Aalborg Universitetshospital har i et tidligere projekt indsamlet data fra $970$ hospitalsindlæggelser fra ortopædkirurgisk afdeling. Datasættet er indsamlet gennem digitale patientjournaler i perioden 1. august til 31. oktober år 2014. Disse data er fordelt på 78 forskellige parametre. Flere af parametrene for patienterne er ikke ukendte, hvorfor præprocessering af datasættet anses nødvendigt. Ud fra parametrene fra datasættet vil indlæggelsesvarigheden forsøges estimeres. 


\subsubsection{Prædiktiv model}
\noindent
Til at estimere indlæggelsesvarigheden for patienter kan en prædiktiv model anvendes.
Prædiktiv modellering definerer en proces, hvor en model udarbejdes med henblik på at forudsige hændelser. Denne model skal gøre det muligt at forstå og kvantificere nøjagtigheden af modellens forudsigelser ift. fremtidig data.\cite{Kuhn2013} Denne kvantificering sker på baggrund af algoritmer. 
Inden for sundhedssektoren er det muligt at prædiktere forskellige former for hændelser og forløb ved anvendelse af disse modeller. Dette kan eksempelvis være en forudsigelse om, hvorvidt en patient, indlagt med hjertestop, har risiko for endnu et hjertestop, hvoraf vurderingen f.eks. baseres på demografi, kost samt kliniske målinger\cite{Hastie2008}. %Et andet eksempel herpå kan være en estimering af glukosemængden i blodet hos en diabetiker, hvilket baseres på det infrarøde absorptionsspektrum af patientens blod.\cite{Hastie2008}

\noindent
Prædiktiv modellering kan opdeles i de to kategorier parametrisk og ikke-parametrisk. Forskellen på disse to kategorier ses overordnet ved om antallet af parametre kan varieres samt, hvorvidt alle parametre er kendte. Da flere af parametrene i datasættet ikke er kendte bør der anvendes ikke-parametrisk modellering.\cite{Sheskin2000}

\noindent
%De to katagorier kan opbygges som en matematisk model eller en computerbaseret databehandlingsmodel. %(på engelsk: computational models)
%Matematiske modeller er mindre optimale at avende ved ikke-parametriske datasæt, da \fxnote{HVORFOR? Dette skal uddybes}. 
%Matematiske modeller 
%Computerbaseret modeller kan derimod være vanskelige at opstille ligninger for, da 
%Af denne årsag fokuserer projektet fremadrettet på computerbaseret databehandling, da denne egner sig bedst til det tilgængelige ikke-parametriske datasæt\fxnote{HVORFOR? Dette skal uddybes}.\cite{Sheskin2000} 

Matematiske modeller:

Computerbaseret modeller: 