\chapter{Problemløsning}
På ortopædkirurgisk afdeling på Aalborg Universitetshospital ønskes der, på baggrund af afsnit \ref{kap}, en $100~\%$ kapacitetsudnyttelse med henblik på at udnytte de tilgængelige ressourcer optimalt. Kapacitetsudnyttelse er forholdet mellem aktivitet og kapacitet, hvoraf belægningsgraden er en betydende faktor ift. aktivitet. Det fremgår af afsnit \ref{omfang}, at belægningen på ortopædkirurgisk afdelingen er varierende, herunder opleves en ubalance i kapacitetsudnyttelse. Ved en belægning over $100~\%$ vil afdelingen opleve kapacitetsmangel, hvilket vil betyde at personalet skal yde mere end afdelingen har kapacitet til. Derimod vil afdelingen ved en belægning under $100~\%$ forårsage, at der ikke er fuld udnyttelse af personalets arbejdskraft. Da der ses ubalance i kapacitetsudnyttelsen og belægning kan dette tyde på, at der skal forekomme en effektivisering af planlægningen af patienter på ortopædkirurgisk afdeling. På baggrund af dette opstilles følgende hypotese:\\

\noindent
%\begin{itemize}
%\item 
Indlæggelsesvarigheden for patienter kan benyttes til at effektivisere kapacitetsudnyttelsen. 
%\end{itemize}

\subsection{Prædiktiv model}
For at forbedre kapacitetsudnyttelsen på ortopædkirurgisk afdelingen undersøges det, hvorvidt en prædiktiv model kan anvendes.
%Det fremgår af AFSNIT\fxnote{ref til afsnit}, at ortopædkirugisk afdeling på Aalborg Universitetshospital ikke fastholder en kapacitetsudnyttelse på $100$\%.\fxnote{snak sammen med Linette og Maria} 
Dette forventes muligt ved en forudsigelse af patienternes indlæggelsesvarighed. Aalborg Universitetshospital har i et tidligere projekt indsamlet data fra $1.000$ hospitalsindlæggelser fra ortopædkirurgisk afdeling. Disse data inkluderer bl.a. blodprøveanalyser og knoglescanninger (DXA), hvilket formodes at kunne anvendes til udvikling af en prædiktiv model. 

\noindent
Prædiktiv modellering definerer en proces, hvor en model udarbejdes med henblik på at forudsige hændelser. Denne model skal gøre det muligt at forstå og kvantificere nøjagtigheden af modellens forudsigelser ift. fremtidig data.\cite{Kuhn2013} 
Inden for sundhedssektoren er det muligt at prædiktere forskellige former for hændelser og forløb ved anvendelse af disse modeller. Dette kan eksempelvis være en forudsigelse om, hvorvidt en patient, indlagt med hjertestop, har risiko for endnu et hjertestop. Dette baseres på demografi, kost samt kliniske målinger. Et andet eksempel herpå kan være en estimering af glukosemængden i blodet hos en diabetiker, hvilket baseres på det infrarøde absorptionsspektrum af patientens blod.\cite{Hastie2008}

\noindent
Prædiktiv modellering kan opdeles i de to kategorier parametrisk og ikke-parametrisk. Forskellen på disse to kategorier ses overordnet ved om antallet af parametre kan varieres. I datasættet anvendt i projektet benyttes forskellige antal parametre for flere datapunkter og er derfor ikke-paramtetrisk.\cite{Sheskin2000}

\noindent
En prædiktiv model kan opbygges som en matematisk model eller en computerbaseret databehandlingsmodel (på engelsk: computational models). De matematiske modeller er mindre optimale at avende ved ikke-parametriske datasæt\fxnote{uddyb lidt måske}. Af denne årsag fokuserer projektet fremadrettet på computerbaseret databehandling, da denne egner sig bedst til det tilgængelige ikke-parametriske datasæt. \cite{Sheskin2000} 
