\subsection{Postoperativt}

\subsubsection{Afhentning af patienter}
På baggrund af interview på ortopædkirurgisk afdeling, se bilag \ref{interview1}, fremgår det, at en betydende faktor for udskrivning af patienter, er afhentning af disse. I nogle tilfælde kan patienter, der er klar til at blive udskrevet, være nødsaget til at ligge en dag ekstra på afdelingen. Dette er tilfældet for patienter, der ikke selv har mulighed for at forlade matriklen eller har brug for pleje i hjemmet. Denne service afhænger af samarbejdet med resten af kommunen og udskrivelsestidspunktet. Ifølge interviewet skal kommunen kontaktes før klokken 12, hvis disse patienter skal afhentes og have hjælp. Hvis dette ikke er muligt og patienten er nødsaget til at blive på afdelingen i endnu et døgn, kan dette medføre kapacitetsproblemer. 
Ligeledes kan der opstå et problem, hvis patientens hjem ikke er egnet til den opfølgende behandling eller genoptræning. Dertil kan det kirurgiske indgreb, udført på patienten, have en indvirkning på eksempelvis demens, hvilket kan føre til at patienten mister evnen til at varetage sit eget helbred. Sker dette, kræves det, at patienten overføres til aflastningsplads. Ved mangel på aflastningspladser opstår der ligeledes en situation, hvor patienten er nødsaget til at blive på afdelingen, selvom vedkommende er udskrivningsklar. 


