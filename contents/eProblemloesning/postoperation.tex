\subsection{Postoperativt}
Udover de faktorer der kan have indflydelse på indlæggelsesvarigheden  præoperativt er der faktorer der kan have indflydelse på indlæggelsesvarigheden postoperativt. Dette kan være komplikationer der opstår under eller efter operation, som kan medvirke til en forlænget indlæggelsesvarighed for patienten. Da ortopædkirurgisk afdeling ikke estimere indlæggelsesvarigheden på nuværende tidspunkt er der taget udgangspunkt i informationspjecer fra ortopædkirugisk afdeling på Aalborg Universitetshospital samt studier.

\subsubsection{Komplikation under operation}


\subsubsection{Varigheden af operationen}





\subsubsection{Bevægelighed}
Det tyder på, at bevægelighed efter operation er vigtigt for, at komme sig hurtigt efter operationen. Da dette er med til at mindske risikoen for komplikationer og smerter efter operationen. 





\subsubsection{Afhentning af patienter} 
En betydende faktor for længere indlæggelsesvarighed er udskrivning af patienter. I nogle tilfælde kan patienter, der er klar til at blive udskrevet være nødsaget til at ligge en dag ekstra på afdelingen, da de har brug for pleje i hjemmet efter operationen. Hvis det er nødvendigt med hjemmepleje kræver det, at afdelingen kontakter kommunen før kl. 12 samme dag. Hvis personalet ikke når dette skal patienten blive på afdelingen endnu et døgn. 

Derudover kan nogle patienter, der skal have efterbehandling eller genoptræning,  nødsaget til at blive på afdelingen, da deres hjem ikke er tilpasset til dette. 

Dette kan også gælde patienter, som i forvejen har komorbiditeter, hvor forandringer grundet operationen og indlæggelsen forværrer deres tilstand. Dette kan eksempelvis være demens, hvor patienten mister evnen til at varetage sit eget helbred. Disse patienter skal ofte vente på at der er en aflastningsplads ledig før disse kan udskrives fra afdelingen. 


