\subsection{Postoperativt}
Den planlagte indlæggelsesvarighed for patienter kan ændre sig efter en operation, hvis der opstår komplikationer både løbende samt postoperativt. Hertil kan de opståede komplikationer tilføjes den prædiktive model, således det er muligt at estimere indlæggelsesvarigheden igen. Det ses, at patienter med forlænget indlæggelsesvarighed oftest er ældre, har flere komorbiditeter samt været igennem uforventede komplikationer.\cite{Krell2014} Herudover ses en sammenhæng mellem længere operationstid og flere indlæggelsesdage.\cite{Kjeldsen2015b}



\subsubsection{Ressourcemangel postoperativt}
Det kan i nogle tilfælde være nødvendigt at udsætte elektive patienters operationstid, da der ikke er kirurger til rådighed inden for specialet samt, hvis der sker noget akut, hvor antallet af akutte patienter er stigende. Ved udsættelse af elektive patient vil denne ofte få en ny tid hurtigst muligt. Den nye tid vil være først på dagen, da operationer først på dagen sjældent aflyses pga. akutte tilfælde. Dette gøres for at undgå yderligere udsættelse af elektive patienters operationer. Da operationen for elektive patienter i nogle tilfælde kan være nødvendige at udsætte kan dette betyde at indlæggelsesvarigheden for disse patienter forøges, hvis patienterne i forvejen er indlagt, da de skal vente på en ny operationstid eller i værre tilfælde kan den elektive patient ændre status fra elektiv til akut grundet ventetiden. \ref{tidligere afsnit om elektive og akutte}

%Vil vil undersøge, hvilke komplikationer under og efter en operation, samt hvilken betydning dette kan have på indlæggelsesvarigheden. Herunder kobling mellem alder, operations tid og diabetes. 

%Herunder vil vi undersøge sammenhængen mellem operations- og indlæggelsesvarigheden. Dette vil vi blive illustreret med en figur.


%En tidligere rehabilitering af patienter kan have indflydelse på indlæggelsesvarigheden. Dertil kan smerte ligeledes have indflydelse på indlæggelsesvarigheden.


%Der kan ligeledes være behov for pleje efter en indlæggelse, hvorfor en hjemmepleje kan være nødvendig. Det kan være der ikke er hjemmepleje til rådighed på det given tidspunkt, hvorfor patienterne bliver nødsaget til at være indlagt i længere tid. 