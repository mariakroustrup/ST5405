\chapter{Analyse af prædiktiv model}
På nuværende tidspunkt har OA ikke et redskab, der kan forudsige indlæggelsesvarigheden mhp. at planlægge patienter. Dette kapitel analyserer, hvilken effekt et planlægningsredskab vil kunne medføre  på afdelingen. En mulig løsning til sådan et redskab kan være at udarbejde en prædiktiv model. 
%Da der på nuværende tidspunkt ikke er et redskab til at forudsige indlæggelsesvarigheden er det ikke muligt at planlægge, hvornår der skal indkaldes elektive patienter ift. kapaciteten på OA. Sundhedspersonalet har dog en idé om, hvor længe patienterne er indlagt ift. operationstype, men dette tages ikke med i planlægningen af patienter, som beskrevet i bilag \ref{bilag01}. Dermed mangler OA et redskab til at estimere indlæggelsesvarigheden af patienter mhp. at planlægge elektive patienter. 


\section{Prædiktiv modellering} \label{praemodel}
En prædiktiv model er en statistisk model, der kan forudsige fremtidige hændelser. En sådan en model anvender tidligere data til forudsigelse af hændelser, i lignende data, på baggrund af en algoritme. Nogle typer af prædiktive modeller, eks. machine learning, kan behandle data af komplekse størrelser, hvilket kan skabe ny information om ukendte eller skjulte sammenhænge. Dette kan muliggøre, at hidtil oversete mønstre mellem parametre, kan benyttes til prædiktion. 

På nuværende tidspunkt anvendes prædiktive modeller flere steder. De benyttes bl.a. til forudsigelse af aktiekurser eller forslag til film på Netflix ud fra, hvilke film, der tidligere er set.\cite{DIKU2012}
De prædiktive modeller benyttes ligeledes i sundhedssektoren, hvor de bl.a. benyttes som diagnostik modeller, der ud fra symptomer forsøger at diagnosticerer patienten.\cite{Kuhn2013} Derudover kan det eks. være en forudsigelse om, hvorvidt en patient, indlagt med hjertestop, har risiko for endnu et hjertestop, hvoraf vurderingen f.eks. baseres på demografi, livsstil samt kliniske faktorer\cite{Hastie2008}. 

Dog kan prædiktive modeller ligeledes estimere forkert, hvilket eks. kan have konsekvenser ift. diagnostiske modeller, hvorved patienter vil opleve at få givet en fejldiagnose og i værste fald modtage en forkert behandling. Nogle af disse fejlprædikitoner kan skyldes komplekse variabler såsom menneskelig adfærd. Det kan herved være fordelagtigt at kombinere den prædiktive models resultater med en fagpersons vurdering. Dette kan typisk give et bedre estimat end modellen eller fagpersonen alene.\cite{Kuhn2013} 

Med viden om indlæggelsesvarigheden, ud fra en prædiktiv model, kan det være muligt at anvende dette som et redskab til planlægning af elektive patienter. Ved at sammenholde parametre i aktivitet og kapacitet i den prædiktive model, er det muligt at planlægge aktiviteten ud fra kapaciteten. Dette vil i praksis betyde, at elektive patienters indlæggelse kan planlægges ud fra tilgængeligt sundhedspersonale, rum og udstyr som beskrevet i afsnit \ref{kap}. 


\section{Udarbejdelse}

For at en prædiktiv model kan udarbejdes, er det nødvendigt at tilsidesætte data til et træningssæt. Træningssættet bruges til at udvikle modellen, hvorefter denne testes på det resterende data, kaldet testsættet, for således at måle præcisionen af modellen. Af \figref{traenings} illustreres opdelingen af træningssæt og testsæt.

\begin{figure}[H]
	%\flushleft 
	\centering
	\includegraphics[scale=.7]{figures/xval.png}
	%\flushleft
	\caption{\textit{Datapunkter opdelt i træningssæt og testsæt.}\cite{Kuhn2013}}
	\label{traenings}
\end{figure}

\noindent
For at kunne udarbejde en prædiktiv model er det ligeledes nødvendigt, at data er homogent, da inhomogen data kan forringe præcisionen. 
Det anses derfor nødvendigt at præprocessere data, der indgår i træningssættet inden det kan anvendes til prædiktering.\cite{Kuhn2013}
Præprocessering udføres generelt ved at tilføje surrogatdata eller fjerne samples, hvor der mangler parametre. Desuden er det ofte fordelagtigt at transformere data i tilfælde, hvor data ikke er normalfordelt. Normalfordeling af data giver flere muligheder for statistisk behandling.
Det er desuden vigtigt at analysere datasættets parametre. Hvis parametre har høj korrelation er det sandsynligt, at disse giver samme information, hvorfor nogle parametre kan udelades. Ved at reducere antallet af disse parametre, der behandles i en prædiktiv model, er det muligt at opnå en kortere beregningstid for modellen.
En dårlig præprocessering er en hyppig årsag til fejl. Det er derfor en vigtig del af udarbejdelsen af en prædiktiv model.\cite{Kuhn2013}


\section{Implementering på ortopædkirurgisk afdeling}
Ved implementering af en prædiktiv model forekommer en opstartsfase. I denne fase kan det forekomme, at modellen er upræcis. Dette kan f.eks. skyldes, at modellen er over-fittet til træningssættet. Over-fitting er en almindelig fejl, der kan forekomme, hvis udvikleren ikke har en metodisk tilgang til evaluvere modellen. I dette tilfælde opdages fejlen ikke før modellen benyttes på ny data.  
Det kan således være nødvendigt at tilpasse modellen yderligere i en periode efter den er taget i brug.\cite{Kuhn2013} Ved implementering af en prædiktiv model på OA bør den estimerede indlæggelsesvarighed derfor sammenholdes med den reelle indlæggelsesvarighed. Dette kan styrke modellens fremtidige estimater.

Modellen bør på OA anvendes i forbindelse med planlægning af patienter. Som tidligere nævnt i afsnit \ref{book} er lægesekretæren ansvarlig for planlægning af patienter, hvorfor det forventes, at redskabet primært er tilegnet dette personale. Det anses detsuden fordelagtigt, at fagpersoner, såsom læger eller sygeplejersker, ligeledes får indflydelse på det endelige estimat af indlæggelsesvarigheden, da der således kan tages højde for patienters individuelle ønsker og behov.




\subsection{Brug af databaser}
Da en prædiktiv model finder sammenhænge mellem data, som beskrevet i afsnit \ref{praemodel}, er det nødvendigt, at data er angivet homogent således modellen kan sammenholde data\cite{Kuhn2013}. 
%Det  kan dermed være hensigtsmæssigt at indsamlingen af data kan forløbe automatisk fra patientjournaler og derfor ikke være en yderligere belastning for sundhedspersonalet. 
Dette stiller, under det nuværende system, nogle krav til sundhedspersonalet om indskrivning af data i patientjournaler. Derudover kan det være hensigtsmæssigt for den prædiktive model at modtage data fra flere databaser. Disse kan eks. være patientjournaler, vagtskemaer og inventarlister.

%Et eksempel på dette kan være data over kapaciteten der er til rådighed for afdelingen, således det er muligt at planlægge aktiviteten ud fra kapaciteten.


%\subsection{Ændringer i arbejdsrutiner}\label{arbjedsrut}
%Som tidligere nævnt i afsnit \ref{book} indkalder lægesekretæren i forvejen patienter ud fra patientens ønske til operationsdag eller en specifik kirurg. Ved at anvende et redskab, som en prædiktiv model, muliggøres det, at lægesekretæren kender til den estimerede indlæggelsesvarighed for patienten inden operationen har fundet sted. 


\section{Effekt af en prædiktiv model}
En prædiktiv model muliggør angiveligt en bedre planlægning, da lægesekretæren har et estimat af patienternes indlæggelsesvarighed.
Det kan derfor være muligt på forhånd at udskyde elektive patienter, hvis der i perioder er udsigt til kapacitetsmangel.
Derudover kan det være muligt at indkalde elektive patienter i perioder, hvor afdelingen har mere kapacitet end aktivitet. 
Dette forventes at medføre en balance i kapacitetsudnyttelsen, hvilket kan have en positiv indvirkning for sundhedspersonale og patienter. 

%En prædiktiv model vil kunne medvirke til, at elektive patienter planlægges ud fra tilgængeligt sundhedspersonale. Dette kan medvirke til bedre udnyttelse af kapaciteten, da antallet af patienter i højere grad vil afspejle antallet af personale til rådighed. 
%Det kan derved være muligt at opnå balance mellem aktivitet og kapacitet på OA. 



\subsection{Sundhedspersonale} \label{sundper}
%Da implementeringen af en prædiktiv model muliggør planlægning af elektive patienter ud fra kapaciteten, herunder tilgængeligt sundhedspersonale 
forventes det, at sundhedspersonalet får en mere stabil arbejdsbyrde. 

Der skal dog tages forbehold for akut patient belastning. (effekten kan dog variere efter akut patientbelastning)

\fxnote{Perspektivering: På nuværende tidspunkt varetager to sygeplejersker to til otte patienter fordelt mellem sig, som nævnt i afsnit \ref{arb_per}. Denne fordeling forventes at kunne planlægges ud fra antallet af personale ift. antallet af patienter, hvormed arbejdsbyrden tilpasses.}

%Med den prædiktive model kan det, som tidligere nævnt i afsnit \ref{arbejdsrut}, være muligt at aflyse og dermed udsætte patienters operationer, hvis der i perioder er udsigt til kapacitetsmangel. 

Ligeledes kan sundhedspersonalet underrettes og indkaldes før kapacitetsmanglen opstår.


Ved planlægning reduceres risikoen for overarbejde for personalet, da planlægning af patienter forventes at give en bedre struktur og overblik. Det kan derfor være med til at afvikle perioder med kapacitetsmangel.\fxnote{"Kig på den anden af sagen" - Linette}


\subsection{Patienter}
%En prædiktiv model vil kunne medføre en mere stabil kapacitetsudnyttelse, da aktiviteten planlægges ud fra kapaciteten. 
Hensigten med modellen er at planlægge antallet af patienter ud fra kapaciteten, hvorved det forventes, at patientplejen vil være mere stabil. En mere stabil patientpleje kan medvirke til en bedre patientsikkerhed. 

\fxnote{revurder: Ydermere vil reduceringen af overarbejde for personalet medvirke til en bedre kvalitet af behandlingen af patienterne.}


Redskaber til planlægning af patienter kan medvirke til, at OA kan forebygge, at patienter placeres på gangarealer, andre afsnit og matrikler, hvilket medvirker til at bibeholde patienters privatliv, som nævnt i afsnit \ref{patsik}. Dertil reduceres risikoen for tilkaldelse af brandvagter i forbindelse med kapacitetsmangel. 
%Dette kan udover sikkerhed på afdelingen medføre besparelser for OA. 








