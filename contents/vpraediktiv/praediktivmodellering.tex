\chapter{Analyse af prædiktiv model}
På nuværende tidspunkt har OA ikke et redskab, der kan forudsige indlæggelsesvarigheden mhp. at planlægge patienter.
%Da der på nuværende tidspunkt ikke er et redskab til at forudsige indlæggelsesvarigheden er det ikke muligt at planlægge, hvornår der skal indkaldes elektive patienter ift. kapaciteten på OA. Sundhedspersonalet har dog en idé om, hvor længe patienterne er indlagt ift. operationstype, men dette tages ikke med i planlægningen af patienter, som beskrevet i bilag \ref{bilag01}. Dermed mangler OA et redskab til at estimere indlæggelsesvarigheden af patienter mhp. at planlægge elektive patienter. 
En mulig løsning til dette kan være at anvende en prædiktiv model. 


\section{Prædiktiv modellering} \label{praemodel}
En prædiktiv model anvender tidligere data til at forudsige fremtidige hændelser i lignende data på baggrund af algoritmer. Ligeledes kan nogle typer af prædiktive modeller behandle data af en kompleks størrelse, hvilket kan muliggøre prædiktering af ukendte sammenhænge. Dette medvirker til, at flere parametre kan indgå i modellen.\cite{Kuhn2013}


I sundhedssektoren anvendes prædiktive modeller på nuværende tidspunkt til at prædiktere hændelser og forløb. Dette kan eksempelvis være en forudsigelse om, hvorvidt en patient, indlagt med hjertestop, har risiko for endnu et hjertestop, hvoraf vurderingen f.eks. baseres på demografi, livsstil samt kliniske faktorer\cite{Hastie2008}. Det ses derfor muligt at anvende en lignende model i en anden sundhedsfaglig kontekst.


Med viden om indlæggelsesvarigheden, ud fra en prædiktiv model, kan det være muligt at anvende dette som et redskab til planlægning af elektive patienter. Ved at sammenholde parametre i aktivitet og kapacitet i den prædiktive model, er det muligt at planlægge aktiviteten ud fra kapaciteten. Dette vil i praksis betyde, at elektive patienters indlæggelse kan planlægges ud fra tilgængeligt sundhedspersonale, rum og udstyr som beskrevet i afsnit \ref{kap}. 


\section{Implementering af en prædiktiv model}
Ved implementering af en prædiktive model på OA, kan der forekomme ændringer i arbejdsrutiner. Modellen bør anvendes i forbindelse med planlægning af patienter til operation. Som tidligere nævnt i afsnit \ref{book} er lægesekretæren ansvarlig for indkaldelse af patienter, hvorfor det forventes, at redskabet primært er tilegnet dette personale.\fxnote{Passer denne tekst stadig efter afsnittet er fikset?! hvem ved?!}


\subsection{Opstartsfase}
%Når en prædiktiv model udarbejdes, gøres dette på baggrund af tidligere data. Dette skal foregå før implementering af modellen kan finde sted. Ved at implementere en prædiktiv model på OA vil denne derfor kræve en opstartsfase, der forholder sig til den aktuelle data på afdelingen. Grundet opstartsfasen kan der forekomme en tidsperiode, hvor den prædiktive model ikke er præcis da denne skal tilpasses afdelingen. Dertil bør den prædiktive models output sammenholdes med den reelle indlæggelsestid, for at forbedre fremtidige outputs. Længden af denne fase kan afhænge af mængden af patientdata og nøjagtigheden af modellen.\cite{Kuhn2013}


\subsection{Brug af databaser}
Da en prædiktiv model finder sammenhænge mellem data, som beskrevet i afsnit \ref{praemodel}, er det nødvendigt, at data er angivet homogent således modellen kan sammenholde data\cite{Kuhn2013}. 
%Det  kan dermed være hensigtsmæssigt at indsamlingen af data kan forløbe automatisk fra patientjournaler og derfor ikke være en yderligere belastning for sundhedspersonalet. 
Dette stiller, under det nuværende system, nogle krav til sundhedspersonalet om indskrivning af data i patientjournaler. Derudover kan det være hensigtsmæssigt for den prædiktive model at modtage data fra flere databaser. Disse kan eks. være patientjournaler, vagtskemaer og inventarlister.

%Et eksempel på dette kan være data over kapaciteten der er til rådighed for afdelingen, således det er muligt at planlægge aktiviteten ud fra kapaciteten.


%\subsection{Ændringer i arbejdsrutiner}\label{arbjedsrut}
%Som tidligere nævnt i afsnit \ref{book} indkalder lægesekretæren i forvejen patienter ud fra patientens ønske til operationsdag eller en specifik kirurg. Ved at anvende et redskab, som en prædiktiv model, muliggøres det, at lægesekretæren kender til den estimerede indlæggelsesvarighed for patienten inden operationen har fundet sted. 


\section{Effekt af en prædiktiv model}
En prædiktiv model muliggør angiveligt en bedre planlægning, da lægesekretæren har et estimat af patienternes indlæggelsesvarighed.
Det kan derfor være muligt på forhånd at udskyde elektive patienter, hvis der i perioder er udsigt til kapacitetsmangel.
Derudover kan det være muligt at indkalde elektive patienter i perioder, hvor afdelingen har mere kapacitet end aktivitet. 
Dette forventes at medføre en balance i kapacitetsudnyttelsen, hvilket kan have en positiv indvirkning for sundhedspersonale og patienter. 

%En prædiktiv model vil kunne medvirke til, at elektive patienter planlægges ud fra tilgængeligt sundhedspersonale. Dette kan medvirke til bedre udnyttelse af kapaciteten, da antallet af patienter i højere grad vil afspejle antallet af personale til rådighed. 
%Det kan derved være muligt at opnå balance mellem aktivitet og kapacitet på OA. 



\subsection{Sundhedspersonale} \label{sundper}
%Da implementeringen af en prædiktiv model muliggør planlægning af elektive patienter ud fra kapaciteten, herunder tilgængeligt sundhedspersonale 
forventes det, at sundhedspersonalet får en mere stabil arbejdsbyrde. 

Der skal dog tages forbehold for akut patient belastning. (effekten kan dog variere efter akut patientbelastning)

\fxnote{Perspektivering: På nuværende tidspunkt varetager to sygeplejersker to til otte patienter fordelt mellem sig, som nævnt i afsnit \ref{arb_per}. Denne fordeling forventes at kunne planlægges ud fra antallet af personale ift. antallet af patienter, hvormed arbejdsbyrden tilpasses.}

%Med den prædiktive model kan det, som tidligere nævnt i afsnit \ref{arbejdsrut}, være muligt at aflyse og dermed udsætte patienters operationer, hvis der i perioder er udsigt til kapacitetsmangel. 

Ligeledes kan sundhedspersonalet underrettes og indkaldes før kapacitetsmanglen opstår.


Ved planlægning reduceres risikoen for overarbejde for personalet, da planlægning af patienter forventes at give en bedre struktur og overblik. Det kan derfor være med til at afvikle perioder med kapacitetsmangel.\fxnote{"Kig på den anden af sagen" - Linette}


\subsection{Patienter}
%En prædiktiv model vil kunne medføre en mere stabil kapacitetsudnyttelse, da aktiviteten planlægges ud fra kapaciteten. 
Hensigten med modellen er at planlægge antallet af patienter ud fra kapaciteten, hvorved det forventes, at patientplejen vil være mere stabil. En mere stabil patientpleje kan medvirke til en bedre patientsikkerhed. 

\fxnote{revurder: Ydermere vil reduceringen af overarbejde for personalet medvirke til en bedre kvalitet af behandlingen af patienterne.}


Redskaber til planlægning af patienter kan medvirke til, at OA kan forebygge, at patienter placeres på gangarealer, andre afsnit og matrikler, hvilket medvirker til at bibeholde patienters privatliv, som nævnt i afsnit \ref{patsik}. Dertil reduceres risikoen for tilkaldelse af brandvagter i forbindelse med kapacitetsmangel. 
%Dette kan udover sikkerhed på afdelingen medføre besparelser for OA. 








