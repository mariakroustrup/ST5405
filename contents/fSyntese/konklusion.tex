\section{Konklusion}
På baggrund af diskussionen konkluderes projektets problemformulering. Udtalelser fra OA indikerer, at afdelingen oplever et problem med kapacitetsmangel. Denne problemstilling er dog ikke tydelig i den offentligt tilgængelige statistik, hvormed det vurderes fordelagtigt for afdelingen at indsamle intern statistik, der tydeliggør problemets omfang. 
Det er ikke muligt at undgå kapacitetsmangel med en prædiktiv model, da det er umuligt at forudsige antallet af akutte patienter. Modellen skal ses som et redskab til at hjælpe afdelingen med at overholde den normerede kapacitet.
Det vurderes desuden, at et yderligere studie omkring, hvilke parametre, der påvirker indlæggelsesvarigheden kan være nødvendigt for at udarbejde en prædiktiv model, der indberegner flest mulige aspekter af kapacitetsudnyttelsen på OA.
Det konkluderes på baggrund af litteraturen, at en prædiktiv model kan anvendes til at forudsige indlæggelsesvarigheden på OA. Det er imidlertid ikke muligt, på baggrund af denne projektrapport og tilgængelige data, at vurdere en prædiktiv models præcision. Yderligere studier er således påkrævede for at vurdere en prædiktiv models præcision ift. indlæggelsesvarighed.