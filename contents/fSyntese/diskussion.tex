\chapter{Diskussion} \label{diskussion}


Formålet med projektet er at analysere en prædiktiv model til estimering af indlæggelsesvarigheden for patienter mhp. at planlægge elektive patienter på OA. I dette afsnit vil hhv. metode og analysering af model diskuteres i ift. til besvarelse af den opstillede problemformulering. \\


\section{Metode}
Litteratursøgningen for dette projekt har taget udgangspunkt i litteratur omhandlende den danske sundhedssektor, da fokus har været på OA. Afgrænsningen til den danske sundhedssektor har medvirket til begrænset litteratur, hvorfor det har været nødvendigt at afholde et interview på OA. 
For at få større viden samt udbytte af interviewet, vil det være fordelagtigt at afholde interviews med mere personale fra samtlige afsnit på OA. Herunder skal det diskuteres, hvorvidt flere sekretærer burde interviewes for således at opnå en større forståelse af deres nuværende planlægning af patienter. Derudover bør der foretages opfølgende interviews for at belyse og besvare evt. nye spørgsmål opstået efterfølgende. 


\section{Problemstilling}
På OA opleves der i perioder ubalance i kapacitetsudnyttelsen, hvilket kan skyldes, at OA på nuværende tidspunkt ikke har et redskab til estimering af indlæggelsesvarighed for patienter mhp. på at planlægge elektive patienter. Da de elektive patienter udgør $32~ \%$ af patienterne på OA er det ikke kendt, hvorvidt problemet med ubalance i kapacitetsudnyttelse kan afhjælpes med en prædiktiv model. Dog ses en prædiktiv model som en mulighed til forbedring af planlægning, hvorfor en balance tilstræbes. Hvis indlæggelsesvarigheden kan estimeres kan de elektive patienter planlægges tættere, hvorfor kapaciteten på afdelingen udnyttes. 


Ud fra det tilgængelige statistik for OA ses den gennemsnitlige belægning ikke som værende et problem ift. kapacitetsmangel, dog ses den gennemsnitlige belægning ofte under $100~%$, hvilket derved ikke er optimal udnyttelse af kapaciteten på OA. Statistikken tyder derfor på et problem om manglende aktivitet ift. kapacitet. Personalet udtaler, at problemet er i form af perioder med svær kapacitetsmangel. Disse opleves dog ikke ofte, men personalet vurderer, at perioderne forværres hver gang. 


En bedre planlægning af patienter vil kunne medføre, at den maksimale samt minimale belægning pr. måned på \figref{maxminbelaeg} vil tilnærme sig $100~%$, hvorved gennemsnittet ligeledes vil tilnærme sig $100~%$ belægning. Dette vil angiveligt forårsage, at afdelingen vil opleve bedre balance i kapacitetsudnyttelsen. Forholdet mellem aktivitet og kapacitet vil nødvendigvis ikke opnå en balance, da planlægningen kun kan forekomme af de elektive patienter. Resultatet af den prædiktive model vil dog kunne præciseres ved at indføre udskrivningstidspunktet for patienter for således at revurdere præcisionen af indlæggelsesvarigheden.
Gennem interview med OA og en analyse af afdelingen er det kommet tilkende, at der er andre problemstillinger, der ikke kan afhjælpes med en prædiktiv model. Dette er eksempelvis ift. hjemsendelse af patienter, der skal have hjemmepleje.
