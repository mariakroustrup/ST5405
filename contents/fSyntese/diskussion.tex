\section{Diskussion}
*** DER MANGLER EN INDLEDNING ***


\subsection{Prædiktering af indlæggelsesvarighed}
Det skal vurderes, hvor ofte systemet skal anvendes til at prædiktere indlæggelsesvarighed. Det blev i \ref{problemløsning} belyst, at flere parameter har indflydelses på indlæggelsesvarigheden både præ- og postoperativt. Hvis systemet skal anvendes præoperativt er det i flere tilfælde kun muligt at estimere indlæggelsesvarigheden for elektive patienter og ikke akutte patienter. Systemet er derfor primært egnet til elektive patienter. Derimod kan indlæggelsesvarigheden for akutte patienter estimeres, hvis estimeringen sker postoperativt. Ligeledes kan der opstå komplikationer under operationen, hvorved postoperative parametre som operationsvarighed  og muligheden for træning dagen efter kan have indflydelse på indlæggelsesvarigheden, hvorfor det skal være muligt at estimere varigheden for elektive patienter igen. 

For at kunne anvende estimering af indlæggelsesvarigheden til at udnytte kapaciteten skal estimeringen vurderes på baggrund af normerede sengepladser på afdelingen og antallet af akutte indlæggelser. Det er ikke muligt, at forudsige antallet af akutte patienter der indlægges, men det er muligt at estimere indlæggelsesvarigheden postoperativt for disse patienter. Dette kan sammenholdes med indlæggelsesvarigheden for elektive patienter og dermed kan det være muligt for ortopædkirurgisk afdeling at udskyde elektive patienter før den normerede kapacitet opnås eller indkalde elektive patienter, hvis der er for lidt aktivitet ift. kapacitet.

\subsection{Personalearbejde}
På ortopædkirurgisk afdeling planlægges patienternes indlæggelsesvarighed ud fra sundhedspersonalets erfaring med lignende patienter \ref{bilagC, bilagD}. Ved at anvende en prædiktiv model som hjælpemiddel til estimeringen af indlæggelsesvarighed sammen med tidligere erfaring, er der mulighed for få en mere præcis estimering. 
En estimering af indlæggelsesvarigheden kan  derudover være et redskab til planlægning af personalets arbejde. Dette kan medvirke til en bedre strukturering og mindske arbejdsbyrden og dermed skabe en bedre balance mellem kapacitet og aktivitet. Modsat kan en bedre planlægning af arbejdet medfører, at arbejdsbyrden forværres, hvis der i tilfælde af ledige normerede sengepladser indkaldes flere elektive patienter med henblik på at udnytte kapaciteten. 


Dette kan skabe problemer ift. handleplanen for år 2017, hvor elektive patienter ventetid på operationen skal forkortes med 7 dage, som nævnt i afsnit \ref{indl}. Dertil er der allerede et eksisterende behov for, at sundhedspersonalet, ved underkapacitet er nødsaget til, at påtage sig en større arbejdsbyrde end normalt. \ref{bilagC} Herudover bør der ikke blot tages hensyn til antallet af patienter, men ligeledes patienternes plejetyngde. \ref{bilagC} Nogle patienter kan have mere behov for pleje eksempelvis patienter med komorbiditeter, hvorfor det ligeledes kan være relevant at planlægge personalets arbejde efter plejetyngde. 


****** Der mangler lidt en overgang måske ellers skal der overskrifter på afsnittene. *****

\noindent
Den prædiktive skal være et anvendeligt redskab for personalet, hvilket betyder at det ikke må være tidskrævende. Hvis dette ikke er tilfældet, vil det forværrer tidspresset under underkapacitet og i givet fald blive nedprioriteret. Det bør derfor overvejes, hvordan systemet skal fungere på afdelingen ift. hentning af data fra patienterne. Hvis systemet kan hente data fra eksempelvis clinical suite eller elektroniske patientjournaler automatisk vil indsamlingen af data om patienten ikke kræve yderligere tid for personalet. 


**** Mangler en overgang her også. ****


I situationer med mangel på fysiske pladser, er det som tidligere nævnt i afsnit \ref{patsik}, nogle gange nødvendigt at flytte patienter ud på hospitalsgangene. Ved opnåelse af større overblik over patientantal samt antallet af indlæggelser og udskrivelser, kan det i højere grad tilstræbes at undgå patienter på hospitalsgangene. Dette kan medføre større patientsikkerhed i tilfælde af brand, og færre komplikationer med brandtilsynet som beskrevet i afsnit \ref{patsik, bilagC}. 

*** VI MANGLER AT LAVE DISKUSSION FOR PROBLEMLØSNINGEN ***