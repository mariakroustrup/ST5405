\section{Diskussion} \label{diskussion}
Formålet med projektet er at analysere en prædiktiv model til estimering af indlæggelsesvarigheden for patienter mhp. at planlægge elektive patienter på OA. I dette afsnit vil hhv. metode og problemstilling diskuteres ift. besvarelse af den opstillede problemformulering. \\


\subsection{Metode}
Litteratursøgningen for dette projekt har primært taget udgangspunkt i litteratur omhandlende den danske sundhedssektor, da projektrapporten omhandler OA. Afgrænsningen til den danske sundhedssektor har medvirket til begrænset litteratur, hvorfor det har været nødvendigt at afholde et interview på OA. 
For at få større viden samt udbytte af interviewene, vil det være fordelagtigt at afholde interviews med mere personale og fra samtlige afsnit på OA. Der stilles ligeledes spørgsmål ved, hvorvidt flere sekretærer burde interviewes for således at opnå en større forståelse af OA's nuværende planlægning af patienter. Derudover bør der afholdes opfølgende interviews for at belyse og besvare evt. nye spørgsmål opstået efterfølgende. 


\subsection{Problemstilling}
På OA opleves der i perioder ubalance i kapacitetsudnyttelsen. OA på nuværende tidspunkt ikke har et redskab til estimering af indlæggelsesvarigheden for patienter mhp. på at planlægge elektive patienter. Da de elektive patienter udgør $32~ \%$ af patienterne på OA er det ikke kendt, hvorvidt problemet med ubalance i kapacitetsudnyttelse kan afhjælpes med en prædiktiv model. Dog ses en prædiktiv model som en mulighed til forbedring af planlægning, hvorfor en balance tilstræbes. Hvis indlæggelsesvarigheden kan estimeres kan de elektive patienter planlægges tættere, hvorfor kapaciteten på afdelingen udnyttes bedre. 


Ud fra det tilgængelige statistik for OA ses den gennemsnitlige belægning ikke som værende et problem ift. kapacitetsmangel. Den gennemsnitlige belægning ses dog ofte under $100~\%$, hvilket derved ikke er optimal udnyttelse af kapaciteten på OA. Statistikken tyder derfor på et problem om manglende aktivitet ift. kapacitet. Overlæge på OA udtaler dog, at problemet er i form af perioder med svær kapacitetsmangel. Disse opleves ikke ofte, men det vurderes, at perioderne forværres hver gang.[\ref{sten}] 

En bedre planlægning af patienter vil kunne medføre, at den maksimale samt minimale belægning pr. måned på \figref{maxminbelaeg} vil tilnærme sig $100~\%$, hvorved gennemsnittet ligeledes vil tilnærme sig $100~\%$ belægning. Dette vil angiveligt forårsage, at afdelingen vil opleve bedre balance i kapacitetsudnyttelsen. Det vil nødvendigvis ikke være muligt at opnå en balance i forholdet mellem aktivitet og kapacitet, da planlægningen kun kan forekomme af de elektive patienter, der ikke udgør den største del af patienter på OA. 
Dog vil resultatet af den prædiktive model angiveligt give et positivt udbytte ift. planlægningen. For at revurdere præcisionen af estimatet af indlæggelsesvarigheden ses det fordelagtigt at indføre udskrivelsestidspunktet for patienterne efterfølgende.

Gennem interview med OA og en analyse af afdelingen er det kommet tilkende, at der er andre problemstillinger, der ikke kan afhjælpes med en prædiktiv model. Dette er eksempelvis ift. hjemsendelse af patienter, der har behov for hjemmepleje efter udskrivelse.
