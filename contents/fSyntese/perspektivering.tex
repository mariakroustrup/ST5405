\section{Perspektivering} 
En prædiktiv model på OA er som udgangspunkt et redskab til planlægning af indkaldelse af patienter ud fra kapaciteten på afdeling med henblik på en bedre strukturering af patient indlæggelser. Implementering af modellen vil ikke kun hjælpe lægesekretæren til at have et bedre overblik over patienter på afdelingen, men kan anvendes til at omstrukturere personalets arbejdsopgaver ud fra kapaciteten på afdelingen. Der bør dertil undersøges, hvorvidt modellen kan medvirke til flere ændringer og dermed effektivisere afdelingen yderligere. Der kan bl.a. effektiviseres ift. afhentning af patienter, hvor der som beskrevet i afsnit \ref{XXX}, skal ske en kontakt med kommunen. Dette er ikke et problem som en prædiktiv model direkte kan afhjælpe, men modellen kan informere personalet om, hvornår patientens udskrivelse er estimeret til og på baggrund af dette kontakte kommunen i tidligere. Derfor skal der flere undersøgelser på afdelingen ift. problemer der indirekte medfører begrænsninger for kortere indlæggelsesvarighed af patienter. 


Udover ændringer for afdelingen skal det overvejes, hvilken metode inden for prædiktiv modellering der er mest hensigtsmæssig at estimere indlæggelsesvarigheden ud fra på baggurnd af tilgængelig data på afdelingen. Hertil skal det også vurderes, hvordan indskrivningen af nye parametre skal designes så der opstår færrest komplikationer og fejl under indskrivningen. Der bør undersøges om designet af den digitale patientjournal skal ændres mhp. at kunne hente data autonomt. 


Hvis en prædiktiv model kan anvendes på ortopædkirurgisk afdeling forventes det ligeledes muligt at kunne implementere en tilsvarende model på andre afdelinger på Aalborg Universitetshospital. Implementering af en prædiktiv model på andre afdelinger bør undersøges nærmere. 
