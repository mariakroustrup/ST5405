\section{Perspektivering} 
En prædiktiv model på OA er som udgangspunkt et redskab til planlægning af patienter ud fra kapaciteten på afdeling mhp. effektivisering af kapacitetsudnyttelsen. Implementering af modellen vil ikke kun hjælpe lægesekretæren til at have et bedre overblik over patienter på afdelingen, men kan anvendes til at omstrukturere personalets arbejdsopgaver ud fra kapaciteten på afdelingen. Det bør dertil undersøges, hvorvidt modellen kan medføre flere ændringer og dermed effektivisere afdelingen yderligere. 
%Der kan bl.a. effektiviseres ift. afhentning af patienter, hvor der skal ske en kontakt med kommunen. Dette er ikke et problem som en prædiktiv model direkte kan afhjælpe, men modellen kan informere personalet om, hvornår patientens udskrivelse er estimeret til og på baggrund af dette kontakte kommunen tidligere. Derfor skal der flere undersøgelser på afdelingen ift. problemer der indirekte medfører begrænsninger for kortere indlæggelsesvarighed af patienter.


Udover ændringer for afdelingen skal det overvejes, hvilken metode inden for prædiktiv modellering, der er mest hensigtsmæssig til at estimere indlæggelsesvarigheden på baggrund af tilgængelig data på afdelingen. Ligeledes vurderes, hvordan indskrivningen af nye parametre skal designes så der opstår færrest komplikationer og fejl under indskrivningen. 
%Det bør undersøges om designet af den digitale patientjournal skal ændres mhp. at kunne hente data autonomt. 
Derudover bør det undersøges om der er nogle eksterne faktorer, uden for OA, der har indflydelse på indlæggelsesvarigheden for patienter. Dette kan eks. være, at der er en tendens til, at flere indlægges i vinterperioder.

Ved en succesfuld implementering af en prædiktiv model til estimering af indlæggelsesvarighed, forventes det også muligt at udarbejde en tilsvarende model til at prædiktere andre parametre på afdelingen. Dette kunne eks. være dødelighed efter operation.
Hvis en prædiktiv model kan anvendes på OA forventes det ligeledes muligt at kunne implementere en tilsvarende model på andre afdelinger på Aalborg Universitetshospital.