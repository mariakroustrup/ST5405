\section{Baggrund for projektet}
Projektet er stillet af Sten Rasmussen fra ortopædkirugisk afdeling og Christian Kruse fra endokrinologisk afdeling. Formålet er at udarbejde en prædiktiv model til at forudsige indlæggelsesvarigheden på ortopædkirugisk afdeling baseret på et eksisterende datasæt med 1.000 hospitalsindlæggelser. 

\section{Indledning-ish}
Flere danske hospitalsafdelinger oplever i perioder at have flere patienter end der er kapacitet til. Dette kaldes overbelægning og kan være et problem, afhængigt af afdelingen og varigheden.\cite{SDS2015} Den høje belægningsgrad resulterer bl.a. i, at sundhedspersonalet får mindre tid pr. indlagt patient, hvilket kan medføre gener for både personale og patient.\cite{Kjeldsen2015}

% ======== Er ikke sikker på om denne del skal med eller ej
Ifølge en undersøgelse fra Dansk Sygeplejeråd, mener hver anden regionalt ansat sygeplejerske på tværs af regionerne, at den travle arbejdsdag går ud over patienternes sikkerhed\cite{Kjeldsen2015}.
% ========
\fxnote{Det kunne nok være fint med nogle økonomiske konsekvenser, men jeg synes ikke vi har noget solidt endnu. Ellers ved jeg ikke hvor meget der skal med af konsekvenser, da det initierende problem lægger op til at analysen skal undersøge dette, så det er lidt underligt at ridse dem alle op nu.. }

\subsection{Inititerende problem(er) - udkast - slet subsection efter brug}
\subsubsection{Generel og uden ordet overbelægning:}
\textbf{Hvor udbredte er belægningsrelaterede problemer på ortopædkirugisk afdeling på Aalborg universitetshospital og hvad er konsekvensen af en belægningsgrad over, såvel som under, $100$\%}

\subsubsection{Specifik og med ordet overbelægning:}
\textbf{Hvor stort et problem er overbelægning på ortopædkirugisk afdeling på Aalborg universitetshospital og hvordan påvirkes personalet af dette}