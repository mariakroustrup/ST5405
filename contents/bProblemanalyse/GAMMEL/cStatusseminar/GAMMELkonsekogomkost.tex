\subsection{Konsekvenser ved overbelægning}
Hvis der sker overbelægning på ortopædkirurgisk afdeling på Aalborg Universitetshospital er der nogle retningslinjer for varetagelse af personalets arbejdsopgaver samt omflytning af patienter, med henblik på at nedbringe overbelægningen hurtigst muligt.

\subsubsection{Omstruktuering af personalets arbejdsopgaver} \label{Tilret}
Der er udarbejdet en arbejdstilrettelæggelse af Region Nordjylland, der har til formål at sikre patientens behov, kvalitetssikring, udnyttelse af kompetencer med henblik på at finde en balance mellem ressourcer og krav i den pågældende situation. Dette kan betyde at det afgåede vagthold skal blive indtil en midlertidig løsning er fundet samt indkaldelse af det næste vagthold tidligere. Det kan i nogle tilfælde være nødvendigt at låne ressourcer fra andre afsnit eller indkalde personale fra vikarbureauet. Derudover undersøges om behandlingen af elektive patienter kan aflyses.\cite{Bjerg2016}

De ekstra patienter, der behandles under overbelægning bliver fordelt mellem det tilstedeværende sundhedspersonale, og overskrider dermed den forventede patientbyrde. Dette resulterer i, at sundhedspersonalet har kortere tid til den enkelte patient, hvorfor risikoen for fejl øges, hvilket betyder, at det ikke er muligt at give patienterne den nødvendige behandling efter det kirurgiske indgreb.\cite{Dinges2004,Aiken2002} 

Hvis sundhedspersonalet er nødsaget til at arbejde længere end den normale arbejdstid, viser dette sig at have en negativ indvirkning på personalet.\cite{Kjeldsen2015,Dinges2004} Undersøgelser viser, at dette resulterer i en presset arbejdsdag og dermed en forringet kvalitet af behandlingen.\cite{Kjeldsen2015} Et amerikansk studie har på baggrund af undersøgelser fra år $2002$ påvist, at fejlene hovedsageligt opstår, når personalet har arbejdsdage på mere end $12$ timer.\cite{Dinges2004}. Undersøgelsen skal ses i perspektiv med de danske overenskomster for sundhedspersonalet. Det fremgår dog af Dansk Sygeplejeråd, at hver anden regionalt ansat sygeplejerske på tværs af regionerne mener, at den travle arbejdsdag påvirker patienternes sikkerhed.\cite{Kjeldsen2015}

\subsubsection{Omflytning af patienter}
Nordjyllands Beredskabsstyrelse har opstillet retningslinjer for hvordan overbelægningen varetages. Det forsøges i første omgang at udfylde alle stuerne på ortopædkirugisk afdeling. Hvis dette ikke er muligt, flyttes patienter til andre hospitalsafdelinger. Når der ikke er plads på de andre hospitalsafdelinger, placeres patienterne i samtalerum og flugtvejsgange som en midlertidig nødløsning.\cite{Beredskab2016} 

Overflytningen og ophold i samtalerum samt flugtvejsgange bevirker til, at patienterne oplever et skærpet privatliv.\cite{Madsen2014} Derudover kan det belaste fysiske og psykiske forhold for patienterne såvel som pårørende.\cite{Heidmann2014} Som nævnt i afsnit \ref{Tilret} øges risikoen for fejl ved overbelægning og dertil ses det at mortalitetsraten øges med $1,2~\%$ ved en overskridelse af sengebelægningskapaciteten på $10~\%$ ifølge et dansk studie fra år 2014. \cite{Madsen2014} Hertil understreges det, at der kan være flere parametre, der påvirker mortaliteten og det nødvendigvis ikke er overbelægning der er den primære årsag til øget mortalitet. Overbelægning giver derfor et forøget pres for at få patienterne udskrevet, således at der opnås en normalbelægning og  den fysiske kapacitet ikke overstiges.

\subsection{Omkostninger ved overbelægning}
Udover omstrukturering af arbejdsopgaver samt omflytning af patienter, er der nogle juridiske problemstillinger, som skal overholdes for at disse ændringer er lovmæssigt sikre. Dette medvirker tilkaldelse af brandvagter for at skærpe sikkerheden ift. evakuering under brand. Derudover er det samtidigt vigtigt, at det bestræbes på at overholde løn- og overenskomsterne for sundhedspersonalet.

\subsubsection{Brandsikkerhed}
Ved overbelægning kontaktes en brandvagt, således at brandvagten kan være tilgængelig på afdelingen så snart overbelægningen finder sted. Hvis normal belægningstilstand er mulig inden for fire timer efter overbelægningen påbegyndes, er det ikke nødvendigt at tilkalde en brandvagt. En brandvagt kan højst overvåge to afdelinger på samme etage, hvorfor det kan være nødvendigt at der indkaldes flere. Det er afdelingens ansvar at afvikle overbelægningen hurtigst muligt, dette kan gøres ved at udskrive patienter eller overflytte patienter til sengestuer på andre afdelinger.\cite{Beredskab2016} Ved tilkaldelse af brandvagter som er juridisk nødvendigt ved overbelægning fratages der penge fra det overordnede budget for ortopædkirurgisk afdeling. 

\subsubsection{Overenskomster}
Som tidligere nævnt i afsnit \ref{Tilret}, forlænges sundhedspersonalets arbejdsdag under overbelægning. En normal arbejdsuge for danske sygeplejersker er 37 timer. I tilfælde af overarbejde må en arbejdsuge for en sygeplejerske, ifølge arbejdstidsaftalen indgået med Dansk Sygpelerråd, ikke overstige 48 timer. Planlægning af en sygeplejerskes normalarbejde skal derudover finde sted 24 timer før fremmøde, hvilket ikke inkluderer overarbejde.\cite{Danske2015}  
 

