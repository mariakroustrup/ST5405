%%%%% Mangler fxnote

%%% OMFLYTNING AF PATIENTER
%% Nogle af fxnotes skal skrives efter samtale med flere sygeplejersker
%% Der skal skrives noget om elektive patienter ud fra et tidligere et afsnit som ikke er skrevet endnu.

%%% TILRETTELÆGGELSE AF PERSONALE
%% Sygeplejersker: Hvor mange patienter skal i under normale omstændigheder varetage? Og hvordan fungerer det under overbelægning fordeles de enkelte patienter mellem jer? Erfaring?


%%% JURIDISKE PROBLEMSTILLINER:
%Hvis overbelægning er mere end 50 dage brydes loven (hvis det med elektive patienter kommer ind i omflytning af patienter.)
% Sygeplejerske/Sten/Økonomiafdeling: dette skal undersøges om det er fra det samlede budget eller om det bliver taget fra ortopædkirurgisk afdeling?
% Sygeplejersker: Hvordan prioriteres pauser under overbelægning

%%% OPSUMERING:
%%MANGLER KILDER PÅ NOGLE AF SÆTNINGERNE - kilderne skal tages når vi har fået svar fra sygeplejerskerne. 


\section{Konsekvenser og omkostninger ved overbelægning}
Ved overbelægning er det vigtigt, at situationen afvikles straks, da det bl.a. kan skabe problemer ift. arbejdsopgaver samt skræpe privatlivet for patienten og forøge mortalitetsraten. \cite{Madsen2014}
Ved overbelægning tilkaldes og tilrettelægges arbejdet for personalet, og der stræbes på at finde en balance mellem de eksisterende ressourcer, og de krav der stilles til den enkelte afdeling. \cite{Bjerg2016} Grundet pladsmangel ved overbelægning, skal nogle patienter omflyttes, dette sker ud fra nogle konkrete retningslinjer. \cite{Beredskab2016} Ved overbelægning skal der yderligere overholdes lovgivninger om sikkerhed og overenskomster for sundhedspersonalet. \cite{Beredskab2016}


\subsection{Tilrettelæggelse af personale} \label{Tilret}
Ved overbelægning gælder en arbejdstilrettelæggelse som er udarbejdet af Region Nordjylland og omfatter ortopædkirurgisk afdeling. Dennes formål er, at sikre patientens behov, kvalitetssikring, udnyttelse af kompetencer med henblik på at finde en balance mellem ressourcer og krav i den pågældende situation. Ved overbelægning påtager lederen eller dennes stedfortræder ansvaret for at finde en løsning. Dette betyder at det afgående vagthold skal blive indtil en midlertidig løsning er fundet. Derudover undersøges det om behandlingen af elektive patienter kan aflyses, og hvorvidt der bør indkaldes personale til at dække det manglede fremmøde. Dertil kan det være nødvendigt at indkalde det næste vagthold tidligere. Det kan blive en nødvendighed at låne ressourcer fra andre afsnit eller indkalde personale fra vikarbureauet. Den sidste løsning er at forlænge arbejdstiden for personalet på det afgående vagthold til situationen kan varetages af de resterende. \cite{Bjerg2016}

\noindent 
Ved normale omstændigheder varetages XX antal patienter pr. sygeplejerske. De ekstra patienter, der behandles under overbelægning bliver fordelt mellem det tilstedeværende sundhedspersonale, og overskrider dermed den forventede patientbyrde. \fxnote{Sygeplejersker: Hvor mange patienter skal i under normale omstændigheder varetage? Og hvordan fungerer det under overbelægning fordeles de enkelte patienter mellem jer? Erfaring?}
Dette resulterer i, at sundhedspersonalet har kortere tid til den enkelte patient, hvorfor risikoen for fejl øges, hvilket betyder, at det ikke er muligt at give patienterne den nødvendige behandling efter det kirurgiske indgreb. \cite{Dinges2004,Aiken2002} Udover overbelægning har reduceringen af sengepladser på $30~\%$ medvirket til, at øge sygeplejerskernes arbejdsbyrde med $40~\%$ fra år 2001 til 2015. \cite{Kjeldsen2015}


Hvis sundhedspersonalet er nødsaget til at arbejde længere end den normale arbejdstid, viser dette sig at have en negativ indvirkning på personalet.\cite{Kjeldsen2015,Dinges2004} Undersøgelser viser, at dette resulterer i en presset arbejdsdag og dermed en forringet kvalitet i behandlingen. \cite{Kjeldsen2015} Et amerikansk studie har på baggrund af undersøgelser fra år 2002 påvist, at fejlene hovedsageligt opstår, når personalet har arbejdsdage på mere end 12 timer.\cite{Dinges2004}. Undersøgelsen skal ses i perspektiv med de danske overenskomster for sundhedspersonalet. Det fremgår dog af Dansk Sygeplejeråd, at hver anden regionalt ansat sygeplejerske på tværs af regionerne mener, at den travle arbejdsdag påvirker patienternes sikkerhed \cite{Kjeldsen2015}. 
%Overbelægning medfører derfor gene for både personale og patienter. 
 
\subsection{Omflytning af patienter}
For ortopædkirurgisk afdeling på Aalborg Universitetshospital er der opstillet retningslinjer af Nordjyllands Beredskabsstyrelse for, hvordan overbelægningen varetages. Til at starte med, forsøges det at udfylde alle stuerne på ortopædkirugisk afdeling. Hvis dette ikke er muligt, flyttes patienter til andre hospitalsafdelinger. Når der ikke er plads på de andre hospitalsafdelinger, placeres patienterne i samtalerum og flugtvejsgange som en midlertidig nødløsning. \cite{Beredskab2016} De patienter, der overflyttes er ofte patienter, som er på grænsen til at blive udskrevet \fxnote{Sygeplejersker: Vi vil gerne høre om der prioriteres i forhold til hvilke patienter der flyttes. Er der en bestemt afdeling i flytter patienterne over på eventuelt en afdeling der ligner ortopædkirurgisk?}.
\noindent

Overflytningen og ophold i samtalerum samt flugtvejsgange bevirker til, at patienterne oplever et skærpet privatliv. \cite{Madsen2014} Derudover kan det belaste fysiske og psykiske forhold for patienterne såvel som pårørende. \cite{Heidmann2014} Som nævnt i afsnit \ref{Tilret} øges risikoen for fejl ved overbelægning og dertil ses det at mortalitetsraten øges med $1,2~\%$ ved en overskridelse af sengebelægningskapaciteten på $10~\%$ ifølge et dansk studie fra år 2014. \cite{Madsen2014} Hertil understreges det, at der kan være flere parametre, der påvirker mortaliteten og det nødvendigvis ikke er overbelægning der er den primære årsag til øget mortalitet. Overbelægning giver derfor et forøget pres for at få patienterne udskrevet, således at der opnås en normalbelægning og  den fysiske kapacitet ikke overstiges. \fxnote{Tidligere afsnit: Hvordan er fordelingen af elektive og akutte patinter? Kan elektive patienter tages ind før der er normalbelægning?} 

 
\subsection{Juridiske problemstillinger}
Udover omstrukturering af arbejdsopgaver samt omflytning af patienter, er der nogle juridiske problemstillinger, som skal overholdes for at disse ændringer er lovmæssigt sikre. Dette medvirker tilkaldelse af brandvagter for at skærpe sikkerheden ift. evakuering under brand. Derudover er det samtidigt vigtigt, at det bestræbes på at overholde løn- og overenskomsterne for sundhedspersonalet.

\subsubsection{Brandsikkerhed}
Ved overbelægning kontaktes en brandvagt, således at brandvagten kan være tilgængelig på afdelingen så snart overbelægningen finder sted. Hvis normal belægningstilstand er mulig inden for fire timer efter overbelægningen påbegyndes, er det ikke nødvendigt at tilkalde en brandvagt. En brandvagt kan højst overvåge to afdelinger på samme etage, hvorfor det kan være nødvendigt at der indkaldes flere. Det er afdelingens ansvar at afvikle overbelægningen hurtigst muligt ved at udskrive patienter eller overflytte patienter til sengestuer på andre afdelinger. \cite{Beredskab2016} Foruden at være et juridisk problem, fratages der penge fra det overordnede budget for ortopædkirurgisk afdeling hver gang en brandvagt tilkaldes. \fxnote{Sygeplejerske/Sten/Økonomiafdeling: dette skal undersøges om det er fra det samlede budget eller om det bliver taget fra ortopædkirurgisk afdeling?} 

\subsubsection{Overenskomster}
Som tidligere nævnt i afsnit \ref{Tilret}, forlænges sundhedspersonalets arbejdsdag under overbelægning. En normal arbejdsuge for danske sygeplejersker er 37 timer. I tilfælde af overarbejde må en arbejdsuge for en sygeplejerske, ifølge arbejdstids aftalen indgået med Dansk Sygpelerråd, ikke overstige 48 timer. Planlægning af en sygeplejerskes normalarbejde skal derudover finde sted 24 timer før fremmøde, hvilket ikke inkluderer overarbejde. \cite{Danske2015} \fxnote{Sygeplejersker: Hvordan prioriteres pauser under overbelægning} Dette medfører nogle juridiske overvejelser, som skal tages i betragtning når sundhedspersonalet tilkaldes ekstraordinært, således at overenskomster og brandsikkerhed overholdes. 


\subsubsection{Opsummering}
Overbelægning medfører at sundhedspersonalets arbejdsbyrde forøges, hvilket resulterer i kortere tid til den enkelte patient, hvormed den tilsigtede pleje svækkes. \cite{Dinges2004,Aiken2002} Derudover kan der opstå flere fejl fra personalet ved overarbejde.\cite{Dinges2004} Da patienterne skal flyttes til samtalerum og flygtvejsgange kan dete belaste patienten både fysisk og psykisk. \cite{Madsen2014,Heidmann2014} Dette kan skabe et øget pres for at få patienterne udskrevet for hurtigst muligt at opnå en normalbelægning. Udover disse konsekvenser skal de lovmæssige regler overholdes, hvilket kan være omkostningsfuldt for ortopædkirurgisk afdelingen, ift. indkaldelse af brandvagter samt overenskomster for sundhedspersonalet. \cite{Beredskab2016,Danske2015} Det er derfor både omkostningsfuldt og har konsekvenser for ortopædkirurgisk afdeling ved overbelægning.
