\section{Ubalance i kapacitetsudnyttelse}
Ved kapacitetsmangel på ortopædkirurgisk afdeling forekommer en omstrukturering af personalets arbejdsopgaver, som sikre patientens behov, opretholdelse af kvalitet og udnyttelse af kompetencer. Dette er med henblik på at opnå en balance mellem de ressourcer og de krav, der stilles i den pågældende situation.\cite{Bjerg2016}  %Generelt er personalet den begrænsende faktor for planlægning og kapacitetsudnyttelse\cite{Company2013}. 

\subsection{Arbejdsvilkår} \label{Per_sik}
%Hvordan påvirkes personalets arbejdsrutiner ved kapacitetsmangel? Hvor længes må deres arbejdsdag maksimalt være? Stiger antallet af fejl når personalet varetager en større arbejdsbyrde end planlagt? Hvor omkostningsfuldt er det? 

I tilfælde af kapacitetsmangel er der udarbejdet en arbejdstilrettelæggelse af Region Nordjylland for personalet på ortopædkirurgisk afdeling. Ved kapacitetsmangel påtager lederen, eller dennes stedfortræder, ansvaret for at finde en løsning på dette problem. Dette kan betyde, at det afgående vagthold skal blive indtil en midlertidig løsning er fundet eller en tidligere indkaldelse af det næste vagthold. I nogle tilfælde kan det være nødvendigt at låne ressourcer fra andre afsnit eller indkalde personale fra vikarbureauet. Derudover undersøges det, hvorvidt behandlingen af elektive patienter kan aflyses.\cite{Bjerg2016} 

Ved overarbejde må en arbejdsuge for en sygeplejerske, ifølge arbejdstidsaftalen indgået med Dansk Sygeplejeråd, ikke overstige 48 timer\cite{Danske2015}. Pauserne holdes, således de passer ind i arbejdsrytmen og nogle dage holdes der ikke pause. Herudover kan sygeplejerskene tilkaldes fra en pause, hvis det er nødvendigt. Hvis sundhedspersonalet er nødsaget til at arbejde længere end den normale arbejdstid, viser dette sig at have en negativ indvirkning på personalets arbejdesopgaver\cite{Dinges2004}. Overarbejde kan resultere i en presset arbejdsdag og dermed en forringet kvalitet af behandlingen\cite{Kjeldsen2015}. Dertil mener hver anden regionalt ansat sygeplejerske på tværs af regionerne, at den travle arbejdsdag påvirker patienternes sikkerhed\cite{Kjeldsen2015}.

%Den større patientbyrde, der skal varetages ved en belægning over 100\%, medvirker til en øget arbejdsbyrde for det tilstedeværende personale.\fxnote{Spørgsmål til sygeplejersker: Hvor mange patienter skal i under normale omstændigheder varetage? Og hvordan fungerer det under overbelægning? Fordeles de enkelte patienter mellem jer?} \cite{Dinges2004,Aiken2002,Madsen2014}
%Det er forsøgt at kapacitsoptimere, med en reduktion af sengepladser på $30~\%$ i perioden $1996$ til $2011$ \cite{Dinges2004,Aiken2002,Madsen2014} 


\subsection{Patientsikkerhed}
%Hvordan påvirker kapacitetsmangel ift. brandsikkerhed? Laver personalet flere fejl, som går ud over patienterne ved overbelægning? Hvordan fungerer omrokering af patienter (gangarealer, vaskerum ect.)? Hvor omkostningsfuldt er det for OA’s budget? brandsikkerhed og Genindlæggelse
Under perioder med kapacitetsmangel er det ofte nødvendigt at overflytte patienter til andre afdelinger, gangarealer eller fyldte stuer, herved er det ofte patienter, der snart udskrives, der overflyttes. Ved flytning af patienter, flyttes de indbyrdes mellem afdelingens afsnit eller til andre matrikler i eksempelvis Farsø, Hjørring eller Frederikshavn. Overordnet ønskes det at beholde børn, traume- og rygpatienter på afdelingen. Eksempelvis kan der ved rygpatienter opstå ændringer samt komplikationer ift. udstyr og varetagelse, hvilket kan resulterer i, at patienten er nødsaget til at starte forfra med forløbet. Ved overflytning til andre matrikler er det oftest færdigtbehandlede patienter, der kan starte på genoptræning eller videre mobilisering. Overflytningen kan belaste både fysiske og psykiske forhold for patienter såvel som pårørende\cite{Heidmann2014}. Herunder kan skærpet privatliv forekomme hos patienter, der er flyttet til gangarealer eller fyldte stuer\cite{Madsen2014}. 

Som nævnt i afsnit \ref{Per_sik} forringes kvaliteten af behandlingen ved overarbejde, dertil ses det ligeledes, at mortalitetsraten øges med $1,2~\%$ ved en overskridelse af belægningen med $10~\%$, ifølge et dansk studie fra år $2014$\cite{Madsen2014}. Hertil understreges det, at der kan være ukendte parametre, der påvirker mortalitetsraten, og det nødvendigvis ikke er belægning, der er den primære årsag til en øget mortalitet. For at undgå forringet kvalitet af behandling forsøges det at få patienterne udskrevet tidligere, således et ønske om balance mellem aktivitet og kapacitet opnås.

%\fxnote{Tidligere afsnit: Hvordan er fordelingen af elektive og akutte patinter? Kan elektive patienter tages ind før der er normalbelægning?} 

Der tilkaldes en brandvagt til afdelingen, hvis en belægning over $100~\%$ har fundet sted i over $4$ timer for således at sikre patienterne ved evakuering under brand. En brandvagt kan højest overvåge to afdelinger på samme etage, hvorfor det kan være nødvendigt, at der indkaldes flere. Det er afdelingens ansvar at afvikle belægningsproblemet og kapacitetsmanglen hurtigst muligt ved at udskrive patienter eller overflytte patienter til andre afdelinger. Hver gang der tilkaldes en brandvagt faktureres dette af Aalborg Universitetshospital.\cite{Beredskab2016}


