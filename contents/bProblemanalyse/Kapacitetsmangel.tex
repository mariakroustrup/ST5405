\section{Betydning af kapacitetsmangel}
Hvis der opstår kapacitetsmangel på ortopædkirurgisk afdeling sker der en omstrukturering af personalets arbejdsopgaver som sikre patientens behov, opretholdelse af kvalitet og udnyttelse af kompetencer. Generelt er personalet den begrænsende faktor for planlægning og kapacitetsudnyttelse\cite{Company2013}. Dette er med henblik på at finde en balance mellem de ressourcer og de krav det er i den pågældende situation. \cite{Bjerg2016} 

\subsection{Personalesikkerhed} \label{Per_sik}
%Hvordan påvirkes personalets arbejdsrutiner ved kapacitetsmangel? Hvor længes må deres arbejdsdag maksimalt være? Stiger antallet af fejl når personalet varetager en større arbejdsbyrde end planlagt? Hvor omkostningsfuldt er det? 

I tilfælde af ekstrem underkapacitet er der udarbejdet en arbejdstilrettelæggelse Region Nordjylland for personalet på ortopædkirurgisk afdeling. Ved kapacitetsmangel påtager lederen eller dennes stedfortræder ansvaret for at finde en løsning på dette problem. Dette kan betyde at det afgående vagthold skal blive indtil en midlertidig løsning er fundet, samt indkalde det næste vagthold tidligere. I nogle tilfælde kan det være nødvendighed at låne ressourcer fra andre afsnit eller indkalde personale fra vikarbureauet. Derudover undersøges det om behandlingen af elektive patienter kan aflyses.\cite{Bjerg2016} I tilfælde af overarbejde må en arbejdsuge for en sygeplejerske, ifølge arbejdstidsaftalen indgået med Dansk Sygeplejeråd, ikke overstige 48 timer\cite{Danske2015}.  \fxnote{Spørgsmål til sygeplejersker: Hvordan prioriteres pauser under overbelægning?} Hvis sundhedspersonalet er nødsaget til at arbejde længere end den normale arbejdstid, viser dette sig at have en negativ indvirkning på personalet.\cite{Dinges2004} Dette resulterer i en presset arbejdsdag og dermed en forringet kvalitet af behandlingen. Dertil menner hver anden regionalt ansat sygeplejerske på tværs af regionerne, at den travle arbejdsdag påvirker patienternes sikkerhed.\cite{Kjeldsen2015} 

Derudover medvirker den større patientbyrde, som det tilstedeværende sundhedspersonale skal varetage under et belægningsproblem, til en øget arbejdsbyrde. \fxnote{Spørgsmål til sygeplejersker: Hvor mange patienter skal i under normale omstændigheder varetage? Og hvordan fungerer det under overbelægning? Fordeles de enkelte patienter mellem jer?} Det er forsøgt at kapacitsoptimere, med en reduktion af sengepladser på $30~\%$ i perioden $1996$ til $2011$.\cite{Dinges2004,Aiken2002,Madsen2014} 


\subsection{Patientsikkerhed}
%Hvordan påvirker kapacitetsmangel ift. brandsikkerhed? Laver personalet flere fejl, som går ud over patienterne ved overbelægning? Hvordan fungerer omrokering af patienter (gangarealer, vaskerum ect.)? Hvor omkostningsfuldt er det for OA’s budget? brandsikkerhed og Genindlæggelse
Når kapaciteten ikke er tilstrækkelig og det er nødvendigt at overflytte patienter, overflyttes ofte de patienter, som er på grænsen til at blive udskrevet \fxnote{Sygeplejersker: Vi vil gerne høre om der prioriteres i forhold til hvilke patienter der flyttes. Er der en bestemt afdeling i flytter patienterne over på eventuelt en afdeling der ligner ortopædkirurgisk?}.
\noindent
Overflytningen af patienter bevirker til, at de oplever et skærpet privatliv. \cite{Madsen2014} Derudover kan det belaste fysiske og psykiske forhold for patienterne såvel som pårørende. \cite{Heidmann2014} Som nævnt i afsnit \ref{Per_sik} øges risikoen for fejl ved et belægningproblem og dertil ses det at mortalitetsraten øges med $1,2~\%$ ved en overskridelse af sengebelægningskapaciteten på $10~\%$ ifølge et dansk studie fra år 2014. \cite{Madsen2014} Hertil understreges det, at der kan være flere parametre \fxnote{Hvilke parametre????}, der påvirker mortaliteten og det nødvendigvis ikke er overbelægning der er den primære årsag til øget mortalitet. Overbelægning giver derfor et forøget pres for at få patienterne udskrevet, således at der opnås en normalbelægning og  den fysiske kapacitet ikke overstiges. \fxnote{Tidligere afsnit: Hvordan er fordelingen af elektive og akutte patinter? Kan elektive patienter tages ind før der er normalbelægning?} 


På grund af de ekstra patienter tilkaldes en brandvagt til afdelingen for at sikre patienten ved evakuering under brand. Hvis normal belægningstilstand er mulig inden for fire timer efter belægningsproblemet påbegyndes, er det ikke nødvendigt at tilkalde en brandvagt. En brandvagt kan højst overvåge to afdelinger på samme etage, hvorfor det kan være nødvendigt at der indkaldes flere. Det er afdelingens ansvar at afvikle overbelægningen hurtigst muligt ved at udskrive patienter eller overflytte patienter til sengestuer på andre afdelinger. \cite{Beredskab2016} Foruden at sikre patienten under et belægningsproblem er det omkostningsfuldt for ortopædkirurgisk afdeling hver gang en brandvagt tilkaldes. \fxnote{Sygeplejerske/Sten/Økonomiafdeling: dette skal undersøges om det er fra det samlede budget eller om det bliver taget fra ortopædkirurgisk afdeling?} 


