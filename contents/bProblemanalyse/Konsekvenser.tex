\subsection{Konsekvenser af overbelægning}


\subsubsection{Personle og patienter}
Overbelægning på hospitaler, hvilket medfører forlængede arbejdsdage\fxnote{lav indskudt sætning om}, viser at have en negativ indvirkning på sundhedspersonalet. \citep{Kjeldsen2015} \citep{Dinges2004} Overbelægning resulterer i, at sundhedspersonalet har kortere tid til den enkelte patient, hvilket ligeledes øger risikoen for fejl. Fejlene opstår hovedsageligt, når personalet har arbejdsdage på mere end 12 timer.\fxnote{tydeliggør at det ikke er et dansk studie} \citep{Dinges2004} Dette medvirker til genindlæggelser og flere sengedage for patienterne, og kan føre til forlængelse af patienternes indlæggelsestid. Et amerikansk studie fra år 2014 indikerer en forøgelse af indlæggelsestiden ved forøget arbejdsbyrde for sundhedspersonalet\citep{Elliott2014}. Det viser sig dog ikke at være muligt at effektivisere sundhedspersonale \fxnote{de kan godt effektiveres, skriv at de ikke kan have flere patienter samt at de får depressioner}, da en dansk undersøgelse viser, at sygeplejerskernes arbejdsbyrde er forøget med 40\% fra 2001 til 2015.\citep{Kjeldsen2015}  \fxnote{omskriv afsnit}

\subsubsection{Fysiske forhold}

Overbelægning har en negativ effekt på både sundhedspersonalets og patienternes psyke. Ved forøgelse i patientbyrde, erfares det, at sandsynlighed for, at sundhedspersonalet brænder ud forøges\fxnote{slet brænd ud - arbejsbyrde, mister lysten til at arbejde}. \citep{Aiken2002} 
Patienterne på gangene har ikke mulighed for at få samme privatliv som i stuerne. \citep{Madsen2014} \textit{"Overbelægningen medfører uværdige, utrygge og stærkt kritisable forhold for patienterne}" - Bjarne Hastrup, administrerende direktør for Ældre Sagen. \citep{Politiken2013}
Grundet mangel på sundhedspersonale ift. antal patienter, kan der eksempelvis ved brand være forøget risiko for ikke at kunne evakuere alle patienterne. \citep{Madsen2014} \fxnote{uddyb......... inddrag citat...........}

\subsubsection{Dødelighed}
Ifølge et dansk studie fra år 2014, øges risikoen for dødeligheden på hospitalet med ca. 1,2 \% ved en forøgelse af senge belægning på 10\%. \citep{Madsen2014} \fxnote{vend sætningen} Et canadisk studie viser ligeledes, at en forøgelse med 10\% i belægning på akutafdelingen vil medføre en øget dødelighed og flere genindlæggelser.\citep{McCusker2014} Disse tal sammenlignes ikke direkte med danske tal, da der er tale om forskellige sundhedsvæsener, men tendensen kan dog ses som en indikation for, at overbelægning på sygehuse har en negativ effekt på mortalitetsraten.

herefter er det påvist at sandsynligheden for hospitalets mortalitet stiger 9\% sammenlignet med ved underbelægning. \citep{Madsen2014}
Mortaliteten angiver forholdet mellem antal døde indenfor et givet tidspunkt og størrelsen af en befolkning. \citep{denstoredanskeordbog2} Et studie beskriver dog at denne reduktion ikke nødvendigvis har en betydning for mortaliteten, men kan være et tegn på effektivisering af sygehusene. \citep{Madsen2014}
