\subsection{Konsekvenser af overbelægning}


\subsubsection{Personle og patienter}
Overbelægning på sygehuse, hvilket medfører forlængede arbejdsdage, viser at have en negativ indvirkning på sundhedspersonalet. \citep{Kjeldsen2015} \citep{Dinges2004} Overbelægning af patienter resulterer i, at sundhedspersonalet har kortere tid til den enkelte patient, hvilket ligeledes øger risikioen for fejl. Fejlene opstår hovedsageligt, når personalet har arbejdsdage på mere end 12 timer. \citep{Dinges2004} Dette medvirker til genindlæggelser og flere sengedage for patienterne, og kan føre til forlængelse af patienternes indlæggelsestid. Et amerikansk studie fra 2014 indikerer en forøgelse af indlæggelsestiden ved forøget arbejdsbyrde for sundhedspersonalet\citep{Elliott2014}. Det viser sig dog ikke at være muligt at effektivisere sundhedspersonale, da en dansk undersøgelse viser, at sygeplejerskernes arbejdsbyrde er forøget med 40\% fra 2001 til 2015.\citep{Kjeldsen2015}  

\subsubsection{Fysiske forhold}

Overbelægning har herudover negativ effekt på både sundhedspersonalets og patienternes psyke. Det viser sig at ved forøgelse i patientbyrde, erfares det at sandsynlighed for at sundhedspersonalet brænder ud forøges. \citep{Aiken2002} 
Patienterne på gangene har ikke mulighed for at få samme privatliv som i stuerne.\citep{Madsen2014} \textit{"Overbelægningen medfører uværdige, utrygge og stærkt kritisable forhold for patienterne}" - Bjarne Hastrup, administrerende direktør for Ældre Sagen. \citep{Politiken2013}
Grundet mangel på sundhedspersonale ift. antal patienter, kan der eksempelvis ved brand være forøget risiko for ikke at kunne evakuere alle patienterne. \citep{Madsen2014}

\subsubsection{Dødelighed}
Ifølge et dansk studie fra 2014, forøges risikoen for dødeligheden på hospitalet med ca. 1,2 \% ved en forøgelse af senge belægning på 10\%. \citep{Madsen2014} Et canadisk studie viser en forøgning af belægning på 10\% på akutafdelingen vil medføre en øget dødelighed og genindlæggelserate.\citep{McCusker2014} Disse tal er ikke helt sammenlignelige med Danmark, da der er tale om forskellige afdelinger. Tendensen kan dog ses som en indikation for at overbelægning på sygehuse har en negativ effekt på mortalitetsraten.