%%%%% Mangler fxnote

%%% OMFLYTNING AF PATIENTER
%% Nogle af fxnotes skal skrives efter samtale med flere sygeplejersker
%% Der skal skrives noget om elektive patienter ud fra et tidligere et afsnit som ikke er skrevet endnu.

%%% TILRETTELÆGGELSE AF PERSONALE
%% Sygeplejersker: Hvor mange patienter skal i under normale omstændigheder varetage? Og hvordan fungerer det under overbelægning fordeles de enkelte patienter mellem jer? Erfaring?


%%% JURIDISKE PROBLEMSTILLINER:
%Hvis overbelægning er mere end 50 dage brydes loven (hvis det med elektive patienter kommer ind i omflytning af patienter.)
% Sygeplejerske/Sten/Økonomiafdeling: dette skal undersøges om det er fra det samlede budget eller om det bliver taget fra ortopædkirurgisk afdeling?
% Sygeplejersker: Hvordan prioriteres pauser under overbelægning


\section{Konsekvenser og omkostninger ved overbelægning}
Ved overbelægning er det vigtigt, at situationen afvikles straks, da det bl.a. kan skabe problemer ift. arbejdsopgaver samt give et skærpet privatliv for patienten og forøgelse i mortalitetsrate. \cite{Madsen2014}
Ved overbelægning tilkaldes og tilrettelægges arbejdet for personalet, og der stræbes på at finde en balance mellem de eksisterende ressourcer, og de krav der stilles til den enkelte afdeling. \cite{Bjerg2016} Derudover omflyttes  patienter ud fra nogle konkrete retningslinjer. \cite{Beredskab2016} Herudover skal lovgivninger om sikkerhed og overenskomster for sundhedspersonalet overholdes. \cite{Beredskab2016}


\subsection{Tilrettelæggelse af personale}
Under overbelægning gælder en arbejdstilrettelæggelse som er udarbejdet af Region Nordjylland og omfatter ortopædkirurgisk afdeling. Dennes formål er, at sikre patientens behov, kvalitetssikring, udnyttelse af kompetencer med henblik på at finde en balance mellem ressourcerne og krav i den pågældende situation. Ved overbelægning påtager lederen eller dennes stedfortræder ansvaret for at finde en løsning, dette vil i første omgang betyde, at det afgående vagthold skal blive indtil en midlertidig løsning er fundet. Derudover undersøges det om elektive patienter kan aflyses, og hvorvidt der bør indkaldes personale til at dække det manglede fremmøde. Dertil kan det være nødvendigt at indkalde det næste vagthold tidligere. Det kan blive en nødvendighed, at låne ressourcer fra andre afsnit eller indkalde personale fra vikarbureauet. Den sidste løsning er, at forlænge arbejdstiden for flere på det afgående vagthold til situationen kan varetages af de resterende. \cite{Bjerg2016}

\noindent 
Ved normale omstændigheder varetages XX antal patienter pr. sygeplejerske. De ekstra patienter, der behandles under overbelægning bliver fordelt mellem det tilstedeværende sundhedspersonale, og overskrider dermed den forventede patientbyrde. \fxnote{Sygeplejersker: Hvor mange patienter skal i under normale omstændigheder varetage? Og hvordan fungerer det under overbelægning fordeles de enkelte patienter mellem jer? Erfaring?}
Dette resultere i at sundhedspersonalet har kortere tid til den enkelte patient, hvorfor risikoen for fejl øges, hvorved det ikke er muligt at give patineterne den optimale behandling. \cite{Dinges2004} \cite{Aiken2002} Udover overbelægning har reduceringen af sengepladser på $30~\%$ yderligere medvirket til, at øge sygeplejerskernes arbejdsbyrde med $40~\%$ fra år 2001 til 2015. \cite{Kjeldsen2015}


Hvis sundhedspersonalet er nødsaget til, at blive udover den normale arbejdstid, viser dette sig at have en negativ indvirkning på personalet.\cite{Kjeldsen2015} \cite{Dinges2004} Undersøgelser viser, at dette resulterer i en stresset arbejdsdag og dermed en forringet kvalitet af behandlingen. cite{Kjeldsen2015} Et amerikansk studie har på baggrund af undersøgelser fra år 2002 påvist, at fejlene hovedsageligt opstår, når personalet har arbejdsdage på mere end 12 timer.\cite{Dinges2004}. Et andet amerikansk studie fra år 2014 indikerer en forøgelse af indlæggelsestiden ved forøget arbejdsbyrde for sundhedspersonalet\cite{Elliott2014}. Begge undersøgelser skal ses i perspektiv med de danske overenskomster for sundhedspersonalet. Det fremgår dog af Dansk Sygeplejeråd, at hver anden regionalt ansat sygeplejerske på tværs af regionerne mener, at den travle arbejdsdag påvirker patienternes sikkerhed \cite{Kjeldsen2015}. 


\subsection{Omflytning af patienter}
For ortopædkirurgisk afdeling på Aalborg Universitetshospital er der opstillet retningslinjer af Nordjyllands Beredskabsstyrelse for, hvordan overbelægningen varetages. I første omgang forsøges det, at udfylde alle stuerne på afdelingen. Hvis dette ikke er muligt, flyttes patienter til andre hospitalsafdelinger. Når der ikke er plads på de andre hospitalsafdelinger placeres patienterne i samtalerum og flugtvejsgange som en midlertidig nødløsning. \cite{Beredskab2016} De patienter, der overflyttes er ofte patienter, som er på grænsen til at blive udskrevet \fxnote{Sygeplejersker: Vi vil gerne høre om der prioriteres i forhold til hvilke patienter der flyttes. Er der en bestemt afdeling i flytter patienterne over på eventuelt en afdeling der ligner ortopædkirurgisk?}.


\noindet
Overflytningen og ophold i samtalerum samt flugtvejsgange bevirker til, at patienterne oplever et skærpet privatliv. \cite{Madsen2014} Derudover kan det belaste fysiske og psykiske forhold for patienterne såvel som pårørende. \cite{Heidmann2014} Ved overbelægning øges mortaliteten med $1,2~\%$ ved en øget sengebelægning på $10~\%$ ifølge et dansk studie fra år 2014. \cite{Madsen2014} Hertil skal det understreges, at der kan være flere faktorer, der spiller ind på mortaliteten og det nødvendigvis ikke er overbelægning der er den primære grund. Overbelægning giver derfor et forøget pres for at få patienterne udskrevet, således at der opnås en normalbelægning. \fxnote{Tidligere afsnit: Hvordan er fordelingen af elektive og akutte patinter? Kan elektive patienter tages ind før der er normalbelægning?}.


\subsection{Juridiske problemstillinger}
Udover omstrukturering af arbejdsopgaver samt omflytning af patienter er der nogle juridiske problemstillinger, som skal overholdes for at disse ændringer er lovmæssigt sikkert. Dette medvirker tilkaldelse af brandvagter for at skærpe sikkerheden ift. evakuering under brand. Derudover er det samtidigt vigtigt, at det bestræbes på, at overholdelse løn- og overenskomsterne for sundhedspersonalet.

\subsubsection{Brandsikkerhed}
Ved opdagelse af et belægningsproblem kontaktes en brandvagt, således at brandvagten kan være tilgængelig på afdelingen så snart overbelægningen finder sted. Hvis normal belægningstilstand er mulig inden for fire timer efter overbelægningen sker, er det ikke nødvendigt at tilkalde en brandvagt. En brandvagt kan højst overvåge to afdelinger på samme etage, hvorfor det kan være nødvendigt at der indkaldes flere. Det er afdelingens pligt, at  afvikle overbelægningen hurtigst muligt ved at udskrive patienter eller overflytte patienter til sengestuer på andre afdelinger. \cite{Beredskab2016} For uden, at være et juridisk problem, fratages der penge fra det overordnede budget for ortopædkirurgisk afdeling hver gang en brandvagt tilkaldes. \fxnote{Sygeplejerske/Sten/Økonomiafdeling: dette skal undersøges om det er fra det samlede budget eller om det bliver taget fra ortopædkirurgisk afdeling?} 

\subsubsection{Overenskomster}
Som tidligere nævnt, forlænges sundhedspersonalets arbejdsdag under overbelægning. En normal arbejdsuge for danske sygeplejersker er på 37 timer. I tilfælde af overarbejde må en arbejdsuge for en sygeplejerske, ifølge arbejdstids aftalen indgået med Dansk Sygpelerråd, ikke overstige 48 timer. Planlægning af en sygeplejerskes normalarbejde skal derudover finde sted 24 timer før fremmøde, hvilket ikke inkluderende overarbejde. \cite{Danske2015} \fxnote{Sygeplejersker: Hvordan prioriteres pauser under overbelægning} Dette medfører nogle juridiske overvejelser, som skal tages i betragtning når sundhedspersonalet tilkaldes ekstraordinært. 

\noindent
\fxnote{Hvilke 3 ting er problemet opsummer dette - skriv måske om?? }
Overbelægning er både omkostningsfuldt og kræver omstrukturering samt skaber forringelse for patienten såvel som arbejdsmiljøet for sundhedspersonalet. Det kan være nødvendigt, at omstrukturere arbejdsformen og miljøet pga.  overbelægning hvor det bestræbes på at udnytte de ressourcer der til rådighed, og opfylde de krav der er til afdelingen både juridisk samt sikkerhedsmæssigt. 
