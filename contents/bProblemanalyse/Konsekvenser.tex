\subsection{Konsekvenser af overbelægning}


\subsubsection{Personle og patienter}
Overbelægning på hospitaler viser at have en negativ indvirkning på sundhedspersonalet, da det medfører forlængede arbejdsdage.\citep{Kjeldsen2015} \citep{Dinges2004} Overbelægning resulterer i, at sundhedspersonalet har kortere tid til den enkelte patient, hvilket ligeledes øger risikoen for fejl. Ifølge et amerikansk studie fra år 2002 opstår fejlene hovedsageligt, når personalet har arbejdsdage på mere end 12 timer.\citep{Dinges2004} Dette medvirker til genindlæggelser og flere sengedage for patienterne, og kan føre til forlængelse af patienternes indlæggelsestid. Et amerikansk studie fra år 2014 indikerer en forøgelse af indlæggelsestiden ved forøget arbejdsbyrde for sundhedspersonalet\citep{Elliott2014}. Sygeplejerskernes arbejdsbyrde er ifølge en dansk undersøgelse øget med 40\% fra år 2001 til år 2015. Undersøgelsen viser desuden, at dette resulterer i en stresset arbejdsdag og dermed en ringere kvalitet af behandlingen.\citep{Kjeldsen2015}  

\subsubsection{Fysiske forhold}

Overbelægning har en negativ effekt på både sundhedspersonalets og patienternes psyke. Det erfares at sunhedspersonalet bliver stressede og ikke er i stand til at give patienterne den optimale behandling ved forøget patienbyrde. \citep{Aiken2002} 
Ved overbelægning kan det være en nødvendighed at flytte nogle af patienterne ud på gangene. Disse patienter har ikke det samme privatliv som i stuerne. Herudover kan dette være en hindring for de resterende patienter i tilfælde af brand, hvilket ligeledes vil øge risikoen for at sundhedspersonalet ikke vil være i stand til at evakuere alle patienter i tide. \citep{Madsen2014} Følgende citat viser, hvilken påvirkning overbelægning har for patienterne. 

   \citep{Madsen2014} \textit{"Overbelægningen medfører uværdige, utrygge og stærkt kritisable forhold for patienterne}" - Bjarne Hastrup, administrerende direktør for Ældre Sagen. \citep{Politiken2013}


\subsubsection{Dødelighed}
Ifølge et dansk studie fra år 2014, resulterer en øget belægning på 10\% i en øget risiko for dødelighed på hospitalet med 1,2\%. \citep{Madsen2014}  Et canadisk studie viser ligeledes, at en forøgelse med 10\% i belægning på akutafdelingen vil medføre en øget dødelighed og flere genindlæggelser.\citep{McCusker2014} Disse tal sammenlignes ikke direkte med danske tal, da der er tale om forskellige sundhedsvæsener, men tendensen kan dog ses som en indikation for, at overbelægning på sygehuse har en negativ effekt på mortalitetsraten, der angiver forholdet mellem antal døde indenfor et givet tidspunkt og størrelsen af en befolkning. \citep{denstoredanskeordbog2}  