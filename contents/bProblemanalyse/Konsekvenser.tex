\section{Konsekvenser og omkostninger ved overbelægning}
- Ved overbelægning sættes dominoeffekt i gang for at få styr på situationen
- bla bla bla. 
- Dette omfatter omflytning af patienter,ny tilrettelæggelse af personalets arbejde og giver nogle juridiske udfordringer... Evakuering under brand...

\subsection{Omflytning af patienter}
Ved overbelægning på Aalborg Universitetshospital, hvilket også gør sig gældende for ortopædkirurgisk afdeling, er der opstillet retningslinjer af Nordjyllands Beredskabsstyrelse for, hvordan overbelægningen varetages. I første omgang forsøges det at udfylde alle stuerne på afdelingen, hvis dette ikke er muligt flyttes de resterende patienter til andre hospitalsafdelinger. Hvis ingen af de nævnte løsninger er mulige placeres patienterne i samtalerum og flugtvejsgange som en midlertidig nødløsning. Hvis der på et senere tidspunkt skulle blive plads, skal patienten straks flyttes tilbage på stuerne. \cite{Beredskab2016} Overflytningen af patienter til andre afdelinger på hospitalet kræver et overblik og kan være tidskrævende samt give komplikationer ift. behandling af patienten, da patienterne ikke altid befinder sig på den afdeling, hvor de er indlagt. \cite{KILDE}


- Personale der kan flytte dem
- Forvirring for personalet - fejl ift. at vide hvor patienterne befinder sig. 
- Hvordan sker prioritering af patienter, hvilke patienter flyttes til gangene?
- Patienter: øget pres for at komme hjem, skærpet privatliv, forvirring 
%De ekstra patienter, der ligger på stuerne, gangene og vaskerummene, pga. overbelægning, er en større udfordring ved evakuering under brand. Pladsmangel, som medfører, at patienterne opholder sig i vaskerummene og på gangene, bevirker desuden til, at patienterne oplever et skærpet privatliv. \cite{Madsen2014}
- Mortalitetsraten stiger - der er mange faktorer der spiller ind og det nødvendigvis ikke kun er overbelægning
%Ifølge et dansk studie fra år 2014, øges risikoen for mortalitet med $1,2~\%$ ved en øget belægning på $10~\%$. \cite{Madsen2014}. Husk at der er flere faktorer det spiller ind og overbelægning er nødvendigvis ikke dødsårsagen. 


\subsubsection{Tilrettelæggelse af personale}
Under overbelægning gælder en arbejdstilrettelæggelse som sikre patientens behov, kvalitetssikring, udnyttelse af kompetencer med henblik på at finde en balance mellem ressourcerne og krav i den pågældende situation. Ved overbelægning påtager lederen eller dennes stedfortræder ansvaret for at finde en løsning, dette vil i første omgang betyde at det afgående vagthold skal blive indtil en løsning er fundet. Det undersøges om planlagte patienter kan aflyses. Derudover indkaldes der personale til at dække det manglede fremmøde i de timer, hvor det er nødvendigt samt indkalde det nye vagthold tidligere. Det er muligt at låne ressourcer fra andre afsnit/klinikker i det omfang det er muligt eller indkalde personale fra vikarbureauet. Den sidste løsning er, at forlænge arbejdstiden for flere på det afgående vagthold til situationen kan varetages af de resterende. \cite{Bjerg2016}

Disse løsninger kan have konsekvenser for personalets varetagelse af patienterne.... \cite{KILDE- }

- Hvor lang tid har personalet til den enkelte patient normalt?
- Tilkaldelse af mere personalet flere skal have løn/overarbejde?
- Hvordan med ferie og fridage?
- 
%Overbelægning på hospitaler medfører forlængede arbejdsdage, hvilket viser sig at have en negativ indvirkning på sundhedspersonalet.\cite{Kjeldsen2015} \cite{Dinges2004} Overbelægning resulterer i, at sundhedspersonalet har kortere tid til den enkelte patient, hvorfor risikoen for fejl øges. %Overbelægning medvirker derfor til flere sengedage for patienterne. I takt med reduceringen af sengepladser på $30~\%$ som beskrevet i \autoref{sec:overbelaegning}, er sygeplejerskernes arbejdsbyrde øget med $40~\%$fra år 2001 til 2015 ifølge en dansk undersøgelse.
% Ifølge et amerikansk studie fra år 2002 opstår fejlene hovedsageligt, når personalet har arbejdsdage på mere end 12 timer.\cite{Dinges2004} Et andet amerikansk studie fra år 2014 indikerer en forøgelse af indlæggelsestiden ved forøget arbejdsbyrde for sundhedspersonalet\cite{Elliott2014}. Hvor længe må sygeplejerske arbejde ift. overenskomst osv.??? 
%Ifølge en undersøgelse fra Dansk Sygeplejeråd, mener hver anden regionalt ansat sygeplejerske på tværs af regionerne, at den travle arbejdsdag går ud over patienternes sikkerhed \cite{Kjeldsen2015}.

\subsubsection{Juridisk problemstillinger}
Ved opdagelse af et belægningsproblem kontaktes en brandvagt, således at brandvagten kan være tilgængelig på afdelingen så snart overbelægningen finder sted. Hvis normal belægningstilstand er mulig inden for fire timer efter overbelægningen sker er det ikke nødvendigt at tilkalde en brandvagt. En brandvagt kan højst overvåge to afdelinger på samme etage, hvorfor det kan være nødvendigt at der indkaldes flere brandvagter. Det er afdelingens pligt at  afvikle overbelægningen hurtigt muligt ved at udskrive patienter eller overflytte patienter til sengestuer på andre afdelinger. \cite{Beredskab2016} For uden, at være et juridisk problem, fratages der penge fra det overordnede budget for ortopædkirurgisk afdeling hver gang en brandvagt tilkaldes. \cite{KILDE - er i tvivl om det går ud over den enkelte afdeling eller om det er samlet budget for sygehuset??}.

- Omkostningsfuldt: budget for ortopædkirurgisk.
- Krav på pauser - hvor langt tid må man arbejde?
% sygeplersker må have en gennemsnitlig ugentlig arbejdstid på 48 timer inkl. overarbejde
%En tjeneste kan kun omlægges inden for en 24-timers periode forud for den oprindeligt planlagte tjenestes sluttidspunkt, eller efter den planlagte tjenestes starttidspunkt.
%En dagvagt tirsdag 8-16 kan således ændres til en aftenvagt mandag, nattevagt mellem mandag og tirsdag, aftenvagt tirsdag eller nattevagt mellem tirsdag og onsdag.
%Omlægning af tjenesten skal varsles mindst 1 døgn i forvejen.
%Hvis omlægningen sker med et kortere varsel end ét døgn, betales der et tillæg pr. ændret time på kr. 29,36 (31.3.2000 niveau) – tillægget beregnes pr. påbegyndte ½ time.

- Hvad patienten krav på juridisk set?
- 









%%%% ------Det gamle er her----%%%
%Overbelægning på hospitaler medfører forlængede arbejdsdage, hvilket viser sig at have en negativ indvirkning på sundhedspersonalet.\cite{Kjeldsen2015} \cite{Dinges2004} Overbelægning resulterer i, at sundhedspersonalet har kortere tid til den enkelte patient, hvorfor risikoen for fejl øges. Ifølge et amerikansk studie fra år 2002 opstår fejlene hovedsageligt, når personalet har arbejdsdage på mere end 12 timer.\cite{Dinges2004} Et andet amerikansk studie fra år 2014 indikerer en forøgelse af indlæggelsestiden ved forøget arbejdsbyrde for sundhedspersonalet\cite{Elliott2014}. Overbelægning medvirker derfor til flere sengedage for patienterne. I takt med reduceringen af sengepladser på $30~\%$ som beskrevet i \autoref{sec:overbelaegning}, er sygeplejerskernes arbejdsbyrde øget med $40~\%$fra år 2001 til 2015 ifølge en dansk undersøgelse. Derudover viser undersøgelsen, at dette resulterer i en stresset arbejdsdag og dermed en forringet kvalitet af behandlingen.\cite{Kjeldsen2015}  

%Det erfares, at sundhedspersonalet bliver stressede og ikke er i stand til at give patienterne den optimale behandling ved forøget patientbyrde. \cite{Aiken2002} Ved overbelægning er der flere patienter på stuerne, hvorfor det kan være nødvendigt at flytte nogle af patienterne ud på gangene og vaskerummene. Overbelægning kan derfor have en negativ effekt på både sundhedspersonalets og patienternes behandlingsforløb. Patientbyrden kan ligeledes være en hindring i tilfælde af brand under evakuering. 


%Ifølge et dansk studie fra år 2014, øges risikoen for mortalitet med $1,2~\%$ ved en øget belægning på $10~\%$. \cite{Madsen2014} Et  canadisk studie understøtter, at en forøgelse med $10~\%$ i belægning på akutafdelingen vil medføre en øget mortalitet og flere genindlæggelser.\cite{McCusker2014} Disse tal sammenlignes ikke direkte med danske tal, da der er tale om forskellige sundhedsvæsener, men tendensen kan dog ses som en indikation for, at overbelægning på sygehuse har en negativ effekt på mortalitetsraten.