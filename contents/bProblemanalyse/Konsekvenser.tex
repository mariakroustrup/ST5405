\subsection{Konsekvenser og omkostninger af overbelægning}
Ved overbelægning på Aalborg Universitetshospital, hvilket også gør sig gældende for ortopædkirurgisk afdeling, er der opstillet retningslinjer af Nordjyllands Beredskabsstyrelse for, hvordan overbelægningen varetages. I første omgang forsøges det at udfylde alle stuerne på afdeling, hvis dette ikke er muligt flyttes de resterende patienter til andre hospitalsafdelinger. Hvis ingen af de nævnte løsninger er mulige placeres patienterne i samtalerum, flugtvejsgange og vaskerum som en midlertidig nødløsning. Hvis der på et senere tidspunkt skulle blive plads, skal patienten straks flyttes fra samtalerummene, flugtvejsgangene og vaskerummene. \cite{Beredskab2016} Overflytningen af patienter til andre afdelinger på hospitalet kræver et overblik og kan være tidskrævende samt give komplikationer ift. behandling af patienten. 

\subsubsection{Juridisk problem ved overbelægning}
Ved opdagelse af et belægningsproblem kontaktes en brandvagt, således at brandvagten kan være tilgængelig på afdelingen så snart overbelægningen finder sted. Hvis normal belægningstilstand er mulig inden for fire timer efter overbelægningen startes er det ikke en nødvendighed at tilkalde en brandvagt. En brandvagt kan højst overvåge to afdelinger på samme etage, og om nødvendigt skal der indkaldes flere brandvagter. Det er afdelingens pligt at  afvikle overbelægningen hurtigt muligt ved at udskrive patienter eller overflytte patienter til sengestuer på andre afdelinger. \cite{Beredskab2016} For uden, at være et juridisk problem, fratages der penge fra det overordnede budget for ortopædkirurgisk afdeling hver gang en brandvagt tilkaldes.


\subsubsection{Tilrettelæggelse af personale ved overbelægning}
Under overbelægning gælder en arbejdstilrettelæggelse som sikre patientens behov, kvalitetssikring, udnyttelse af kompetencer med henblik på at finde en balance mellem ressourcerne og krav i den pågældende situation. Ved overbelægning påtager lederen eller dennes stedfortræder ansvaret for at finde en løsning, dette vil i første omgang betyde at det afgående vagthold skal blive indtil en løsning er fundet. Det undersøges om planlagte patienter kan aflyses. Derudover indkaldes der personale til at dække der manglede fremmøde i de timer, hvor det er nødvendigt samt indkalde det nye vagthold tidligere. Det er muligt at låne ressourcer fra andre afsnit/klinikker i det omfang det er muligt eller indkalde personale fra vikarbureauet. Den sidste løsning er, at forlænge arbejdstiden for flere på det afgående vagthold til situationen kan varetages af de resterende. \cite{Bjerg2016}


 
- Ved overbelægninger der en højere mortalitetsrate. 
\subsubsection{Omkostninger}
Overbelægning er omkostningsfuldt for afdelingen bl.a. i forbindelse med tilkaldelse af brandvagter osv.











%%%% ------Det gamle er her----%%%
%Overbelægning på hospitaler medfører forlængede arbejdsdage, hvilket viser sig at have en negativ indvirkning på sundhedspersonalet.\cite{Kjeldsen2015} \cite{Dinges2004} Overbelægning resulterer i, at sundhedspersonalet har kortere tid til den enkelte patient, hvorfor risikoen for fejl øges. Ifølge et amerikansk studie fra år 2002 opstår fejlene hovedsageligt, når personalet har arbejdsdage på mere end 12 timer.\cite{Dinges2004} Et andet amerikansk studie fra år 2014 indikerer en forøgelse af indlæggelsestiden ved forøget arbejdsbyrde for sundhedspersonalet\cite{Elliott2014}. Overbelægning medvirker derfor til flere sengedage for patienterne. I takt med reduceringen af sengepladser på $30~\%$ som beskrevet i \autoref{sec:overbelaegning}, er sygeplejerskernes arbejdsbyrde øget med $40~\%$fra år 2001 til 2015 ifølge en dansk undersøgelse. Derudover viser undersøgelsen, at dette resulterer i en stresset arbejdsdag og dermed en forringet kvalitet af behandlingen.\cite{Kjeldsen2015}  


%Det erfares, at sundhedspersonalet bliver stressede og ikke er i stand til at give patienterne den optimale behandling ved forøget patientbyrde. \cite{Aiken2002} Ved overbelægning er der flere patienter på stuerne, hvorfor det kan være nødvendigt at flytte nogle af patienterne ud på gangene og vaskerummene. Overbelægning kan derfor have en negativ effekt på både sundhedspersonalets og patienternes behandlingsforløb. Patientbyrden kan ligeledes være en hindring i tilfælde af brand under evakuering. 


%Ifølge et dansk studie fra år 2014, øges risikoen for mortalitet med $1,2~\%$ ved en øget belægning på $10~\%$. \cite{Madsen2014} Et  canadisk studie understøtter, at en forøgelse med $10~\%$ i belægning på akutafdelingen vil medføre en øget mortalitet og flere genindlæggelser.\cite{McCusker2014} Disse tal sammenlignes ikke direkte med danske tal, da der er tale om forskellige sundhedsvæsener, men tendensen kan dog ses som en indikation for, at overbelægning på sygehuse har en negativ effekt på mortalitetsraten.