\section{Konsekvenser og omkostninger ved overbelægning}
Ved overbelægning er det vigtigt, at situationen afvikles straks, da det bl.a. kan skabe problemer under evakuering ved brand \fxnote{måske flere grunde}. \cite{Madsen2014} For at kunne afvikle denne sker en omflytning af patienter ud fra nogle konkrete retningslinjer. Derudover tilkaldes og tilrettelæggelses arbejdet for personalet og der stræbes på at finde en balance mellem de ressourcer der til rådighed i den pågældende situation ud fra de krav der i forvejen stilles til den enkelte afdeling. Herudover er der nogle juridiske udfordringer der skal tages højde for som tilkaldelse af brandvagter til afdelingen samt overholdelse af løn og overenskomster for sundhedspersonalet. 


\subsection{Omflytning af patienter}
Ved overbelægning på Aalborg Universitetshospital, hvilket også gør sig gældende for ortopædkirurgisk afdeling, er der opstillet retningslinjer af Nordjyllands Beredskabsstyrelse for, hvordan overbelægningen varetages. I første omgang forsøges det at udfylde alle stuerne på afdelingen, hvis dette ikke er muligt flyttes de resterende patienter til andre hospitalsafdelinger. Hvis ingen af de nævnte løsninger er mulige placeres patienterne i samtalerum og flugtvejsgange som en midlertidig nødløsning. Hvis der på et senere tidspunkt skulle blive plads, skal patienten straks flyttes tilbage på stuerne. \cite{Beredskab2016} De patienter der bliver overflyttet er ofte patienter som er på grænsen til at blive udskrevet \fxnote{[1] - vil vi gerne høre nogle sygeplejersker om}.
Overflytningen og ophold i samtalerum samt flugtvejsgange beviker til, at patienterne oplever et skærpet privatliv. \cite{Madsen2014} Dette kan betyde at patienter følger et øget pres for at komme hjem \fxnote{[2] - mangler kilde.}
Ligeledes kræves det at  sundhedspersonalet har et overblik og kan være tidskrævende samt give komplikationer ift. behandling af patienten, da patienterne ikke altid befinder sig på den afdeling, hvor de er indlagt. \fxnote{[3] Mangler en kilde til at understøtte dette}. Ved overbelægning øges risikoen for mortalitet med $1,2~\%$ ved en øget belægning på $10~\%$ ifølge et dansk studie fra 2014. Hertil skal det understreges at der kan være flere faktorer der spiller ind på mortaliteten og det nødvendigvis ikke er overbelægning der er den primære grund. 


\subsubsection{Tilrettelæggelse af personale}
Under overbelægning gælder en arbejdstilrettelæggelse som er udarbejdet af Region Nordjylland og omfatter ortopædkirurgisk afdeling. Dennes formål er at sikre patientens behov, kvalitetssikring, udnyttelse af kompetencer med henblik på at finde en balance mellem ressourcerne og krav i den pågældende situation. Ved overbelægning påtager lederen eller dennes stedfortræder ansvaret for at finde en løsning, dette vil i første omgang betyde at det afgående vagthold skal blive indtil en løsning er fundet. Det undersøges om planlagte patienter kan aflyses. Derudover indkaldes der personale til at dække det manglede fremmøde i de timer, hvor det er nødvendigt samt indkalde det nye vagthold tidligere. Det er muligt at låne ressourcer fra andre afsnit i det omfang det er muligt eller indkalde personale fra vikarbureauet. Den sidste løsning er, at forlænge arbejdstiden for flere på det afgående vagthold til situationen kan varetages af de resterende. \cite{Bjerg2016}


Da sundhedspersonalet er nødsaget til at blive på afdelingen ved overbelægning medfører dette forlængede arbejdsdage, hvilket viser sig at have en negativ indvirkning på personalet.\cite{Kjeldsen2015} \cite{Dinges2004} Overbelægning resulterer i, at sundhedspersonalet har kortere tid til den enkelte patient, hvorfor risikoen for fejl øges. Ved normale omstændigheder skal personalet varetage XX antal patienter pr. tilkaldt sundhedspersonale , hvilket ved overbelægning fordeles over det tilstedeværende personale \fxnote{[4] - mangler kilde}. Hvis det er nødvendigt, at tilkalde mere personale end planlagt i en kortere periode har dette sine omkostninger ift. overenskomster \fxnote{[5]- kilde på dette}. 


Udover overbelægning har reduceringen af sengepladser på $30~\%$ yderligere medvirket til at øge sygeplejerskernes arbejdsbyrde med $40~\%$ fra år 2001 til 2015 ifølge en dansk undersøgelse. Et amerikansk studie har på baggrund af undersøgelser fra år 2002 påvist at fejlene hovedsageligt opstår, når personalet har arbejdsdage på mere end 12 timer.\cite{Dinges2004}. Et andet amerikansk studie fra år 2014 indikerer en forøgelse af indlæggelsestiden ved forøget arbejdsbyrde for sundhedspersonalet\cite{Elliott2014}. Begge undersøgelser skal ses i perspektiv med de danske overenskomster for sundhedspersonalet. Det fremgår dog af Dansk Sygeplejeråd at hver anden regionalt ansat sygeplejerske på tværs af regionerne mener, at den travle arbejdsdag går ud over patienternes sikkerhed \cite{Kjeldsen2015}. 


\subsubsection{Juridisk problemstillinger}
Ved opdagelse af et belægningsproblem kontaktes en brandvagt, således at brandvagten kan være tilgængelig på afdelingen så snart overbelægningen finder sted. Hvis normal belægningstilstand er mulig inden for fire timer efter overbelægningen sker er det ikke nødvendigt at tilkalde en brandvagt. En brandvagt kan højst overvåge to afdelinger på samme etage, hvorfor det kan være nødvendigt at der indkaldes flere brandvagter. Det er afdelingens pligt at  afvikle overbelægningen hurtigt muligt ved at udskrive patienter eller overflytte patienter til sengestuer på andre afdelinger. \cite{Beredskab2016} For uden, at være et juridisk problem, fratages der penge fra det overordnede budget for ortopædkirurgisk afdeling hver gang en brandvagt tilkaldes. \cite{KILDE - er i tvivl om det går ud over den enkelte afdeling eller om det er samlet budget for sygehuset??}.

- Omkostningsfuldt: budget for ortopædkirurgisk.
- Krav på pauser - hvor langt tid må man arbejde?
% sygeplersker må have en gennemsnitlig ugentlig arbejdstid på 48 timer inkl. overarbejde
%En tjeneste kan kun omlægges inden for en 24-timers periode forud for den oprindeligt planlagte tjenestes sluttidspunkt, eller efter den planlagte tjenestes starttidspunkt.
%En dagvagt tirsdag 8-16 kan således ændres til en aftenvagt mandag, nattevagt mellem mandag og tirsdag, aftenvagt tirsdag eller nattevagt mellem tirsdag og onsdag.
%Omlægning af tjenesten skal varsles mindst 1 døgn i forvejen.
%Hvis omlægningen sker med et kortere varsel end ét døgn, betales der et tillæg pr. ændret time på kr. 29,36 (31.3.2000 niveau) – tillægget beregnes pr. påbegyndte ½ time.

- Hvad patienten krav på juridisk set?
- 



%Kilder vi mangler: [1] Hvordan sker prioritering af patienter, hvilke patienter flyttes til gangene? [2] Patienter: øget pres for at komme hjem, skærpet privatliv, forvirring [3] Forvirring for personalet - fejl ift. at vide hvor patienterne befinder sig. %[4] Hvor lang tid har personalet til den enkelte patient normalt? %[5]Tilkaldelse af mere personalet flere skal have løn/overarbejde?




%%%% ------Det gamle er her----%%%
%Overbelægning på hospitaler medfører forlængede arbejdsdage, hvilket viser sig at have en negativ indvirkning på sundhedspersonalet.\cite{Kjeldsen2015} \cite{Dinges2004} Overbelægning resulterer i, at sundhedspersonalet har kortere tid til den enkelte patient, hvorfor risikoen for fejl øges. Ifølge et amerikansk studie fra år 2002 opstår fejlene hovedsageligt, når personalet har arbejdsdage på mere end 12 timer.\cite{Dinges2004} Et andet amerikansk studie fra år 2014 indikerer en forøgelse af indlæggelsestiden ved forøget arbejdsbyrde for sundhedspersonalet\cite{Elliott2014}. Overbelægning medvirker derfor til flere sengedage for patienterne. I takt med reduceringen af sengepladser på $30~\%$ som beskrevet i \autoref{sec:overbelaegning}, er sygeplejerskernes arbejdsbyrde øget med $40~\%$fra år 2001 til 2015 ifølge en dansk undersøgelse. Derudover viser undersøgelsen, at dette resulterer i en stresset arbejdsdag og dermed en forringet kvalitet af behandlingen.\cite{Kjeldsen2015}  

%Det erfares, at sundhedspersonalet bliver stressede og ikke er i stand til at give patienterne den optimale behandling ved forøget patientbyrde. \cite{Aiken2002} Ved overbelægning er der flere patienter på stuerne, hvorfor det kan være nødvendigt at flytte nogle af patienterne ud på gangene og vaskerummene. Overbelægning kan derfor have en negativ effekt på både sundhedspersonalets og patienternes behandlingsforløb. Patientbyrden kan ligeledes være en hindring i tilfælde af brand under evakuering. 


%Ifølge et dansk studie fra år 2014, øges risikoen for mortalitet med $1,2~\%$ ved en øget belægning på $10~\%$. \cite{Madsen2014} Et  canadisk studie understøtter, at en forøgelse med $10~\%$ i belægning på akutafdelingen vil medføre en øget mortalitet og flere genindlæggelser.\cite{McCusker2014} Disse tal sammenlignes ikke direkte med danske tal, da der er tale om forskellige sundhedsvæsener, men tendensen kan dog ses som en indikation for, at overbelægning på sygehuse har en negativ effekt på mortalitetsraten.