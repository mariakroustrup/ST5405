\subsection{Konsekvenser af overbelægning}


\subsubsection{Personle og patienter}
Overbelægning på hospitaler medfører forlængede arbejdsdage, hvilket viser at have en negativ indvirkning på sundhedspersonalet.\cite{Kjeldsen2015} \cite{Dinges2004} Overbelægning resulterer i, at sundhedspersonalet har kortere tid til den enkelte patient, hvorfor risikoen for fejl øges. Ifølge et amerikansk studie fra år 2002 opstår fejlene hovedsageligt, når personalet har arbejdsdage på mere end 12 timer.\cite{Dinges2004} Et andet amerikansk studie fra år 2014 indikerer en forøgelse af indlæggelsestiden ved forøget arbejdsbyrde for sundhedspersonalet\cite{Elliott2014}. Overbelægning medvirker derfor til flere sengedage samt genindlæggelser for patienterne. I takt med reduceringen af sengpladser på 30\% som det ses i \autoref{sec:overbelaegning}, er sygeplejerskernes arbejdsbyrde øget med 40\% fra år 2001 til år 2015 ifølge en dansk undersøgelse. Derudover viser undersøgelsen, at dette resulterer i en stresset arbejdsdag og dermed en ringere kvalitet af behandlingen.\cite{Kjeldsen2015}  

\subsubsection{Fysiske forhold}

Det erfares, at sunhedspersonalet bliver stressede og ikke er i stand til at give patienterne den optimale behandling ved forøget patienbyrde. \cite{Aiken2002} Ved overbelægning er der flere patienter på stuerne end tilladt, herudover kan det være nødvendigt at flytte nogle af patienterne ud på gangene og til vaskerummene. Dette kan påvirke patienternes privatliv, da det er ikke muligt at få samme den samme ro på gangene som i stuerne. Overbelægning har derfor en negativ effekt på både sundhedspersonalets og patienternes psyke. Den store patient byrde kan ligeledes være en hindring i tilfælde af brand, da det vil øge risikoen for, at sundhedspersonalet ikke vil være i stand til at evakuere alle patienter i tide. \cite{Madsen2014} Bjarne Hastrup, administrerende direktør for Ældre sagen udtaler følgende om, hvilken påvirkning overbelægning har for patienterne. 

   \textit{"Overbelægningen medfører uværdige, utrygge og stærkt kritisable forhold for patienterne}" - Bjarne Hastrup, administrerende direktør for Ældre Sagen. \cite{Politiken2013} 



\subsubsection{Dødelighed}
Ifølge et dansk studie fra år 2014, resulterer en øget belægning på 10\% i en øget risiko for dødelighed på hospitalet med 1,2\%. \cite{Madsen2014}  Et andet canadisk studie understøtter, at en forøgelse med 10\% i belægning på akutafdelingen vil medføre en øget dødelighed og flere genindlæggelser.\cite{McCusker2014} Disse tal sammenlignes ikke direkte med danske tal, da der er tale om forskellige sundhedsvæsener, men tendensen kan dog ses som en indikation for, at overbelægning på sygehuse har en negativ effekt på mortalitetsraten, der angiver forholdet mellem antal døde indenfor et givet tidspunkt og størrelsen af en befolkning. \cite{denstoredanskeordbog2}  