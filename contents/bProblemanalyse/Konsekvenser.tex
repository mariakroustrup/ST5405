\subsection{Personale og patient}

-sygeplejersker er nødsaget til at have flere patienter pr. time (hvilket er nederen for begge parter) 

-Forringet kvalitet i forhold til operationer og medicinering 

-Nedsat komfort for patienter (dårligt miljø)→ længere indlæggelsesperiode?

\subsection{Personale og patient}

\subsubsection{Konsekvenser af overbelægning}

Overbelægning på sygehuse samt forlængede arbejdsdage viser at have en negativ indvirkning på sundhedspersonale. Dette medfører en forhøget risiko for at lave fejl fra sundhedspersonalets side. Fejlene opstår hovedsagligt, når personalet har arbejdsdage på mere end 12 timer.\citep{forogelse2004} Ifølge i et dansk studie fra 2014, forøges risikoen for dødeligheden på hospitalet med ca. 1,2 \% ved en forøgelse i belægning på ti \%.\citep{dodelighed2014} Overbelægning af patienter resulterer i, at sunhedspersonalet har kortere tid til de enkelte patienter, hvilket ligeledes øger risikioen for fejl. Dette resulterer i genindlæggelse og flere sengedage for patienterne. Herudover skaber dette en forringet kvalitet i forhold til operationer og medicinering.

Sengepladserne på sygehusene er de seneste år blevet reduceret, da der ønskes at opnå 100\% kapacitet, hvilket i sidste ende resulterer i overbelægning og fejlbehandling af patienter. (Husk udregning af procent)




sygeplejersker er nødsaget til at have flere patienter pr. time (hvilket er nederen for begge parter)

Forringet kvalitet i forhold til operationer og medicinering 

Nedsat komfort for patienter (dårligt miljø)→ længere indlæggelsesperiode?
 