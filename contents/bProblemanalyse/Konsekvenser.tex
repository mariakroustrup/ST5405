\subsection{Personale og patient}

\subsubsection{Konsekvenser af overbelægning}


Overbelægning har herudover negativ effekt på både sundhedspersonalets og patienternes psyke. Patienterne på gangene har ikke mulighed for at få samme privatliv som i stuerne.\citep{Madsen2014}  Et amerikansk studie fra 2014 indikerer en forøgelse af indlæggelsestiden ved forøget arbejdsbyrde for sundhedspersonalet\citep{Elliott2014}. Dette viser, at overbelægning fører til forlængelse af patienternes indlæggelsestid. Det viser sig dog ikke at være muligt at effektivisere sundhedspersonale, da en dansk undersøgelse viser, at sygeplejerskernes arbejdsbyrde er forøget med 40\% fra 2001 til 2015.\citep{Kjeldsen2015}

\subsubsection{Dødelighed}
 Ifølge et dansk studie fra 2014, forøges risikoen for dødeligheden på hospitalet med ca. 1,2 \% ved en forøgelse af senge belægning på 10\%. \citep{Madsen2014} Et canadisk studie viser en forøgning af belægning på 10\% på akutafdelingen vil medføre en øget dødelighed og genindlæggelserate.\citep{McCusker2014} Disse tal er ikke helt sammenlignelige med Danmark, da der er tale om forskellige afdelinger. Tendensen kan dog ses som en indikation for at overbelægning på sygehuse har en negativ effekt på mortalitetsraten.


\subsubsection{Personle og patienter}
Overbelægning på sygehuse, hvilket medfører forlængede arbejdsdage, viser at have en negativ indvirkning på sundhedspersonalet. \citep{Kjeldsen2015} Dette medfører en øget risiko for at sundhedspersonalet laver fejl. Fejlene opstår hovedsageligt, når personalet har arbejdsdage på mere end 12 timer. \citep{Dinges2004} Overbelægning af patienter resulterer i, at sundhedspersonalet har kortere tid til den enkelte patient, hvilket ligeledes øger risikioen for fejl. Dette medvirker til genindlæggelser og flere sengedage for patienterne. Grundet mangel på sundhedspersonale ift. antal patienter, kan der eksempelvis ved brand være forøget risiko for ikke at kunne evakuere alle patienterne. \citep{Madsen2014}