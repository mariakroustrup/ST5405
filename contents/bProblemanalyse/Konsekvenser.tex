\subsection{Personale og patient}

\subsubsection{Konsekvenser af overbelægning}

Overbelægning på sygehuse samt forlængede arbejdsdage viser at have en negativ indvirkning på sundhedspersonalet. Dette medfører en forhøget risiko for at lave fejl fra sundhedspersonalets side. Fejlene opstår hovedsagligt, når personalet har arbejdsdage på mere end 12 timer.\citep{Dinges2004} Ifølge i et dansk studie fra 2014, forøges risikoen for dødeligheden på hospitalet med ca. 1,2 \% ved en forøgelse i belægning på ti \%.\citep{Madsen2014} Overbelægning af patienter resulterer i, at sunhedspersonalet har kortere tid til de enkelte patienter, hvilket ligeledes øger risikioen for fejl. Dette resulterer i genindlæggelse og flere sengedage for patienterne. Herudover skaber dette en forringet kvalitet i forhold til operationer og medicinering. Et canadisk studie viser en forøgning af belægning på 10\% på akutafdeling, hvilket medfører en forøgning på 3\% i dødelig samt genindlæggelse.\citep{McCusker2014} Disse tal er ikke helt sammenliglige med Danmark, da der er tale om forskellige afdelinger. Tendensen kan dog ses som en indikation for at overbelægning på sygehuse har en negativ effekt.
Sengepladserne på sygehusene i Danmark er fra 1996 til 2011 blevet reduceret med 30\%, da der ønskes at opnå 100\% kapacitet.\citep{Madsen2014} I sidste ende resulterer dette i overbelægning og fejlbehandling af patienter. Derudover er der ved brand forøget risiko for ikke at kunne evakuere alle patienterne pga. mangel på sundhedspersonale ift. antallet af patienter. 
Overbelægning har herudover negativ effekt på både sundhedspersonalets og patienternes psyke. Patienterne på gangene har ikke mulighed for at få samme privatliv som i stuerne.\citep{Madsen2014}  Et amerikansk studie fra 2014 indikerer en forøgelse af indlæggelsestiden ved forøget arbejdsbyrde for sundhedspersonalet\citep{Elliott2014}. Dette viser, at overbelægning fører til forlængelse af patienternes indlæggelsestid. 


sygeplejersker er nødsaget til at have flere patienter pr. time (hvilket er nederen for begge parter)
Forringet kvalitet i forhold til operationer og medicinering 
Nedsat komfort for patienter (dårligt miljø)→ længere indlæggelsesperiode?
 
    