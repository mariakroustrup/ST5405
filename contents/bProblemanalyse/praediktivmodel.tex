%	Predictive modelling
%
HVAD (matematisk model):
Det fremgår af AFSNIT\fxnote{ref til afsnit} at overbelægning forekommer i perioder på ortopædkirugisk afdeling. Denne rapport vil undersøge hvorvidt en prædiktiv model kan afhjælpe denne problematik. 

En prædiktiv model forsøger at forudsige hændelser med matematiske metoder eller databehandling. Med de matematiske modeller udarbejdes et ligningssystem, der kan give et muligt udfald af en fremtidig tilstand, baseret på modellens inputs.

HVOR (matematisk model):
% Examples include time-series regression models for predicting airline traffic volume or predicting fuel efficiency based on a linear regression model of engine speed versus load.


HVAD (databehandlingsmodel):
Databehandlingsmetoden varierer fra de matematiske modeller, idet der benyttes modeller, som kan være vanskelige at opstille ligninger for. Der bruges i stedet ofte simulering til at lave en forudsigelse. Denne metode kaldes også en "black box" prædiktiv model, da modellen ikke giver et indblik i, hvordan den kommer fra input til output.

HVOR: 
% Examples include using neural networks to predict which winery a glass of wine originated from or bagged decision trees for predicting the credit rating of a borrower.


HVORDAN (begge):
% Predictive modeling is often performed using curve and surface fitting, time series regression, or machine learning approaches. Regardless of the approach used, the process of creating a predictive model is the same across methods. The steps are:

%	Clean the data by removing outliers and treating missing data
%	Identify a parametric on nonparametric predictive modeling approach to use
%	Preprocess the data into a form suitable for the chosen modeling algorithm
%	Specify a subset of the data to be used for training the model
%	Train, or estimate, model parameters from the training data set
%	Conduct model performance or goodness-of-fit tests to check model adequacy
%	Validate predictive modeling accuracy on data not used for calibrating the model
%	Use the model for prediction if satisfied with its performance