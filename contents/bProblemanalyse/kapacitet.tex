\section{Kapacitetudnyttelse}
Kapacitetudnyttelse betegner forholdet mellem aktivitet og kapacitet. Aktivitet omhandler patient og kontakt, herunder består kontakt af forundersøgelse, behandling og kontrol. Kapacitet omfatter antallet af personale, udstyr og rum, hvor personalet består af læger, sygeplejersker og sekretærer. Udstyret beskriver antallet af maskiner på en afdeling og antallet af rum beskriver opbevarelsen af udstyret. Den samlede kapacitetsudnyttelse er defineret ud fra, at der produceres mest muligt for de investerede ressourcer.\cite{Company2013} 

\begin{figure}[H]
	\flushleft 
	\centering
	\includegraphics[scale=.5]{figures/Kapacitetsudnyttelse.png}
	\flushleft
	\caption{\textit{Den samlede kapacitetsudnyttelse, som er definineret ved forholdet mellem aktivitet og kapacitet. Aktivitet omfatter antallet af patienter samt kontakter og kapacitet omfatter personale, rum og udstyr.}\cite{Company2013}}
	\label{kapacitet}
\end{figure}

\noindent
Ud fra \figref{kapacitet} fremgår det, at kapacitetsudnyttelse er forholdet mellem aktivitet og kapacitet. Dertil ses aktivitet som antal patienter multipliceret med kontakter. Kapaciteten udgør personale, rum og udstyr lagt sammen. Antallet af patienter, der repræsenterer en del af aktivitet beskriver ligeledes belægning på hospitalets afdelinger.\cite{Company2013} 

Belægning er defineret ud fra antallet af patienter, der er normeret til på en afdeling\cite{Heidmann2014}. Når en $100~\%$ belægning opnås, svarer dette til, at de disponible sengepladser på en afdeling er taget i brug. Ved en belægning på over $100~\%$ betyder det, at der er flere patienter end afdelingen er normeret til, hvilket vil sige, at afdelingen yder mere end der er kapacitet til. Ud fra \figref{kapacitet} vil dette betyde, at der ikke er ligevægt mellem aktivitet og kapacitet, hvilket i dette tilfælde vil forårsage kapacitetsmangel på afdelingen. 


\subsubsection{Belægningsgrad på ortopædkirurgisk afdeling}\label{omfang}
På OA opleves en varierende belægningsgrad for hver måned. Som tidligere nævnt i afsnit \ref{kap} ønskes en fuld kapacitetsudnyttelse, hvoraf alle sengepladser ønskes at være i brug. Belægningsgraden er antallet af de anvendte disponible senge. På \figref{maxminbelaeg} ses belægningsgraden fra år $2014$ til $2015$ på ortopædkirurgisk afdeling.\cite{SDS2015}

\begin{figure}[H]
	\flushleft 
	\centering
	\includegraphics[scale=.45]{figures/maxminoverbelaeg.png}
	\flushleft
	\caption{\textit{Belægningsgraden på ortopædkirurgisk afdelingen på Aalborg Universitetshospital målt over $18$ måneder fra år $2014$ til $2015$. Søjlerne viser belægning ift. $100~\%$, hvortil maksimal og minimum belægning ligeledes illustreres. De blå punkter viser den gennemsnitlige belægning for hver måned.}\cite{SDS2015}}
	\label{maxminbelaeg}
\end{figure}

\noindent
Det fremgår af \figref{maxminbelaeg}, at ortopædkirurgisk afdeling oplever en belægning hhv. over og under den ønskede belægning på $100~\%$. Den maksimale belægning fremkommer i december måned år 2014 og er på $139~\%$. Maksimums belægning kan indikere, at der er flere indlagte patienter end afdelingen er disponeret til, herved har afdelingen oplevet kapacitetsmangel. Minimums belægning forekommer i januar måned år 2014 og er på $68~\%$. Minimums belægning kan indikere, at der ikke har været tilstrækkelige elektive patienter i perioder, hvilket ligeledes medfører ubalance i kapacitetsudnyttelsen. Af \figref{maxminbelaeg} er den gennemsnitlige belægning pr. måned hyppigst under $100~\%$. I oktober og december måned år $2014$ opleves dog en gennemsnitlig belægning over $100~\%$. Den gennemsnitlige belægning ses varierende mellem $90$ og $100~\%$ for de resterende måneder, hvilket kan indikere, at afdelingen oplever kapacitetsmangel i kortvarige perioder.\cite{SDS2015} 
Det fremgår ikke af den anvendte data, hvorvidt belægningen opleves i timer eller flere døgn. Dertil skal der tages forbehold for, at det ikke er angivet om det er elektive eller akutte patienter, der udgør en belægning over $100~\%$.\cite{SDS2015} 
 
For at underbygge belægningsgraden yderligere, illustrerer \figref{antaldage} antal dage pr. måned med en belægningsgrad på over $100~\%$. Denne graf er udarbejdet ud fra ortopædkirurgisk afdeling over de samme 18 måneder som \figref{maxminbelaeg}. \cite{SDS2015} 

\begin{figure}[H]
	\flushleft 
	\centering
	\includegraphics[scale=.4]{figures/antaldage.png}
	\flushleft
	\caption{\textit{Belægningsgrad over $100~\%$ målt i antal dage over $18$ måneder fra år $2014$ til juni $2015$ for ortopædkirugisk afdeling på Aalborg Universitetshospital.}\cite{SDS2015}}
	\label{antaldage}
\end{figure}

\noindent
Det fremgår af \figref{antaldage}, at der i oktober måned år $2014$ opleves en belægning på over $100~\%$ i $19$ dage, sammenlignes dette med oktober måned på \figref{maxminbelaeg} ses en belægning på $130~\%$. Der ses ligeledes en sammenhæng mellem de resterende måneder for de to grafer. 
Ud fra den anvendte data fremgår det ikke, hvor mange patienter, der udgør en belægningsgrad over $100~\%$, samt hvor længe de enkelte patienter er indlagt på afdelingen. Da belægningsgraden og antal dage kan variere for hver måned, anses $18$ måneder ikke som værende repræsentativ for at kunne vurdere problemets omfang. Ud fra belægningsgraden kan det dog tyde på, at en effektivisering af planlægningen af patienter på ortopædkirurgisk afdelingen vil kunne medføre en balance i kapacitetsudnyttelsen.












GEM!
Det kan derfor være nødvendigt, at personalet skal varetage flere patienter samt arbejdsopgaver. Herudover kan det være nødvendigt at tilkalde ekstra personale for at opnå en balance i kapacitetsudnyttelsen.
Hvis der derimod er en belægning på under $100~\%$ er der omvendt færre patienter end afdelingen er normeret til. Dette betyder, at der er flere sengepladser end patienter, hvilket ligeledes fører til en ubalance i kapacitetsudnyttelsen. I denne situation er der mere personale end nødvendigt, hvilket betyder, at der ikke er fuld udnyttelse af personalets arbejdskraft.\cite{Pauly1986} 
