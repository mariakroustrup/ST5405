\subsection{Ortopædkirurgisk Generelt}
Region Nordjylland er bygget op af flere matrikler. Ortopædkirurgi er ledelsesmæssigt organiseret af Aalborg Universitetshospital og består af følgende matrikler: Aalborg (Syd og Nord), Dronninglund, Farsø og Hobro. Hospitalernes primære aktivitet består i ambulante besøg \fxnote{udredning, behandling eller kontrol}, hvor der ikke kræves indlæggelse samt sengedage \fxnote{antallet af dage, hvor en seng er belagt} og udskrivelser \fxnote{antallet af registrerede udskrivelser} \cite{RegionNord2016}. 

Ortopædkirurgisk afdeling på Aalborg Universitetshospital behandler, hvis der er sket en skade  på knogler, muskler, sener eller led. Afdelingen har 10 fagområder: Børneortopædkirurgi, knogle- og rekonstruktion, fod- og ankelkirurgi, knæ- og hoftekirurgi, håndkirurgi, ryg- og bækkenkirurgi, knæ- og idrætsskader, tumor- og sarkomkirurgi, amputationer og sår, samt traumatologi.
Afdelingen er delt op i 5 afsnit bestående af sengeafsnit O1 og O2 på 5. sal, operationsafsnit på 1. sal, sammedagskirurgi afsnit O6, samt ambulatorium.  \cite{Aalborg2016}

\begin{itemize}
\item På sengeafsnittet O1 behandles brud på lårbenshalsen og sportsskader i knæet. Derudover bliver patienter med skader i hånden, forbrændinger eller ætsninger også indlagt her.
\item På sengeafsnittet O2 behandles ryglidelser, fod- og ankelskader, bækkenbrud og patienter med mange brud efter færdselsuheld. Derudover ligger der børn til operation på dette afsnit eventuelt sammen med forældrene. 
\item På den centrale operationsgang opereres der i gennemsnit 15 ortopædkirurgiske patienter i døgnet. 
\item På sammedagskirurgi afsnit O6 foretages mindre operative indgreb, som kræver en kortere operationstid. De fleste indgreb foretages i fuld narkose og patienten udskrives samme dag. Dette er ofte kikkertkirurgi på knæ og operationer i anklen, foden eller hånden. 
\item På ambulatoriet sker kontrol af patienter efter operationer eller efter skadestuen. Dette sker hvis der er behov for det. Derudover undersøges henviste patienter af lægen, hvor det vurderes om hvorvidt patienten skal have en operation. \cite{Aalborg2016}
\end{itemize}

\subsection{Akutte og elektive patienter}
Andelen af akutte indlæggelser i Region Nordjylland udgør $76~\%$, hvor de resterende $25~\%$ er elektive. Hvis der sammenlignes med andre afdelinger på Aalborg Universitetshospital er ortopædkirurgisk afdeling den afdeling der har flest elektive indlæggelser, hvilket svarer til $13~\%$ af de samlede indlæggelser \fxnote{Ortopæd: 3.609 Total: 27.887 = 12,9 \%}. Der er dog en stigning i ambulante besøg på Aalborg Universitetshospital fra år 2010-2013, hvilket har medført at sengedage på hospitalet har været faldende på grund af omlægningen fra stationær aktivitet til ambulant. Dette betyder et fald i sengedage på $6~\%$ for akutte og $19~\%$ for elektive patienter. Der er en stigning i udvikling af aktiviteten målt på antal ydelser for ortopædkirurgi fra år 2010 til 2013, mens det samlede andel for alle specialer er faldende over de 3 år. Antal ydelser for ortopædkirurgi afdeling udgør $4~\%$ af det samlede \fxnote{Ortopæd: 33.578 Total:862.296 =3.89}. Fordelingen af nordjyderne forbrug til ortopædkirurgisk afdeling er $5,6\%$ af det samlede forbrug til de forskellige specialer. Dette svarer til et bruttohonorar på $13.550.874,38$~kr. \cite{RegionNord2016}

\subsubsection{Stigning i akuttepatienter}
På nuværende tidspunkt er der et fald i fordelingen af 0 til 18 år og fra 19 til 65 år, mens det for 65+ er en stigning i aldersfordelingen i Region Nordjylland. Det forventes at andelen af befolkningen i aldersgruppen over 65 år vil stige fra $29~\%$ i år 2014 til $34~\%	$ i år 2025. En akutbehandling er ofte en indgang til det samlede sygehus, hvilket vil sige at der sker en stigning i sundhedsrelaterede ydelser, da det forventes at denne stiger i takt med at andelen af ældre borger stiger. Dette problem skal løses uden at der sker en forringet kvalitet af sundhedssektoren. 