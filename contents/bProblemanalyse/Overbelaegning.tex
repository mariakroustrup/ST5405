<<<<<<< Updated upstream
\subsection{Personale og patient}

\subsubsection{Konsekvenser af overbelægning}

Overbelægning på sygehuse samt forlængede arbejdsdage viser at have en negativ indvirkning på sundhedspersonale. Dette medfører en forhøget risiko for at lave fejl fra sundhedspersonalets side. Fejlene opstår hovedsagligt når personalet har arbejdsdage på mere end 12 timer. Ifølge i et dansk studie fra 2014, forøges risikoen for dødeligheden på hospitalet med ca. 1,2 \% ved en forøgelse i belægning på ti \%.
=======
\subsection{Overbelægning}
\subsubsection{Hvad er overbelægning?}
Definitionen af overbelægning er en overstigelse af indlagte patienter på en given afdeling ift. tilgængelige sengepladser. Overbelægning er estimeret til 85\% af det samlede antal sengepladser, hvorefter sandsynligheden for hospitalets mortalitet \fxnote{hyppigheden af dødsfald i en befolkning angives ved forholdet mellem antallet af døde inden for et givet tidsrum og størrelsen af befolkningen.} stiger sammenlignet med ved underbelægning. [1]


>>>>>>> Stashed changes




<<<<<<< Updated upstream



sygeplejersker er nødsaget til at have flere patienter pr. time (hvilket er nederen for begge parter)

Forringet kvalitet i forhold til operationer og medicinering 

Nedsat komfort for patienter (dårligt miljø)→ længere indlæggelsesperiode?
=======
%[1]   High Levels Of Bed Occupancy Associated With Increased Inpatient And Thirty-Day Hospital Mortality In Denmark
>>>>>>> Stashed changes
