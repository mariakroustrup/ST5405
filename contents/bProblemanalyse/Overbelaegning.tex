\subsection{Personale og patient}

\subsubsection{Konsekvenser af overbelægning}

Overbelægning på sygehuse samt forlængede arbejdsdage viser at have en negativ indvirkning på sundhedspersonale. Dette medfører en forhøget risiko for at lave fejl fra sundhedspersonalets side. Fejlene opstår hovedsagligt når personalet har arbejdsdage på mere end 12 timer. Ifølge i et dansk studie fra 2014, forøges risikoen for dødeligheden på hospitalet med ca. 1,2 \% ved en forøgelse i belægning på ti \%.







sygeplejersker er nødsaget til at have flere patienter pr. time (hvilket er nederen for begge parter)

Forringet kvalitet i forhold til operationer og medicinering 

Nedsat komfort for patienter (dårligt miljø)→ længere indlæggelsesperiode?
