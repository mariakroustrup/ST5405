\subsection{Kapacitetsudnyttelse}

Et belægningsproblem en hospitalsafdeling kan som nævnt i afsnit \ref{sec:overbelaegning} medføre en række komplikationer under ved overbelægning. Dette afsnit har til formål at analysere konsekvenserne ved ikke at udnytte afdelingens fulde kapacitet.

Ved en kapacitet på $100~\%$ er, som nævnt i afsnit \ref{sec:overbelaegning}, alle normerede sengepladser til rådighed anvendte. Ved brug af ikke normerede senge finder overbelægning sted. Ved en belægningsgrad på $110~\%$ yder personalet mere arbejde end planlagt, men omkostningerne af hospitalspersonale pr. patient er umiddelbart lavere en ved $100~\%$ belægning. Der tages dog forbehold for medicin, overarbejds- og brandvagtpersonale omkostninger.
Hvis alle de normerede senge ikke udnyttes og der eksempelvis er en belægningsgrad på $90~\%$, er alt personalet der er til rådighed ikke udnyttet efter almindelig arbejdsbyrde. Dette betyder at prisen pr. patient er højere end ved belægning af alle de normerede senge. \cite{Pauly1986} Dermed er omkostningerne for én ved lav kapacitetsudnyttelse omkostningsfulde for sundhedsvæsnet, da ortopædkirurgisk afdeling ikke behandler elektive patienter der kunne havde været behandlede. Det er defor vigtigt at finde en buffer i kapaciteten der er udregnet efter både elektive og akutte patienter, med henblik på at have en kapacitets udnyttelse på $100~\%$. 