\subsection{Konsekvenser af overbelægning}


\subsubsection{Ubalance mellem ressourcer og krav}
- Personalet skal varetage andre opgaver da de bliver underbemandet. Hvilket vil sige at der bliver flere patienter pr. sygeplejersker. 
- Ved overbelægninger der en højere mortalitetsrate. 




\subsubsection{Brandsikkerhed ved overbelægning}
Ved overbelægning på Aalborg Universitetshospital er der opstillet retningslinjer af Nordjyllands Beredskabstyrrelse for hvordan overbelægningen varetages. \citep{Beredskab2016} Ved overbelægning på den enkelte afdeling forsøges det at udfylde alle sengestuerne på afdelingen. Når det ikke længere er muligt at have flere patienter på sengestuerne flyttes de resterende patienter til andre hospitalsafdelinger. Patienter kan også flyttes til samtalerum samt vaskerum og gange. \citep{Beredskab2016} Dog flyttes patienter ikke på eller i vaskerum medmindre at alle andre muligheder er udtømte. 
Ved opdagelse af et belægningsproblem kontaktes en brandvagt, således at brandvagten kan være tilgængelig på afdelingen så snart overbelægningen finder sted. Hvis normal belægningstilstand er mulig inden for fire timer efter overbelægningen startes er det ikke en nødvendighed at tilkalde en brandvagt. En brandvagt kan højst overvåge to afdelinger på samme etage, og om nødvendigt skal der indkaldes flere brandvagter. \citep{Beredskab2016} Ved overbelægning skal denne afvikles hurtigt muligt ved at udskrive patienter eller overflytte patienter til sengestuer på andre afdelinger. 

\subsubsection{Juridisk}

Ved opdagelse af et belægningsproblem kontaktes en brandvagt, således at brandvagten kan være tilgængelig på afdelingen så snart overbelægningen finder sted. Hvis normal belægningstilstand er mulig inden for fire timer efter overbelægningen sker er det ikke nødvendigt at tilkalde en brandvagt. En brandvagt kan højst overvåge to afdelinger på samme etage, hvorfor det kan være nødvendigt at der indkaldes flere brandvagter. Det er afdelingens pligt at  afvikle overbelægningen hurtigt muligt ved at udskrive patienter eller overflytte patienter til sengestuer på andre afdelinger. \cite{Beredskab2016} For uden, at være et juridisk problem, fratages der penge fra det overordnede budget for ortopædkirurgisk afdeling hver gang en brandvagt tilkaldes. \cite{KILDE - er i tvivl om det går ud over den enkelte afdeling eller om det er samlet budget for sygehuset??} I budgetfordelingen for Aalborg Universitetshospital i 2017 indgår det desuden at ventetiden på en operation for elektive patienter skal reduceres fra 57 dage til 50 dage. \cite{Budget2016} Dette betyder optimering i planlægning ift. kapacitet. Under overbelægningstilstand kan sundhedspersonalets arbejdsdag forlænges. \cite{Kjeldsen2015} En normal arbejdsuge for en sygeplejerske er 37 timer i Danmark. \cite{Danske2015} I tilfælde af overarbejde må en arbejdsuge for en sygeplejerske ifølge, \textit {Aftale om visse aspekter i forbindelse med tilrettelæggelse af arbejds- tiden, § 6, stk. 2}, ikke overstige 48 timer. Omplanligning af en sygeplejerskes normalarbejde skal derudover finde sted 24 timer før fremmøde. Dette er ikke inkluderende overarbejde. Pauser for sygeplejersker indgår i den planlagte arbejdstid, og tildeles ved arbejdsdage på over seks timer. 

- Omkostningsfuldt: budget for ortopædkirurgisk.
- Krav på pauser - hvor langt tid må man arbejde?
% sygeplersker må have en gennemsnitlig ugentlig arbejdstid på 48 timer inkl. overarbejde
%En tjeneste kan kun omlægges inden for en 24-timers periode forud for den oprindeligt planlagte tjenestes sluttidspunkt, eller efter den planlagte tjenestes starttidspunkt.
%En dagvagt tirsdag 8-16 kan således ændres til en aftenvagt mandag, nattevagt mellem mandag og tirsdag, aftenvagt tirsdag eller nattevagt mellem tirsdag og onsdag.
%Omlægning af tjenesten skal varsles mindst 1 døgn i forvejen.
%Hvis omlægningen sker med et kortere varsel end ét døgn, betales der et tillæg pr. ændret time på kr. 29,36 (31.3.2000 niveau) – tillægget beregnes pr. påbegyndte ½ time.

- Hvad patienten krav på juridisk set?







