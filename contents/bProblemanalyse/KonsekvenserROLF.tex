\subsection{Konsekvenser af overbelægning}


\subsubsection{Ubalance mellem ressourcer og krav}
- Personalet skal varetage andre opgaver da de bliver underbemandet. Hvilket vil sige at der bliver flere patienter pr. sygeplejersker. 
- Ved overbelægninger der en højere mortalitetsrate. 




\subsubsection{Brandsikkerhed ved overbelægning}
Ved overbelægning på Aalborg Universitetshospital er der opstillet retningslinjer af Nordjyllands Beredskabstyrrelse for hvordan overbelægningen varetages. \citep{Beredskab2016} Ved overbelægning på den enkelte afdeling forsøges det at udfylde alle sengestuerne på afdelingen. Når det ikke længere er muligt at have flere patienter på sengestuerne flyttes de resterende patienter til andre hospitalsafdelinger. Patienter kan også flyttes til samtalerum samt vaskerum og gange. \citep{Beredskab2016} Dog flyttes patienter ikke på eller i vaskerum medmindre at alle andre muligheder er udtømte. 
Ved opdagelse af et belægningsproblem kontaktes en brandvagt, således at brandvagten kan være tilgængelig på afdelingen så snart overbelægningen finder sted. Hvis normal belægningstilstand er mulig inden for fire timer efter overbelægningen startes er det ikke en nødvendighed at tilkalde en brandvagt. En brandvagt kan højst overvåge to afdelinger på samme etage, og om nødvendigt skal der indkaldes flere brandvagter. \citep{Beredskab2016} Ved overbelægning skal denne afvikles hurtigt muligt ved at udskrive patienter eller overflytte patienter til sengestuer på andre afdelinger. 

\subsubsection{Omkostninger}











%%%% ------Det gamle er her----%%%
%Overbelægning på hospitaler medfører forlængede arbejdsdage, hvilket viser sig at have en negativ indvirkning på sundhedspersonalet.\cite{Kjeldsen2015} \cite{Dinges2004} Overbelægning resulterer i, at sundhedspersonalet har kortere tid til den enkelte patient, hvorfor risikoen for fejl øges. Ifølge et amerikansk studie fra år 2002 opstår fejlene hovedsageligt, når personalet har arbejdsdage på mere end 12 timer.\cite{Dinges2004} Et andet amerikansk studie fra år 2014 indikerer en forøgelse af indlæggelsestiden ved forøget arbejdsbyrde for sundhedspersonalet\cite{Elliott2014}. Overbelægning medvirker derfor til flere sengedage for patienterne. I takt med reduceringen af sengepladser på $30~\%$ som beskrevet i \autoref{sec:overbelaegning}, er sygeplejerskernes arbejdsbyrde øget med $40~\%$fra år 2001 til 2015 ifølge en dansk undersøgelse. Derudover viser undersøgelsen, at dette resulterer i en stresset arbejdsdag og dermed en forringet kvalitet af behandlingen.\cite{Kjeldsen2015}  


%Det erfares, at sundhedspersonalet bliver stressede og ikke er i stand til at give patienterne den optimale behandling ved forøget patientbyrde. \cite{Aiken2002} Ved overbelægning er der flere patienter på stuerne, hvorfor det kan være nødvendigt at flytte nogle af patienterne ud på gangene og vaskerummene. Overbelægning kan derfor have en negativ effekt på både sundhedspersonalets og patienternes behandlingsforløb. Patientbyrden kan ligeledes være en hindring i tilfælde af brand under evakuering. 


%Ifølge et dansk studie fra år 2014, øges risikoen for mortalitet med $1,2~\%$ ved en øget belægning på $10~\%$. \cite{Madsen2014} Et  canadisk studie understøtter, at en forøgelse med $10~\%$ i belægning på akutafdelingen vil medføre en øget mortalitet og flere genindlæggelser.\cite{McCusker2014} Disse tal sammenlignes ikke direkte med danske tal, da der er tale om forskellige sundhedsvæsener, men tendensen kan dog ses som en indikation for, at overbelægning på sygehuse har en negativ effekt på mortalitetsraten.