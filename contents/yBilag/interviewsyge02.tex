\subsection{Sygeplejerske på sengeafsnit 02} \label{bilagO2}
\textbf{Til at starte med vil vi så spørge, hvem er du?}\\
\noindent
\textit{Jeg hedder Lars og er almindelig sygeplejerske på gulvet.}\\
\noindent
\textbf{Hvor længe har du arbejdet på denne her afdeling?} \\
\noindent
\textit{$28$ år.}\\
\noindent
\textbf{Må vi citere dig for det her interview?}\\
\noindent
\textit{Ja det må I godt.}\\
\noindent
\textbf{Til at starte med at høre, hvor mange arbejdstimer har du gennemsnitligt om ugen?}\\
\noindent
\textit{Almindelig fuldtidsstilling, $37$ timer.}\\
\noindent
\textbf{Hvor lang tid er det fordelt over ift. om det er gennemsnitligt for en måned eller om det er gennemsnit hver uge?}\\
\noindent
\textit{Vi har en arbejdsplan, der hedder otte uger i alt, hvor jeg så veksler mellem dagvagter og aftenvagter.}\\
\noindent
\textbf{Har i så vagtskifte, når du siger dag og nat. Er der nogle bestemte tidspunkter de her vagtskifte finder sted?}\\
\noindent
\textit{Ja, på alle mine hverdage, dagvagter, arbejder jeg fra $7$ til $15$:$45$, og så møder jeg aftenvagten ind lidt tidligere. De møder ind $15$:$30$, så de har et kvarter til at sætte sig ind i de patienter de skal, og imens de gør det, skal jeg passe klokker.}\\
\noindent
\textbf{Så I får ligesom aflastet og sagt, at det er de her patienter I skal varetage og får fordelt viden.}\\
\noindent
\textit{Ja, men det er meningen, at de skal sætte sig og læse. Og jeg skal kun sige noget, hvis der er noget jeg lige skal af med, ellers skal de læse sig til det. Plus jeg skal tage klokkerne ude på gangen.}\\
\noindent
\textbf{Hvor mange patienter varetager du på en normal vagt? Har I et bestemt antal patienter, der er tildelt?}\\
\noindent
\textit{Nej, det er forskelligt, fordi de veksler meget i sværhedsgraden eller hvordan vi passer dem og selvfølgelig, hvad de fejler - to til fire til seks patienter.}\\
\textbf{Så det er to til seks på området.}\\
\noindent
\textit{Ja.}\\
\noindent
\textbf{Så kommer der noget ift. pauser. Nu har vi læst, at I har nogle pauser, hvis I har de her arbdejdsdage, om de er betalte pause, altså om I kan blive indkaldt I jeres pauser? eller hvordan det foregår?}\\
\noindent
\textit{Det ved jeg faktisk ikke, altså jeg bliver altid indkaldt, nu går vi med de her telefoner.  Så hvis lægen vil have fat i mig, så ringer han, fordi vi har skrevet os på tavler dernede med, at jeg har de patienter. Så har jeg mit navn og denne her telefon, så kan de ringe efter mig, hvis det er, eller fysioterapeuten, eller min kollega, hvis der er nogen. Så der er ikke sådan faste pauser. Man kan altid finde en. I hvert fald ikke mere, det kan godt være, at det har været sådan. Jeg tror også godt man må sige, man går til pause, men kutymen er, at man rejser sig, hvis der er brug for det. Plus det er også vores kloksystem, og det vil sige, jeg kan se, hvis jeg sidder ude og får kaffe, så ringer telefonen eller også, hvis jeg ikke har min telefon, så kan jeg se på en display, om det er min patient, der ringer både på stuen, men også ude på toilettet. Kutymen er, at hvis vi er et par stykker om de samme patienter. Èn af os tager klokken. Det er ikke meningen, at de andre skal tage den.}\\
\noindent
\textbf{Så patienterne er fordelt og tildelt mellem jer?}
\noindent
\textit{Ja.}
\noindent
\textbf{Så er det lidt ift., når I modtager de her patienter. Nu har vi snakket lidt med sekretariatet ift., hvordan I modatager de her patienter. Er det et bestemt tidsrum, du ser, at I modtager de her patienter? Nu snakker vi ift. de elektive patienter dem, der er planlagt om der er et bestemt tidsrum?}\\
\noindent
\textit{Det er der sådan set ikke helt, fordi vi får også afhængig af, hvad dage det er, patienter om eftermiddagen, der skal gøres klar til næste dag. Men altså indkaldte patienter møder om morgenen typisk kl $7$. Det er der også en sat af til at tage imod, så de bliver taget fra ude foran ved sekretæren, også kommer de ind på deres plads og så stiller man de spørgsmål, der er relevante og så sørger man for, at de er klædt om og parat til operationen, hvis de skal ned allerede ved 8-tiden eller lidt senere.}\\
\noindent
\textbf{Så i modtager dem om morgen og gør dem klar til i løbet af dagen?}\\
\noindent
\textit{Ja, det gør vi ved de fleste patienter.}\\
\noindent
\textbf{Så er det ift. de akutte patienter, det er døgnet rundt I modtager dem?}\\
\noindent
\textit{Ja, her på afdelingen.}\\
\noindent
\textbf{Ift. udskrivelse af patienter, foregår det et bestemt tidsrum på dagen, hvor man siger vi udskriver patienter fra kl. $10$ til kl. $14$, ellers udskriver vi ikke eller hvordan foregår det?}\\
\noindent
\textit{Det er også meget forskelligt, fordi dels, så kommer de fra kæmpe optagområder og nogen kommer endnu længere væk fra, så tit er det transporten, der nogengange bestemmer, hvornår vedkommende bliver hentet og alt afhængig af, hvem der henter dem. Hvis det er Falck og de skal hjem eller hvis de skal til et andet sygehus, så har de et tidsrum på to timer eller fire timer. Hvis du bestiller en taxa, så kan du næsten få et klokketidspunkt indenfor fem minutter. Og det synes jeg faktisk, at de er gode til at overholde.}\\ 
\noindent
\textbf{Så I udskriver løbende som der er behov for det?}\\
\noindent
\textit{Ja.}\\
\noindent
\textbf{Så ift., når I vurderer om patienter kan udskrives, hvor ofte gøres det? Er det ift., når der er stuegang eller er det flere gange i løbet af dagen eller hvordan vurderes det?}\\
\noindent
\textit{Nej, altså vi har jo et. Det er jo meget forskelligt for vores patienter, fordi vi har små børn og vi har unge mennesker og ældre mennesker og så har vi de rigtig gamle. De unge mennesker eller børn kan godt have en operation, hvor man skal have fat i hjemmesygeplejersken til at source dem, og det skal være planlagt i god tid. Også de andre, der aftaler man typisk over computeren om, hvordan man udskriver dem, og hvordan man sender dem hjem.}\\
\noindent
\textbf{Hvor ofte vurderer man så, hvornår de skal hjem og hvornår de er klar. Er det en dialog med patienten eller?}\\
\noindent
\textit{Ja, men typisk skal hjemmeplejeren også være blandet ind i det, og de skal komme til dem, så vil de helst have de ikke kommer for sent, fordi der er jo også forskel på deres dagvagt, hvor de er flest på arbejde. Og så kommer man over i en aftenvagt, hvor de ikke er ret mange på arbejde. Så det gælder om at få dem hjem i dagstiden. Det er meget vigtigt.}\\
\noindent
\textbf{Okay. Så ift. om der er nogle tidspunkter, hvor I ikke længere udskriver patienter og indlægger patienter f.eks. om natten?}\\
\noindent
\textit{Ikke indlægger, fordi vi er en akut afdeling. Der tager vi alt, hvad der kommer. Og udskriver, vi kan godt have en trafikulykke, hvor vi tror, at der er en hel masse, men hvor det hele bliver afviklet eller det hele bliver godkendt, at patienten ikke fejler noget. Og det kan typisk være en knallertulykke eller en bilulykke, hvor man kommer ind og kommer til observation. Det kan være om eftermiddagen eller det kan også være først på aftenen, hvor lægen kommer op og vurderer patienten og siger du fejler ikke noget og du må have lov til at komme hjem, men det er deres afgørelse.}\\
\noindent
\textbf{Nu siger du lægen? Er det læger, der vurderer patienterne, der kommer ind akut eller er det en kirurg?}\\
\noindent
\textit{Altså jeg har kun kirurger på min afdeling, og jeg må ikke som sygeplejerske sende nogen hjem på eget initiativ.}\\
\noindent
\textbf{Så de skal have været forbi?}\\
\noindent
\textit{En læge skal have godkendt og jeg skal i kontakt med en læge på en eller anden måde for at spørge om vedkommende må komme hjem, og det kan også være et barn med en brækket arm, der er opereret i løbet af formiddagen eller eftermiddagen, som hvis de ellers har det godt med smerter og ingen kvalme, så må man gerne have lov til at sende dem hjem om aftenen, hvis lægen siger god for det og det kan godt være pr. telefonen.}\\
\noindent
\textbf{Nu kommer vi ind på noget ift. sengepladser. Har I nogle sengepladser som I altid forbeholder til akutte patienter ift., når I fordeler sengene?}\\
\noindent
\textit{ Nej, det kan vi ikke mere, fordi vi har rimelig mange patienter til daglig, og vi er faktisk også begyndt og lade være med tænke på, at der ligger to eller tre mænd på en stue eller kvinder. Nu bruger vi simpelthen bare de pladser vi har, så vi ikke skal flytte rundt. En kvinde kan godt overnatte sammen med to mænd fordi vi har forhæng for.}\\
\noindent
\textbf{Så der er noget diskretion?}\\
\noindent
\textit{Ja. Og så er der nogle, der bliver flyttet om næste dag.}\\
\noindent
\textbf{Nu kommer vi nemlig ind på det her med, hvis der ikke er sengepladser nok på de her værelser, og du siger, at I godt kan finde på at placere dem bare inde på stuerne. Er I nødt til at have patienter liggende på gangen nogen gange? Nu det ikke natten over, nu det bare i løbet af dagen?}\\
\noindent
\textit{Ja, det har vi gjort tidligere, men brandvæsnet er efter os. Vi må ikke. For nogle år tilbage der var vi fuldstændig fyldt op. Også alle vores patienter har jo på en eller anden måde næsten brug for et hjælpemiddel af en slags. Det kan være krykker, de fylder ingenting, men gangstativ, kørestole, bækkenstole, hvad ved jeg. Alt fylder og det hele det kom herop til os, så det står heroppe til den enkelte patient med navn på. Det stod engang ude på gangen, det må det ikke for brandtilsynet. Så kom vi det ind på stuerne, så kom vi de ekstra patienter, du snakker om, de akutte, de blev stående ude på gangen. De må det ikke. Så kørte vi dem ind, så siger brandtilsynet, det er patienternes sikkerhed, så der skal vi køre patienterne ind på stuerne, hvis der er brand f.eks. og hjælpemidlerne må vi simpelthen bare placere i et andet rum. Det er ligegyldigt hvad, der må ikke være noget på gangene. Og det gjorde så, at vi fra, for et par år siden, hvor vi normalt har fire sengestue, sagde så vil vi kun have tre sengestuer for at have tre patienter på stuen, men så have et hjørne til alle hjælpemidlerne. Det kan vi sådan nogenlunde holde.}\\
\noindent
\textbf{Så I har valgt at sige nu tager vi det her udstyr og flytter det ind på stuen?}\\
\noindent
\textit{Ja, og hvis nummer fire patienter kommer, så får vedkommende pladsen og så skal vi finde et andet sted til udstyret.}\\
\noindent
\textbf{Jamen så er det lidt ift., hvis I så får for mange patienter, altså akutte patienter, og I ikke kan placere dem på de her stuer, da de er fyldt op. Hvad gør I så? Nu har vi læst, at der er trådt nogle regler i kraft med, at de bliver flyttet ned til andre stuer og andre afdelinger og om du kan bekræfte det?}\\
\noindent
\textit{Ja, det bliver så bare styret et andet sted fra. Det bliver styret fra AMA, Akut medicinsk modtagerafsnit. Der sidder nogle koordinatorer, som skal sørge for, at deres patienter der ovre fra eller fra modtagelsen, det medicinske regi, at der er plads til dem dvs., hvis de medicinske afsnit ikke har pladser nok, så kan vi risikere, at skal have en medicinsk patient eller to og det sørger den koordinator for.} \\
\noindent
\textbf{Så det er ikke jer, der styrer de patienter, og hvordan de flyttes?}\\
\noindent
\textit{Nej, vi skal melde tilbage. Jeg ved ikke om vi skal i dagligdagen, men i weekenderne, der skal vi melde tilbage. Jeg tror det er inden kl. 11 ellers så er det omkring kl. $9$ til koordinatoren og sige, at vi har to ledige pladser ellers forventer vi to ledige pladser lidt senere over middagen eller sådan noget, også har de råderum over dem i princippet.}\\
\noindent
\textbf{Ja, indtil I får en akut patient, der overtager.}\\
\noindent
\textit{Ja, så finder vi så ud af, hvad vi skal gøre. Det er lige trådt i kraft.}\\
\noindent
\textbf{Okay. Så, hvis I har nogle af de her ekstra patienter ift. hverdage, så bliver I sat på arbejde til at udfylde stuen eller udfylde afdelingen ift., hvor meget personale, der møder op eller hvordan fungerer det?}\\
\noindent
\textit{Nej, altså der jo planlagt, hvem der møder ind på den der otte ugers plan, så de skal bare møde ind, og så skal vi fordele patienterne, når vi kommer om morgenen. Og det har vi så gjort ved, at vi har typisk fire fem specialer, og de specialer har vi valgt os ind på af egen interesse, så jeg har, i en lang periode, så skal jeg passe rygpatienter primært. Men det kan godt være, at jeg skal passe børn, og det kan også være jeg skal passe noget andet eller noget tredje eller medicinske patienter. Det er forskelligt fra dag til dag. Men vi prøver at koncentrere os om dem vi har valgt os ind på.}\\
\noindent
\textbf{Ja, fordi det lidt ift., hvis der kommer for mange patienter og I ikke kan varetage dem, så har vi læst, at I har et vikarbureau I kontakter og får noget ekstra personale ind, og det kan du bekræfte, at det er sådan det fungerer?} \\
\noindent
\textit{Ja sådan set, hvis vi er i nød, så må vi ringe.}\\
\noindent
\textbf{Ja også ift. overarbejde om det er noget I bliver meget påvirket af altså, at I bliver nødt til at blive ekstra, når I for eksempel har de her vagtskifte, at I bliver nødt til at blive ekstra for at hjælpe det nu kommende vagthold?}\\
\noindent
\textit{Nej, det synes jeg ikke er noget problem. Det er meget sjældent.}\\
\noindent
\textbf{Okay, så det er sjældent det sker.}\\
\noindent
\textit{Ja, fordi så er det mig det hænger på eller de to andre, fordi vi er der jo til at lappe over. De resten er jo taget hjem, så der er kun tre at trække på, hvis der er nogle som ikke møder op eller vi har overset noget, eller der er et eller andet akut.}\\
\noindent
\textbf{Nu har vi har arbejdet med det her med, at I nogle gange har for mange patienter eller for lidt personale på arbejdet. Hvad er det, der først begrænser jer, er det, at I mangler plads på afdelingen eller er det, at I ikke har nok personale eller hvornår er det de her stress situationer opstår?}\\
\noindent
\textit{Altså det er sjældent vi har for mange patienter. Vi kan godt have mange patienter, men det behøver altså ikke være for mange, men det er nærmere, at vi ikke er nok personale til dem vi har af patienter, fordi mange af vores patienter kræver meget. Altså og kræver meget der mener jeg, at det tager lang tid at gøre ting færdige.}\\ 
\noindent
\textbf{Ja, altså passe og plejer?}\\
\noindent
\textit{Ja, passe og pleje og sørge for at tingene er i orden som det nu skal være, hvis det er helt rigtigt.}\\
\noindent
\textbf{Så det er ikke det fysiske, at det er pladsen heroppe, der mangler og at det er udstyret, men det er mere personalemangel nogen gange ift. opgavebyrden?}\\
\noindent
\textit{Ja, til opgavebyrden. Det er nok det rigtige ord. Det er fordi der mange, lige på min afdeling, at vi har nogle, hvor ting vitterligt tager lang tid og det bliver værre og værre, og der kommer flere og flere af den slags patienter ind fordi folk bliver ældre.}\\
\noindent
\textbf{Og det du siger er rygafdeling eller rygpatienter.}\\
\noindent
\textit{Ja, men vi har også mange med sår. Også har de ældre mennesker kommer typisk med andre skavanker, hvor det også tit er det vi skal tage os af, fordi det kan være sukkersyge eller dårlig blodomløb, hvor de kommer ind med et lille sår på en tå som vi skal tage os af.}\\
\noindent
\textbf{Så, når I har nogle elektive patienter og nogle akutte patienter, og når de her akutte patienter kommer og jeres arbejdsbyrde bliver større, løber I så bare hurtigere og gøre jeres arbejde hurtigere?}\\
\noindent
\textit{Ja.}\\
\noindent
\textbf{Udskyder I så de her elektive patienter og siger, at vi kan først operere jer i morgen eller hvordan fungerer det?}\\
\noindent
\textit{Ja, det kan sagtens tænkes. Men det er ikke os som sygeplejerske eller social og sundhedsassistent, der bestemmer det. Det er rent lægeligt administrativt.}\\
\noindent
\textbf{Okay, det er administrativt?}\\
\noindent
\textit{Ja. Man kan godt sige, at vi er spændt for, sådan rent teknisk eller administrativt, så kører det op på et lidt andet plan.}\\
\noindent
\textbf{Så er det ift., når I får de her patienter ind, vurderer I, hvor lang tid skal de ligge?}\\
\noindent
\textit{Nej, lad os sige kun rygpatienterne. Der vil jeg sige der har vi, hvis det er planlagt, så har vi, at det tager en uge f.eks., men nogle gange der går lidt længere tid, fordi der er ting der ikke lige går efter kalenderen, men ellers passer det nogenlunde. Men vi har mange sårpatienter, hvor det går langt over tiden fordi vi efterhånden er så specialiserede lægeligt og fordi vores afdeling er en sårafdeling så får vi alt, altså så får vi alle de patienter, der fejler noget med deres sår i en eller anden forbindelse.}\\
\noindent
\textbf{I forbindelse med vores projekt, går vi ind og kigger på om der er nogle parametre I går ind og vurderer. Nu siger du, vi har, hvis der er noget med ryggen, så ved I, at det tager en uge. Men er der noget andet f.eks. et skema I skal gå efter, hvor der er nogen ting I skal krydse af altså, hvor gammel er patienten er eller?}\\
\noindent
\textit{Nej, lige specielt rygsektoren har jo gjort sig god ved, at det er så systematiseret, at det er de samme kategorier end en skæv ryg på en ung pige. Der kommer næsten en hver $14$. dag, og de bliver jo opereret på samme måde og bliver passet på samme måde af fysioterapeuterne som kommer med det samme, så vi er hele tiden i gang med dem, så det forløb er planlagt til en uge.}\\
\noindent
\textbf{Okay, så det er vurdering ud fra erfaring simpelthen?}\\
\noindent
\textit{Ja, og ligesådan, vi har mange cancerpatienter, der kommer, hvor de pludseligt ikke kan mærke deres ben,  tværsnitssyndrom hedder det. De bliver også opereret sådan og sådan. Forløbet er sådan og sådan. De  skal op, de skal i gang og de skal holdes vedlige. Vi skal ikke presse dem, men vi skal sørge for, at de holder sig i gang.}\\
\noindent
\textbf{De skal være aktiverede?}\\
\noinden
\textit{Ja, hvorimod folk, der bliver amputeret, det er en anden kategori patienter, fordi de har tit dårligt blodomløb, sukkersyge, dårligt syn. Altså de er ikke så nemme at køre på den måde. Der skrider det hele.}\\
\noindent
\textbf{Så der er forskellige parametre indenfor hver kategori?}\\
\noindent
\textit{Ja meget.}\\
\noindent
\textbf{Og det er baseret på erfaring og ikke noget dokumenteret system?}\\
\noindent
\textit{Nej, det har jeg i hvert fald ikke styr på. Men rygsektoren det kører på de firkantede kasser og det kører bare.}\\
\noindent
\textbf{Så er det kun her til sidst om der er noget du har lyst til at tilføje eller noget du har lyst til at fortælle ift. jeres arbejde heroppe?}\\
\noindent
\textit{Nej, men det nye er, at man skriver til kommunen med udskrivelser og det gør man i så god tid som muligt, også kan man hele tiden rette på computeren, hvad man har skrevet eller hvad ens kollega har skrevet, så de hele tiden er opdateret.}\\
\noindent
\textbf{Så det er lidt det der nogengange begrænser jer heroppe, at det er svært at komme af med patienten bagefter eller?}\\
\noindent
\textit{Nej, hvis man er ude i god tid og får det skrevet ordentligt, og det ikke fordi, at man skal skrive så meget fordi det også blevet sådan et kasse system. Man skal bare huske det. Der er nogle gange vi glemmer det.}\\
\noindent
\textbf{Også, når der kommer akutte patienter?}\\
\noindent
\textit{Ja, men det er alligevel, at man skal skrive til kommunen, at man gør det og så nogle gange, hvis vi har lidt travlt, og man skal skrive det dagen før inden kl. $12$, tror jeg. Og hvis jeg først skal ind nu og skrive det nu, altså nu er kl. $13$. Se så skal jeg beholde patienten en dag mere.}\\
\noindent
\textbf{Så det er på den måde det foregår?}\\
\noindent
\textit{Ja, men der er så noget jeg ikke har helt styr på endnu, men der er noget, som hedder udskrivningsenheden. Jeg ved ikke om de sidder og kigger på, hvad jeg har skrevet, men når jeg så melder dem til i morgen eller i overmorgen, så ved jeg ikke om de sidder og kontrollerer og ringer til kommunen. Det har jeg ikke styr på.}\\
\noindent
\textbf{Når I flytter patienterne rundt, nu siger du, hvis I har en akut patient, så får de lov til at ligge på gangene først eller ryger de ind på stuen?}\\
\noindent
\textit{De ryger ind på stuen.}\\
\noindent
\textbf{Så prioriterer man så at få flyttet nogle af dem, der senere i deres helingsforløb eller elektive patienter ud eller?}\\
\noindent
\textit{Aldrig ude på gangen.}\\
\noindent
\textbf{Okay, det er ned på andre stuer eller?}\\
\noindent
\textit{Ja, så kan vi finde på det.}\\
\noindent
\textbf{Okay, så det er de elektive patienter og dem der er mindst kritiske akutte som I flytter ned på andre afdelinger?} \\
\noindent
\textit{Nej, ikke andre afdelinger. Det gør vi ikke så meget af, men vi kan godt finde på at flytte en patient fra vores afdeling over på vores naboafdeling, som også er en ortopædkirurgisk afdeling. I dagligdagen er vi delt op i to forskellige ting. Det hedder stadigvæk det samme og det er det samme. Der veksler vi hele tiden mellem patienterne, når det er de akutte, fordi om aftenen så kan de godt have travlt, hvor vi så tager deres patienter og næste dag, så er der faldet lidt ro over det, så hører de til derovre og så flytter vi dem derover. Men vi flytter dem ikke på en mave afdeling eller en medicinsk afdeling. Det er ikke det der foregår, men det vi snakkede om lige før med AMA, der har rådighed over hele sygehuset.}\\
\noindent
\textbf{Er der forskel på, hvordan man prioriterer de her patienter? Og hvem man flytter eller er det forskelligt?}\\
\noindent
\textit{Ja, der er det blevet sådan, at vi passer på børnene og vi passer på rygpatienter og vi passer på traumepatienter. Hvis der kommer nogle med hoftebrud, det er sådan blevet en folkesygdom. De hører til ovre på den anden afdeling, men dem passer vi også, hvis de ikke kan eller har tid til det, og har plads til dem. Men vi sender helst ikke børn over og vi må ikke sende rygpatienter over. Førhen hed det også knogle konstruktion, som også er meget specielt på vores afdeling. Dem vil vi heller ikke have andre steder, fordi det er sådan noget, hvor vi forlænger benene med noget apparatur og der er kun nogle få læger, som er her og de vil ikke have dem andre steder hen, fordi så skrider det hele og så går der noget galt med deres udstyr og varetagning. Det er ligesådan med rygpatienter. De vil altså helst have, at de er her, for de har oplevet, det er også det med erfaring. De har sendt folk ud af huset, hvor det andet sygehus ikke har varetaget det som de skulle eller har kørt deres eget system, så ødelægger det noget for patienten, også kan de måske starte forfra og det vil de ikke have mere.}\\ 
\noindent
\textbf{Så vil vi sige tak for interviewet.}\\
\noindent
\textit{Velbekomme.}\\


