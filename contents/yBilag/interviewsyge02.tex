\section{Sygeplejersker på sengeafsnit 02} \label{bilagO2}
\textbf{Hvem er du?}\\
\noindent
\textit{Jeg hedder Lars, og er almindelig sygeplejersker på gulvet.}\\
\noindent
\textbf{Hvor længe har du arbejdet på denne afdeling?} \\
\noindent
\textit{28 år.}\\
\noindent
\textbf{Må vi citere dig?}\\
\noindent
\textit{Ja det må vi godt.}\\
\noindent
\textit{Hvor arbejdstimer har du gennemsnitligt om ugen?}\\
\noindent
\textit{ Almindelig fuldtidsstilling, 37 timer.}\\
\noindent
\textbf{Hvordan er disse fordelt på ugen ift. gennemsnit om måned eller uge?}\\
\noindent
\textit{Vi har en arbejdsplan der hedder 8 uger i alt, hvor jeg så veksler mellem dagvagter og aftenvagter.}\\
\noindent
\textbf{Har i så vagtskifte når du snakker om dag og aften vagter, eller er der nogle bestemte tidspunkter disse vagtskifte finder sted.}\\
\noindent
\textit{Ja, på alle mine hverdage (dagvagter) arbejder jeg fra 7 til 15:45 og så møder jeg aftenvagt ind lidt tidligere, dette vil sige 15:30, så jeg har et kvarter til at sætte mig ind i, hvilke patienter der skal arbejdes med, mens de andre passer klokker.}\\
\noindent
\textbf{Så I får aflastet og får afvide at, det er disse patienter I skal varetage og får fordelt viden.}\\
\noindent
\textit{ja, meningen er at de skal sidde og læse hvis jeg skal af med og plus jeg skal tage klokkerne af på gangen.}\\
\noindent
\textbf{Hvor mange patienter varetager du på en normalt vagt? Har I et bestemt antal patienter der er tildelt?}\\
\noindent
\textit{ Nej, det er forskelligt fordi de veksler meget i sværhedsgraden eller, hvordan vi passer dem og selvfølgelig hvad de fejler - to til fire til seks patienter.}
\textbf{ Så det er to til seks på området.}\\
\noindent
\textit{Ja.}\\
\noindent
\textbf{Kan I blive indkaldt i jeres pauser? eller hvordan foregår det?}\\
\noindent
\textit{Det ved jeg faktisk ikke, altså jeg bliver altid indkaldt hvis lægen vil have fat i mig da vi går rundt med de her telefoner, fordi vi har skrevet os på tavler dernede, hvor man kan se, hvilke patienter jeg har, så kan man ringe efter mig, så der er egentlig ikke faste pauser, men man kan altid finde en og plus det er også vores klokke system dvs. at jeg kan se hvis jeg drikker kaffe så ringer telefonen og hvis jeg ikke har en telefon så ka jeg se på en display om det er min patient, som ringer både på stuen eller på toilettet.} 
\textbf{Er det et bestemt tidsrum i modtager de her patienter altså de elektive patienter dem, som er planlagt?}\\
\noindent
\textit{Det er der sådan set ikke helt, fordi vi får også afhængig af, hvilke dage det er patienter om eftermiddagen, der skal gøres klar til næste dag, men altså med indkaldte patienter møder man om morgenen typisk kl 7, der er der også sat en af til at modtage dem, så de bliver taget ude foran ved sekretæren, også kommer de ind på deres plads og så stiller man de spørgsmål som er relevante og så sørger man for at de er klædt om og parat til operationen, hvis de skal ned allerede ved 8 tiden eller lidt senere.}\\
\noindent
\textbf{Så i modtager dem om morgen og gør dem klar til i løbet af den?}\\
\noindent
\textit{Ja, det gør vi ved de fleste patienter.}\\
\noindent
\textbf{Så er det ift. akutte patienter det er døgnet rundt I modtager dem.}\\
\noindent
\textit{Ja, her på afdelingen. }\\
\noindent
\textbf{Ift. udskrivelse af patienter foregår det på et bestemt tidsrum på dagen, hvor man siger vi udskriver patienter fra kl. 10 til kl. 14 ellers udskriver vi ikke eller, hvordan foregår det?}\\
\noindent
\textit{Det er også meget forskelligt, fordi tildels så kommer de fra forskellige områder så ofte er det transporten, der nogle gange bestemmer, hvornår vedkommende bliver hentet og alt afhængig af, hvis det er Falck der skal hente dem eller om de skal til et andet sygehuset }\\
\noindent
\textbf{Så i udskriver løbende når der er behov for det.}\\
\noindent
\textit{ja det gør vi også.}\\
\noindent
\textbf{Så ift. når I vurderer om patienten kan udskrives hvor ofte gøres dette? Når der er stuegang eller er det flere gange i løbet af dagen eller, hvordan vurderes dette?}\\
\noindent
\textit{Nej vi har jo et, det er jo meget forskelligt fra patienter, da vi har små børn og vi har unge mennesker og ældre mennesker og de rigtig gamle - de unge mennesker og børn kan godt have en operation, hvor man skal have fat i hjemmesygeplejerskerne til at source dem, og det skal være planlagt i god tid også de andre, der aftaler man typisk over computeren om, hvordan man udskriver dem og, hvordan man sender dem hjem.}\\
\noindent
\textbf{Hvor ofte vurderer man så, hvornår de skal hjem og hvornår de er klar, er det en dialog med patienten eller?}\\
\noindent
\textit{ Ja, men typisk skal hjemmeplejeren også være blandet ind i det og de skal komme hen til dem og de vil ikke komme alt for sent, fordi der er også forskel på deres dag vagter, da de fleste af dem er på arbejde i dag vagterne og ikke har så mange på aftenvagterne, da det gælder om at få dem hjem i dags tiden, det meget vigtigt.}
\textbf{ Ift om der er nogle tidspunkter, hvor I ikke længere udskriver patienter og indlægger patienter fx om natten.}\\
\noindent
\textit{Ikke indlæggelser, da vi er en akut afdeling der tager vi alt, hvad der kommer og udskriver. Der kan godt have fundet en trafikulykke, hvor vi tror der er en hel masse, men hvor det hele bliver afviklet eller det hele bliver godkendt at patienten ikke fejler noget og det kan typisk være en knallertulykke eller en bilulykke, hvor man kommer ind og får lavet en observation og det kan være om eftermiddagen eller det kan være om aftenen, hvor lægen kommer op og vurderer patienten og siger og at man ikke fejler noget og du må have lov til at komme hjem, men det er deres afgørelse.}\\
\noindent
\textbf{Nu siger du lægen? Er det læger der vurderer patienterne, som kommer ind akut eller er det en kirurgi der?}\\
\noindent
\textit{Altså jeg har kun kirurger på min afdeling, og jeg må ikke som sygeplejerske sende nogen hjem på eget initiativ.}\\
\noindent
\textbf{Så lægen skal have været forbi?}\\
\noindent
\textit{Lægen skal have godkendt og jeg skal have været i kontakt med en læge på en eller anden måde for at spørge om vedkommende må komme hjem og det kan også være en brækket arm i løbet af formiddagen eller eftermiddagen som er blevet opereret, og hvis de har det godt ift. smerter og ingen kvalme så må man gerne have lov til at sende dem hjem om aftenen, hvis lægen giver god for det og det kan godt være over telefonen.}\\
\noindent
\textbf{Har I nogle sengepladser som i altid forbeholder til akutte patienter ift. når i fordeler sengene?}\\
\noindent
\textit{ Nej, det kan vi ikke mere fordi vi har rimelig mange patienter til daglig og vi er faktisk også begyndt og lade være med tænke på 2 eller 3 mænd på en stue eller kvinder . nu bruger vi bare de pladser vi har så vi ikke skal flytte rundt. En kvinde kan godt overnatte med to mænd fordi vi har forhæng for.}\\
\noindent
\textbf{Så der er noget diskretion?}\\
\noindent
\textit{Ja. Og så er der nogle der bliver flyttet om næste dag.}\\
\noindent
\textbf{Er I nødt til at have patienter liggende på gangen nogle gange? Måske ikke lige natten over, men altså bare dagen?}\\
\noindent
\textit{Ja det har vi gjort, men brandvæsnet er efter os. Vi må ikke. For nogle år tilbage, der var vi fuldstændig fyldt op også alle vores patienter har jo på et eller andet tidspunkt brug for en hjælpende hånd (gangstativ eller krykker), som står oppe ved os til den enkelte patient. Det stod engang på gangen og det må det ikke for brandtilsynet. Så kom vi det ind på stuerne også kom de ekstra patienter ud på gangene og det må de ikke, også siger brandtilsynet at vi skal køre patienterne ind på stuerne da deres sikkerhed er væsentlig højere end hjælpemidlerne ift. brand. Så hjælpemidlerne må vi blot placere i et andet rum, der må ikke være noget på gangen og det gjorde så at vi for et par år siden, hvor vi normalt har 4 sengestuer sagde vi så, at vi kun vil have 3 sengestuer for at have tre patienter på stuen, men så havde de et hjørne hvor det så hed 4 senge til alle hjælpemidler.}\\
\noindent
\textbf{Så i har valgt at sige nu tager vi det her udstyr og flytter det ind på stuen.}\\
\noindent
\textit{Ja og hvis der så kommer en patient mere så må vi flytte udstyret.}\\
\noindent
\textbf{jamen så er det lidt ift., hvis I så får for mange patienter altså akutte patienter og I ikke kan placere dem på de stuerne, da de er fyldt op, hvad gør I så? Vi har læst, at der er trådt nogle regler i kræft med at de bliver flyttet ned til andre stuer og andre afdelinger og om du kan bekræfte det?}\\
\noindent
\textit{ja, det bliver bare styret et andet sted fra, det blir styret fra Ama, Akut modtagelse medicin afsnit, der sidder nogle koordinator, som skal sørge for at deres patienter der ovrefra eller modtagelsen (det medicinske regi) at der er plads for dem, det vil sige, hvis de medicinske afsnit ikke har pladser nok så kan vi risikerer, at vi skal have en medicinsk patient eller to og det står den koordinator for.} \\
\noindent
\textbf{Så det er ikke jer der styrer de patienter, hvordan de flyttes?}\\
\noindent
\textit{Nej, vi skal melde tilbage, jeg ved ikke om vi skal i dagligdagen, men I weekenderne, der skal vi melde tilbage inden kl 11 ellers så er det kl 9 til koordinatoren og sige, at vi har to ledige pladser ellers forventer vi to ledige pladser lidt senere over middagen eller sådan noget, også har de råderum over dem i princippet.}\\
\noindent
\textbf{Ja indtil I får en akut patient der overtager.}\\
\noindent
\textit{Ja, så finder vi så ud af, hvad vi skal gøre.}\\
\noindent
\textbf{Så vil vi høre hvis I har nogle af de her ekstra patienter ift. hverdage så bliver I sat på arbejde til at udfylde stuen eller udfylde afdelingen ift. hvor meget personale der møder op eller, hvordan fungerer det?}\\
\noindent
\textit{Nej altså der jo planlagt, hvem der møder ind på den der 8 ugers plan så de skal bare møde ind, og så skal vi fordele patienterne når vi kommer om morgenen og det har så gjort, at vi har typisk har fire til fem specialer som vi selv har og er valgt på baggrund af egen interesse så jeg har i en lang periode passet rygpatienter primært og det kan godt være jeg skal passe børn og det kan også være jeg skal passe noget andet eller tredje eller medicinske patienter det er forskellig fra dag til dag, men vi prøver og koncentrerer os om dem vi har valgt os ind på.}\\
\noindent
\textbf{Ja fordi det lidt ift., hvis der kommer for mange patienter og ikke kan varetage dem, så har vi læst, at I har et vikarbureau som I kontakter og I får noget ekstra personale ind, og det kan du bekræfte at det er sådan det fungerer?} \\
\noindent
\textit{Ja sådan set, hvis vi er i nød så må vi ringe.}\\
\noindent
\textbf{Ja også ift. overarbejde om det er noget I bliver meget påvirket af, altså I bliver nødt til at blive ekstra, hvis I for eksempel har de her vagtskifte om I bliver nødt til at blive lidt ekstra ift. det ny kommende vagthold}\\
\noindent
\textit{Nej, det synes jeg ikke er noget problem eller det er meget sjældent.}\\
\noindent
\textbf{Okay, så det er sjældent det sker.}\\
\noindent
\textit{ Ja, fordi så er mig der hænger på det eller de to andre, fordi vi er jo der til at lappe over og de resten er jo taget hjem, så der er kun tre at trække på, hvis der er nogle som ikke møder op eller om vi har overset noget, eller der er noget akut. }\\
\noindent
\textbf{Nu har vi har arbejdet med det her med at I nogle gange har for mange patienter eller for lidt personale på arbejdet altså, hvad er det som begrænser jer er det plads på afdelingen eller er det fordi I ikke har nok personale eller, hvornår er det de har stress situationer opstår?}\\
\noindent
\textit{Altså det er sjældent vi har for mange patienter, vi kan godt have mange patienter, men det behøver altså ikke være for mange, men det er nærmere at vi ikke er nok personale til dem vi har af de patienter vi har, da vores patienter kræver meget altså, det jeg mener med kræver meget er, at de kræver meget tid og det tager tid for at gøre tingene færdig. } 
\textbf{Ja, altså passer og plejer?}\\
\noindent
\textit{Ja passer og plejer og få tingene i orden, som det nu skal være, hvis det nu er helt rigtigt.}\\
\noindent
\textbf{Så det er ikke det fysiske at det er pladsen her op der mangler eller at det er udstyret men at det er men mere personalemangel nogle gange? Ift. opgavebyrden?}\\
\noindent
\textit{Ja, det er nok opgavebyrden. Det er fordi der mange lige på afdelingen vi har, hvor nogle ting tager vitterligt langt tid og det bliver værre og værre, og der kommer flere og flere af den slags patienter ind fordi folk bliver ældre.}\\
\noindent
\textbf{Og det du siger er rygafdeling eller rygpatienter.}\\
\noindent
\textit{ Ja, men vi har også mange med sår også har de ældre mennesker, som kommer typisk andre skavanker, hvor det også tit er det vi skal tage os af, fordi det kan være sukkersyge eller dårlig blodomløb, hvor de kommer ind med et lille sår på en tå som vi skal tage os af.}\\
\noindent
\textbf{Men så når i har de her elektive patienter og akutte patienter også når de her akutte patienter kommer og jeres arbejdsbyrde bliver større løber I så bare hurtigere og gøre jeres arbejde hurtigere?}\\
\noindent
\textit{Ja.}\\
\noindent
\textbf{Udskyder I så de her elektive patienter og siger vi kan først operere jer imorgen eller, hvordan fungerer det?}\\
\noindent
\textit{Ja, det kan sagtens tænkes, men der er ikke os som sygeplejerske eller folk som social og sundhedshjælper der bestemmer det, men lægerne og det administrative.}\\
\noindent
\textbf{Okay, så det er det lægelige og det administrative?}\\
\noindent
\textit{Ja, man kan godt sige vi er spændt for rent teknisk eller administrativt ellers kører det op på et lidt andet plan.}\\
\noindent
\textbf{Så er det ift. når I får de har patienter ind vurderer i så, hvor lang tid skal de ligge?}\\
\noindent
\textit{Nej,  kun rygpatienterne der vil vi sige, hvis det er planlagt så har vi,  at det tager en uge for eksempel, men nogle gange går der lidt længere tid, fordi der er ting der ikke lige går efter kalenderen, men ellers passer det nogenlunde, men vi har mange såre patienter, hvor det går langt over tiden, fordi vi efterhånden er så specialiserede og fordi vores afdeling er en sårafdeling så får vi alt, altså alle de patienter der fejler noget i en eller anden  forbindelse.}\\
\noindent
\textbf{I forbindelse med vores projekt går vi ind og kigger på om der er nogle parametre som I går ind og vurderer, som I synes når I ved for eksempel, hvis der er noget med ryggen om I så ved eller om der er et skema som I skal gå efter eller om der er nogle ting i skal krydse af altså, hvor gammel er patienten eller.}\\
\noindent
\textit{Nej, lige specielt rygsektoren har jo gjort sig god ved at det er så systematiserede at det er de samme kategorier end en skæv ryg på en ung mand eller en ung pige som kommer her næsten hver 14. dag. Og de bliver jo opereret på samme måde og bliver passet på samme måde af fysioterpeuterne, som kommer med det samme, så vi er hele tiden i gang med dem så det forløb er planlagt til en uge.}\\
\noindent
\textbf{Okay, så det er vurdering ud fra erfaring simpelthen?}\\
\noindent
\textit{Ja, og lige sådan vi har mange cancerpatienter der kommer og har pludseligt eller, hvor de ikke kan mærke deres ben eller tværsnitssyndrom de bliver også opereret sådan og de skal hele tiden holdes vedlige vi skal ikke presse dem, vi skal sørge for at holde dem i gang.}\\
\noindent
\textbf{De skal være aktiverede, altså at patienterne skal holde sig i gang.}\\
\noindent
\textit{Ja, hvorimod folk der bliver amputeret det er en anden kategori patienter, fordi de tit har dårlig blodomløb eller sukkersyge, de er ikke så nemme at køre på den måde, der skrider det hele.}\\
\noindent
\textbf{Så der er forskellige parametre indenfor forskellig kategori.}\\
\noindent
\textit{Ja meget.}\\
\noindent
\textbf{og det er baseret på erfaring og ikke noget dokumenteret system.}\\
\noindent
\textit{ Nej.}\\
\noindent
\textbf{Har du lyst til at tilføje noget eller?}\\
\noindent
\textit{Nej, men det nye er det at man skriver til kommunen med udskrivelser og det gør man i så god tid som overhovedet muligt også kan man hele tiden rette på computeren ift., hvad man har skrevet eller hvad ens kollega har skrevet så de hele tiden er opdateret. }
\textbf{Så det er lidt det som nogle som begrænser jer her oppe at det nogle gange kan være svært at komme af med patienter bagefter eller?}\\
\noindent
\textit{Nej det er selvfølgelig i god tid og at man får det skrevet ordentligt og det ikke fordi man skal skrive så meget fordi det også blevet sådan en kasse system, man skal bare huske det, der er nogle gange vi glemmer det.}\\
\noindent
\textbf{Og så når der kommer akutte patienter?}\\
\noindent
\textit{Ja, men skal skrive til kommunen at man gør det og når man nogle gange har lidt travlt så skal man skrive det dagen før kl 12 og, hvis jeg først skal ind nu og skrive ind altså nu er klokken 13 så skal jeg beholde patienten en dag mere.}\\
\noindent
\textbf{Så det er på den måde det foregår?}\\
\noindent
\textit{Ja, men jeg har ikke helt styr på det, men der er noget som hedder udskrivningsenheden, jeg ved ikke om de sidder og skriver eller kigger på, hvad jeg har skrevet, men når jeg så melder det i morgen eller overmorgen så ved jeg ikke om de ser på det eller kontrollerer det eller ringer til kommunen, det har jeg ikke styr på.}\\
\noindent
\textbf{Når i flytter patienterne rundt, nu siger du når I har akut patienter så får de lov til at ligge på gangen først eller ryger de ind på stuen.}\\
\noindent
\textit{De ryger ind på stuen.}\\
\noindent
\textbf{Så prioriterer man så at få flyttet nogle af dem der senere i deres helingsforløb eller elektive patienter ud eller?}\\
\noindent
\textit{ Aldrig ude på gangen.}\\
\noindent
\textbf{Okay, så på andre stuer eller.}\\
\noindent
\textit{Ja, så kan vi finde på det.}\\
\noindent
\textbf{Okay så de elektive patienter er dem der er mindre kritiske end akutte som I flytter ned på andre afdelinger.} \\
\noindent
\textit{Nej - ikke andre afdelinger - det gør vi ikke så meget af, men kan godt finde på at flytte en patient fra vores afdeling over til vores nabo som også er en Ortopædkirurgisk afdeling. I dagligdagen er vi delt op i to forskellig ting, det hedder stadigvæk det samme, men der veksler vi stadigvæk mellem patienterne når de er akutte, fordi om aftenen så kan de også have travlt, hvor vi så tager deres patienter og næste dag så er der faldet lidt ro over det så hører de til derovre og så flytter vi dem derover, men vi flytter dem ikke over på en mave afdeling eller en medicinsk afdeling det er ikke det som foregår, men det vi snakkede om lige før med Ama, der har rådighed over sygehuset.}\\
\noindent
\textbf{Er der forskel på, hvordan man prioriterer de her patienter? Og hvem man flytter?}\\
\noindent
\textit{Ja, der har vi også eller det er blevet sådan at vi passer på børnene og vi passer på rygpatienter og vi passer på traumepatienter, hvis der kommer nogle med hoftebrud, det er sådan blevet en folkesygdom, de hører til ovre på den anden afdeling, men dem passer vi også, hvis de ikke kan eller har tid til det og har plads til dem, men vi sender helst ikke børn over eller rygpatienter derover. Førhen hed det også knogle konstruktion, som også er meget specielt på vores afdeling dem vil man jo heller ikke have andre steder, fordi det er sådan noget, hvor vi forlænger benene med noget apparatur og der er kun nogle få læger, som er her og ikke er andre steder, fordi så skrider det hele også er der noget som går galt, med deres udstyr og varetagning og således er det også med rygpatienter de vil altså helst have de er her og har oplevet når man sender folk ud af huset at man ikke har varetaget folk ordentligt som de skulle eller kørt deres eget system så ødelægger det for patienten også kan de måske starte forfra og det vil de ikke have mere.}   \\
\noindent
\textbf{Så vil vi sige tak for interviewet.}\\
\noindent
\textit{Velbekomme.}\\


