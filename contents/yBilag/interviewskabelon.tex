\chapter{Bilag A}

\section{Interview skabelon} \label{bilagA}
*** HER SKAL STÅ EN INDLEDNING ***
\subsection{Eksklusionskriterier}
Vi ønsker at snakke med 3 sygeplejersker (ikke samtidig)
Vi forventer, at interviewet varer omkring 30 minutter per sygeplejerske.
Vi ønsker, at sygeplejerskerne skal have arbejdet på ortopædkirurgisk afdeling på Aalborg Universitetshospital i mindst 1 år.
Vi ønsker ikke, at personalet ser spørgsmålene på forhånd, da vi foretrækker en åben dialog.
Interviewet skal bruges til at underbygge argumenter i rapporten, og personalet vil derfor muligvis blive citeret.


Vi er 5. semester sundhedsteknologistuderende på Aalborg Universitet. Vi har fået stillet projektforslaget: Risikovurdering på ortopædkirurgisk afdeling af Sten Rasmussen (Overlæge på ortopædkirurgisk afdeling). I forhold til dette har vi valgt at undersøge, hvordan indlæggelsesvarigheden kan estimeres på afdelingen med henblik på bedre kapacitetsudnyttelse. Formålet med interviewet er at belyse aspekter, som ikke er mulige at finde i litteratur og dertil bekræfte funden litteraturs validitet.  


Introduktion:
Interviewet omhandler udelukkende ortopædkirurgisk afdeling på Aalborg Universitetshospital, vi frabeder os derfor sammenligninger med andre hospitalsafdelinger. Interviewet forventes at tage 30 minutter. Interviewet er set som en dialog så, hvis du har lyst til at byde ind undervejs er dette i orden.


Hvem er du?
Hvad er din stilling?
Hvor længe har du arbejdet på ortopædkirurgisk afdeling på Aalborg Universitetshospital?
Må vi citere dette interview i rapporten?

\subsection{Interviewet med sygeplejersker på ortopædkirurgisk afdeling}
\textbf{Hvor mange arbejdstimer har du om ugen?}
Der ønskes svar på: Gennemsnit, fordelingen af vagter uge for uge, længden af vagten, forskel på dag- og nattevagter, hvordan foregår vagtskifte. 

\textbf{Hvilke arbejdsopgaver har du på en normal vagt?}
Der ønskes svar på: Hvor mange patienter sygeplejesker varetager. 

\textbf{Hvordan forløber dine pauser?}
Der ønskes svar på: Sygeplejersker skal være til rådighed under pauser samt, hvorvidt denne er påtvungen. 

\textbf{Hvilke patienter modtager i på afdelingen?}
Der ønskes svar på: Hvordan og hvorvidt patienterne skemalægges. Hvordan foregår indlæggelses og udskrivelse samt, hvorvidt der er nogle faste tidspunkter. Planlægges elektive patienter ud fra pladsen til akutte patienter.  

\textbf{Hvor mange sengepladser har i til rådighed på afdelingen?} 
Der ønskes svar på: Er der nogle sengepladser der er forbehold akutte patienter.


\textbf{Hvad sker der, hvis i ikke har flere sengepladser til rådighed på afdelingen?}
Der ønskes svar på: Hvor placeres patienterne, har afdelingen et samarbejde med andre afdelinger, er der prioritering mellem patienterne, fordeles patienterne mellem jer eller tilkaldes der ekstra personale, hvorfor opstår problemet og hvordan begrænser dette sygeplejerskerne, hvad gør de på afdelingen for at løse problemet udskydes elektive patienter. 

\textbf{Hvad sker der, hvis i har for mange sengepladser til rådighed?}

\textbf{Er der en standardliste med checkpunkter af parametre som altid skal registreres for patienter?}
Der ønskes svar på: Hvilke parametre sygeplejerskerne kigger på. 

\textbf{Er der noget du tænker, der er relevant at tilføje? }

\subsection{Interview med lægesekretær på ortopædkirurgisk afdeling}
\textbf{Hvordan planlægges elektive patienter?}
Der ønskes svar på: Planlægges patienter med kort eller lang varsel, estimeres der hvor længe patienten er indlagt med henblik på at planlægge elektive patienter forud, vurderes der specielle parametre i forhold til planlægningen. 

\textbf{Er der noget du tænker, der er relevant at tilføje?}




