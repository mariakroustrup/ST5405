\chapter{Bilag}

\section{Interview skabelon} \label{bilagA}
Før interviewet udført på ortopædkirurgisk afdeling på Aalborg Universitetshospital blev der opstillet eksklusionskriterier på baggrund af formålet med interviewet. Det ønskes, at interviewe to sygeplejersker samt en bookingansvarlig fra ortopædkirurgisk afdeling, som mindst har været ansat på afdelingen i et år. Derudover ønskes det, at interviewet foregår individuelt samt at personalet ikke har set spørgsmålene på forhånd, da der ønskes en åben dialog. Inden interviewet fik informanten introduktion ift. hvad interviewet omhandlede. Skabelonen er opdelt ift. interview med sygeplejesker og bookingansvarlig. 

\subsection{Interviewet med sygeplejersker}
\textbf{Hvor mange arbejdstimer har du om ugen?} \\
\noindent
Der ønskes svar på: Gennemsnit, fordelingen af vagter uge for uge, længden af vagten, forskel på dag- og nattevagter, hvordan foregår vagtskifte. 

\noindent
\textbf{Hvilke arbejdsopgaver har du på en normal vagt?} \\
\noindent
Der ønskes svar på: Hvor mange patienter sygeplejesker varetager. 

\noindent
\textbf{Hvordan forløber dine pauser?} \\
\noindent
Der ønskes svar på: Sygeplejersker skal være til rådighed under pauser samt, hvorvidt denne er påtvungen. 

\noindent
\textbf{Hvilke patienter modtager i på afdelingen?} \\
\noindent
Der ønskes svar på: Hvordan og hvorvidt patienterne skemalægges. Hvordan foregår indlæggelses og udskrivelse samt, hvorvidt der er nogle faste tidspunkter. Planlægges elektive patienter ud fra pladsen til akutte patienter.  

\noindent
\textbf{Hvor mange sengepladser har i til rådighed på afdelingen?}  \\
\noindent
Der ønskes svar på: Er der nogle sengepladser der er forbehold akutte patienter.


\noindent
\textbf{Hvad sker der, hvis i ikke har flere sengepladser til rådighed på afdelingen?} \\
\noindent
Der ønskes svar på: Hvor placeres patienterne, har afdelingen et samarbejde med andre afdelinger, er der prioritering mellem patienterne, fordeles patienterne mellem jer eller tilkaldes der ekstra personale, hvorfor opstår problemet og hvordan begrænser dette sygeplejerskerne, hvad gør de på afdelingen for at løse problemet udskydes elektive patienter. 

\noindent
\textbf{Hvad sker der, hvis i har for mange sengepladser til rådighed?} \\

\noindent
\textbf{Er der en standardliste med checkpunkter af parametre som altid skal registreres for patienter?} 


\noindent
Der ønskes svar på: Hvilke parametre sygeplejerskerne kigger på. 

\noindent
\textbf{Er der noget du tænker, der er relevant at tilføje?}

\subsection{Interview med lægesekretær}
\textbf{Hvordan planlægges elektive patienter?} \\
\noindent
Der ønskes svar på: Planlægges patienter med kort eller lang varsel, estimeres der hvor længe patienten er indlagt med henblik på at planlægge elektive patienter forud, vurderes der specielle parametre i forhold til planlægningen. 

\noindent
\textbf{Er der noget du tænker, der er relevant at tilføje?} \\




