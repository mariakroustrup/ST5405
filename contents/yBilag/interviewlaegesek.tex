\section{Interview med sygeplejersker}





\section{Interview med lægesekretær}

\textbf{Hvor længe har du arbejdet her?}
\textit{Det har jeg i fire år.}

\textbf{Fire år?}
\textit{Ja. Og så har jeg været i ortopædkirurgien, 7 år tidligere i min tid som lægesekretær.}


\textbf{Og må vi citere dig for interviewet?}
\textit{Ja.}


\textbf{Jamen nu vil jeg spørge, hvordan planlægger I de her elektive patienter?}
\textit{Vi har faste pladser. Vi har ikke så mange elektive under indlæggelse, men vi har rigtig mange elektive i dagkirurgi. I dagkirurgi har vi rigtig mange tider og  holder altid nogle enkelte ledige til, hvis der dukker noget akut op under indlæggelse. Det er kun de patienter, der er absolut nødvendigt skal opereres under indlæggelse, der kommer derind og det er ikke altid vi faktisk udnytter alle vores tider. I denne her uge har vi for eksempel ikke brugt vores pladser som vi så har været heldige med fordi der har været mange akutte patienter, så de har haft meget overbelægning eller mange patienter deroppe, så det har faktisk været heldigt i denne her uge. Men dem planlægger vi sådan, at vi som regel altid har plads til lidt ekstra. Vi booker ikke stramt op, kun i absolut nødstilfælde og det er, hvis der kommer en ekstra en på som man er nødt til at skal have plads til.}

\textbf{Er der en procentsats eller en del af sengene, der er afsat til akutte patienter?}
\textit{Nej det tror jeg ikke. Ikke mig bekendt. Altså vi må operere om tirsdagen for vores vedkommende og der må vi sætte en til to patienter på afhængigt af, hvor stort et indgreb det er. Det er af hensyn til operationsgangen og deres kapacitet der. Det er egentligt ikke af hensyn til sengeafdelingen. Det er en del af vores elektiv. Det ved de, at der kan komme to, hvis det er to små operationer. Men en del af de patienter vi sender op på sengeafdelingen, selvom de er nødt til at skulle opereres under indlæggelse, så er det måske noget i forhold til anæstesi sikkerhed og opvågning. Mange af de patienter kan måske godt allerede gå hjem senere på dagen. Det er ikke nødvendigvis ensbetydende med, at de skal ligge deroppe i flere dage. Men der er også dem, der så kommer der op fordi de netop skal ligge der i nogle dage, fordi de skal træne efter operationen. Der er ikke som er 100 procent fastlagt. 


\textbf{Når I indkalder de her elektive patienter, bliver det så gjort med kort eller lang varsel? Hvordan foregår det?}
\textit{Jamen, hvis der sidder en patient henne i ambulatoriet i dag, der skal opereres under indlæggelse, så vil patienten få en tid med hjem i dag. Og hvis det ikke er noget, der haster og hvis det ikke er noget akut, så finder vi første ledige tid under de omstændigheder der nu er. Skal det være en bestemt læge, er det patienten, der siger jeg vil gerne vente til efter 15. December eller under de forhold, der nu er der. Er det sådan at patienten siger første ledige tid, jamen så vil de simpelthen få første ledige tid og det kan godt være i næste uge. Det kan også være om 14 dage, tre uger.}


\textbf{Hvis så man er nødt til at rykke den elektive patient under akutte patienter, hvordan gør man så?}
\textit{Det er igen et spørgsmål om for vores område fordi vi ikke er så stramt booket op under indlæggelse. Så har vi plads til, at vi kan putte dem på måske en uge, måske 14 dage senere. Altså vi får dem på forholdsvis hurtigt og vi skal nok sørge for, at de får en tid. Vi kan også løbe ind i perioder, hvor vi har mange, der skal ligge under indlæggelse. Og så kan vi selvfølgelig godt løbe ind i kniberi med, hvad gør vi så her. Og så er det egentligt vores specialeansvarlige overlæge, der træder ind og siger, jamen så gør vi sådan her. Så er det ham, der ligesom træffer beslutningen om, hvordan håndterer vi det.} 


\textbf{Når I indkalder de her patienter, planlægger de elektive patienter, planlægger I så også efter for eksempel operationstype, hvor lang tid skal de ligge der?}
\textit{Nej.}


\textbf{Er der noget du føler er relevant lige at tilføje.}
\textit{Nej, ikke andet end det her er jo kun for sengeafdelingen. Vores O6 kører jo ligesom et forløb for sig selv. Det er ambulant kirurgi. Det kører ligesom sit liv for sig selv, men der har vi også altid buffer, hvis der er nogen, der bliver aflyst. Vi prøver på at have en regel om, at den første tid bliver som regel ikke booket. Det er den vi holder tilbage, fordi det sjældent, at det er den første patient man aflyser, hvis man aflyser, så opererer man som regel de patienter, der er mødt ind i dagkirurgi. Så vælger man at aflyse dem der endnu ikke er mødt. Dem der først kommer senere på dagen. Så på den måde, hvis der er nogen, der bliver aflyst i dag og de så får en tid på mandag fra morgenstunden af, jamen så møder de ind, så bliver de opereret, så man undgår, at de bliver aflyst mere end én gang. Men hvis vi sætter dem på senere på dagen, så kan vi risikere, at de rent faktisk bliver aflyst en gang mere, hvis det skulle være. 7-9-13 og bank under bordet, det sker heldigvis ikke ret tit, men det kan jo ske på grund af sygdom.}


\textbf{ Bare lige om jeg har forstået det korrekt, så snakker du noget om, at I ud fra, når I planlægger, så planlægger I om, altså ud fra om kirurgen også, om den er tilstede den dag det skal være, så det gør I ud fra erfaring og operationstype.}
\textit{Det er ud fra, det er lægen selv, der har skrevet på. Lægen udfylder en operations tilmelding nu når de skriver patienter op til operation og derpå skriver han skal jeg selv operere denne her patient eller kan det være en anden der gør det. Patienten får også selv lov at vælge og sige jamen ønsker du, at det skal være den læge. Hvis ikke han har skrevet noget på, så får patienten selv lov at vælge. Jamen jeg vil gerne have, at det var den eller nej, jeg vil bare gerne have så hurtig en tid jeg kan få.}


\textbf{Okay, så det er de overvejelser I gør, når det er?}
 \textit{Ja, det er ligesom vores sygeplejersker, der sidder og har papirene og siger jamen vi skal have en tid, der passer ind med det her, men patienten vil bare gerne ind så hurtigt som muligt. Det kan også godt nogen gange være, at lægen skriver han gerne selv vil gøre det, men hvis der går et stykke tid og patienter siger jeg vil bare gerne have det overstået, så kan man måske godt sige okay, fint nok. Det er ikke noget, der er noget specielt kompliceret i, så det vil en anden også godt kunne gøre, det her.}