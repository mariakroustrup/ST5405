\section{Interview med sygeplejersker fra 01}
\textbf{Hvem er du?} \\
\noindent
\textit{Jeg er sygeplejerske på O1, Aalborg sygehus syd.} \\
\noindent
\textbf{Og hvor længe har du været på afdelingen?}\\
\noindent
\textit{Snart 10 år} \\
\noindent
\textbf{Så vil vi høre om vi må citere dig for interviewet og optage?}\\
\noindent
\textit{ Ja, det må I gerne.}  \\
\noindent
\textbf{Her til at starte med så vil vi spørge om, hvor mange arbejdstimer har du om ugen?} \\
\noindent
\textit{34} \\
\noindent
\textbf{Så det er ikke fuldtids, men?} \\
\noindent
\textit{Ja, det svarer til at man har 3 dage fri på 8 uger eller sådan noget.} \\
\noindent
\textbf{Så du kører efter et 8 ugers vagtplan?} \\
\noindent
\textit{Al plan bliver kørt over de 8 uger, hvor det skal gå op i et antal timer.} \\
\noindent
\textbf{Okay, så det er 34 timers gennemsnit om ugen over 8 uger?} \\
\noindent
\textit{Ja.} \\
\noindent
\textbf{Så din vagtlængde det er 7,5 time? } \\
\noindent
\textit{8 eller 9 timers vagter.} \\
\noindent
\textbf{Okay, og det er dag og nat? Eller hvordan fungerer jeres skift?} \\
\noindent
\textit{Jeg kører dag-aften, men nogen kører dag-nat og nogle kører mere eller mindre primært nat eller aften.} \\
\noindent
\textbf{Så der er 3 vagtskift?} \\
\noindent
\textit{Der er 3 vagtskift.} \\
\noindent
\textbf{Okay, det passer med de 8 timer.} \\
\noindent
\textbf{Når du har her på en normal vagt, når du varetager patienter, hvor stor er en patient byrde? hvor mange patienter varetager du?} \\
\noindent
\textit{Hvor mange jeg har ansvaret for?} \\
\noindent
\textbf{Ja} \\
\noindent
\textit{Det er jo ikke primær pleje, så det er sådan lidt. Vi har tit gruppepleje, det betyder, at så er man jo 2 om et vis antal patienter. Men det varierer rigtig meget, hvor mange jeg har ansvaret for.} \\
\noindent
\textbf{Kan du sætte størrelse på, hvor mange I er sammen?} \\
\noindent
\textit{I dagvagt har vi måske typisk, er vi 2 om måske 6-8 patienter. sådan typisk alt efter om vi har overbelægning eller ej, men deromkring. I aftenvagten er vi jo 2 om 10-12 patienter.
} \\
\noindent
\textbf{Nu siger du overbelæning, er det når i har?} \\
\noindent
\textit{Mere end de 20 patienter vi er normerede til.} \\
\noindent
\textbf{Okay, så det er når i har mere end de 20 normerede sengepladser. } \\
\noindent
\textbf{I forhold til dine pauser, hvordan forløber de? Kan I blive indkaldt i jeres pauser? eller har I påtvunget, at I skal holde pauser?} \\
\noindent
\textit{Pauserne, det er når det passer ind i forhold til arbejdsrytmen og nogen dage er der ikke pauser.} \\
\noindent
\textbf{Og man kan blive indkaldt fra sine pauser?} \\
\noindent
\textit{Ja, vi har jo telefon. Vi har sådan nogle dæktelefoner på os, så læger kan ringe til os og vi kan blive kaldt på med opkald fra gangen, fra patienter og sådan noget.}
\textbf{Så i er disponible?} \\
\noindent
\textit{Vi er disponible.} \\
\noindent
\textbf{Fra I møder til I går hjem?} \\
\noindent
\textit{Ja, det er vi.}  \\
\noindent
\textbf{Når I modtager patienter på afsnittet, de elektive patienter, skemalægges det ift. de indlægges på et bestemt tidspunkt eller?
Informant: De planlagte operationer?} \\
\noindent
\textit{Ja, de planlagte.} \\
\noindent
\textit{De kommer typisk om morgenen, fordi de planlagte operationer er jo i dagstid. så det kun, hvis det er den første på programmet, og de bor langvejsfra, så kan de komme dagen før nogen gange. så kommer de måske typisk kl 8 om aftenen.} \\
\noindent
\textbf{Så de bliver indskrevet måske et halvt døgn før indgrebet?} \\
\noindent
\textit{Ja.} \\
\noindent
\textit{Heroppe nogle gange og nogle gange på patienthotellet, det er sådan nyt, at der er nogen der går der ned. Og så kommer de så herop om morgenen, nogle gange kommer de her alligevel.} \\
\noindent
\textbf{Og de akutte patienter de indskrives døgnet rundt?} \\
\noindent
\textit{ Ja, det gør de.} \\
\noindent
\textbf{Er der et bestemt tidsrum I udskriver patienter på? Er der et tidspunkt på døgnet I ikke udskriver patienter på?} \\
\noindent
\textit{Langt de fleste udskrivelser foregår måske mellem 12 og 16, medmindre de er planlagt dagen før, så kan det være om formiddagen. Men det er jo typisk, fordi der først er stuegang kl 9, det er først der lægerne kommer, så 9, kvart over 9. Så før at man får kørt stuegang igennem og de praktiske ting er klar før personen kan komme hjem, så det typisk over middag før de kan. } \\
\noindent
\textbf{Hvor ofte vil du sige, at man vurderer patienternes tilstand, hvornår de kan sendes hjem?}  \\
\noindent
\textit{Altså det vurderer vi jo dagligt i forhold til, hvad vi tænker er realistisk i forhold til hvornår de skal hjem. og tager jo egentligt relativt hurtigt i forløbet af patienterne stilling til, hvad de tænker, og så sender vi jo, hvis de skal have hjælp, plejeforløbsplan afsted. sådan at de kan udskrives.} \\
\noindent
\textbf{Så det vurderes i starten, hvor lang tid kommer de til at ligge og så når det nærmer sig det tidsrum de skulle ligge, så vurderer man så igen?} \\
\noindent
\textit{Ja.} \\
\noindent
\textbf{Fint. Så er det om der er nogen tidspunkter I ikke indlægger patienter. Nu siger du, at akutte patienter kan komme døgnet rundt, men er der nogen tidspunkter I vagtskifter, at I venter?} \\
\noindent
\textit{Nej. De kommer, når det passer fra skadestuen af.} \\
\noindent


\textbf{De elektive patienter, der så er planlagte, hvordan foregår det, når der kommer en akut patient og tager vedkommendes sengeplads? Bliver de udsat?} \\
\noindent
\textit{Det er sjældent de er akut, fordi tit så er det forskellige stuer de bliver opereret på. og de planlagte elektive patienter kører jo på nogle andre stuer og kører rent dagtid. Så derfor så er det sjældent der er et overlap i forhold til de to. Det er mere nogen akutte patienter, der er derhjemme, der bliver kaldt ind til nogle operationer, hvor at der måske er noget der er endnu mere akut, der så kommer og tager deres plads.} \\
\noindent
\textbf{Okay.} \\
\noindent
\textit{Men elektive patienter de kører som regel på nogle andre stuer, så skal der være massiv overbelægning, så bliver de aflyst og så bliver de aflyst allerede måske hjemmefra.} \\
\noindent

\textbf{Så inden de ankommer, så får de en ny tid?} \\
\noindent
\textit{Ja.} \\
\noindent

\textbf{Det var også i forhold til jamen, hvor mange sengepladser har i forbeholdt til akutte i forhold til elektive, om der er en fast procentsats?} \\
\noindent
\textit{Ikke jeg er helt bekendt med i hvert fald.} \\
\noindent

\textbf{Okay. Hvad sker der så, hvis i ikke har flere sengepladser til rådighed? Du siger overbelægning, i ikke har plads til mere end 20 senge i har heroppe, hvad gør I så med patienterne, der kommer, de akutte?} \\
\noindent
\textit{Vi har jo lidt ekstra pladser. Vi er normeret til de 20 patienter, det er det vi har personale til, men vi har 22-24 antal fysiske pladser, hvor de kan være, så det fylder vi jo så op først. Men vi er jo to sengeafdelinger, så det hedder sig, at man fylder op til normeringen og det har vi antal pladser til og så kører den anden afdeling op til normeringen og de har plads til 25 ovre på den anden side af gangen på O2, så det er først når begge afsnit er fyldt helt op med normeret, så fylder vi jo ekstra og vi har begge steder et par ekstra fysiske pladser, hvor folk kan være. vi har bare ikke personale.} \\
\noindent

\textbf{Til at varetage?} \\
\noindent
\textit{Ja. Derefter så kommer de så på gangen, hvis det er, at vi ikke kan sørge for overflytning til andre steder på det tidspunkt.} \\
\noindent


\textbf{Så når du siger andre steder, så mener du?} \\
\noindent
\textit{Andre matrikler, altså vi har Farsø, og vi har Hjørring og vi har Frederikshavn.} \\
\noindent
\textbf{Okay, så i gør det ikke lokalt, i flytter dem ud på andre ortopædkirurgiske matrikler?} \\
\noindent
\textit{Okay, alt efter hvor specialet er og om der er plads de steder, men det jo primært mandag til fredag det er en mulighed. De fleste andre steder har ikke kirurgiske pladser så, hvis de akutte patienter, der kommer, hvor der er behov for operation. Det har de ikke i weekenden de andre steder, så er det jo her på Aalborg, men så forsøger vi så at flytte nogle af dem der måske er færdigbehandlet, der kan komme til videre genoptræning eller videre mobilisering til andre matrikler. } \\
\noindent


\textbf{Fint.} \\
\noindent
\textit{Så har vi også geriatriske pladser jo.} \\
\noindent


\textbf{En gang til?} \\
\noindent
\textit{Vi har sådan nogle geriatriske pladser, altså vi har jo 6 ortopædkirurgiske geriatriske pladser over på geriatriskafsnit, der har vi 6 pladser der, jeg kan ikke huske om det er 4 eller 6 pladser til ortopædkirurgiske patienter, så der har vi vores hoftepatienter der også kommer derover. Det aflaster jo så også os nogen gange.} \\
\noindent

\textbf{Det kan i også aflaste på.} \\
\noindent
\textit{Ja.} \\
\noindent

\textbf{Så i forhold til, nu siger du, at I får flere patienter end Ier normeret senge til. Og det i forhold til personalet som sagt?} \\
\noindent
\textit{Ja.} \\
\noindent


\textbf{Hvad gør i så? Nu har vi ud fra noget litteratur fundet ud af, at I tilkalder noget ekstra vagtpersonale fra vikarbureau eller hvordan foregår det? } \\
\noindent
\textit{Altså det er jo altid en vurderingssag, det jo ikke altid antallet af patienter, men også tyngden af de patienter der er fordi at, hvis det er overbelægning med 4 relativt ukomplicerede håndkirurgiske patienter, hvor de skal ind og have sat skinner i og så går de hjem og man forventer de går hjem samme dag, så er det ikke nødvendigvis man får en ekstra ind. Mens, hvis det nogen der er plejekrævende eller hvis det er rigtig mange håndpatienter også oveni sammen med de andre, så kommer der ekstra vikarer ind. } \\
\noindent


\textbf{Så det er i forhold til speciale eller hvor lang tid de skal?} \\
\noindent
\textit{ Hvad vi også forventer i forhold til, hvornår de skal hjem.} \\
\noindent


\textbf{Okay, så du vil sige generelt, hvis I har problemer på afdelingen når I har for mange patienter, så det personalemangel, der først begrænser jer før plads og udstyr eller?} \\
\noindent
\textit{De mangler jo altid udstyr, så der jo borde, hvis de er på gangen så har jo ingen borde og sådan nogle ting og de har jo ikke noget strøm og ingen skabe de kan låse ting indeni.} \\
\noindent

\textbf{ Nu snakker jeg også om i forhold til behandlingsudstyr eller genoptræningsudstyr.} \\
\noindent
\textit{Vi har det vi har, men mange de får jo hjælpemidler op i forhold til, hvis det er længere og det kommer nede fra hjælpemiddeldepotet og sådan nogle ting, så det er ikke nødvendigvis ikke, hvor mange patienter man har. } \\
\noindent

\textbf{Så nu kom du ind på tidligere, at I også får patienter fra det andet ortopædkirurgisk afsnit herinde, hvis i har en plads her og de mangler plads, men er der andet i gør, hvis i har sengepladser til rådighed? End at tage patienter fra den anden ortopædkirurgiske afdeling? } \\
\noindent
\textit{Vi skal nok få dem fyldt op. } \\
\noindent

\textbf{Okay, så i plejer at få dem fyldt op? } \\
\noindent
\textit{Ja, sådan relativt hurtigt i hvert fald, fordi så kommer de planlagte jo også. Altså jeg var her i går aftes og der var vi jo, vi er normerede til de 20, vi startede med at have 16 patienter eller 17, men da jeg gik hjem, havde vi fået patient nr. 20, så vi var op på det normerede faktisk til midnat, men jeg vidste der kom 3 patienter ind til morgen plus 2 der skulle ringes ind, så vi vidste jo, at vi ville jo komme op over det normerede.} \\
\noindent

\textbf{Så i vidste allerede, at I skulle begynde at aflaste på den anden afdeling og få udskrevet patienter?} \\
\noindent
\textit{Ja} \\
\noindent

\textbf{Okay. Nu er det i forhold til noget med, når I går ind og vurderer, hvor lang tid skal man lægge her om der noget bestemt I går ind og vurderer det ud fra, indlæggelsesvarigheden af patienterne?} \\
\noindent
\textit{ Ja det gør vi jo, men det er alt efter hvad speciale, hvad de fejler, hvad de har fået lavet.}  \\
\noindent


\textbf{Så det er ud fra diagnose og operation?} \\
\noindent
\textit{Ja og grundsygdom, altså grund diagnoser. Man kan jo godt ved nogen sige, at de er meget komplicerede. Man ved det kommer til at tage et længere indlæggelsesforløb.
Interviewer: Men der er ikke noget systematisk tjekliste i skal igennem eller er det bare vurderingssag?} \\
\noindent

\textit{Vi ved, at ca et normalt hoftepatient, der får et brud og der falder, det tager typisk omkring de 5 dage, så vi melder dem hjem til efter de 5 dage efter en operation og så kan man jo justere det bagefter jo.} \\
\noindent

\textbf{Så det er baseret på erfaring?} \\
\noindent
\textit{Ja.} \\
\noindent


\textbf{Er der nogle parametre, der vægter mere end andre? } \\
\noindent
\textit{I forhold til?} \\
\noindent

\textbf{I forhold til indlæggelsesvarigheden?} \\
\noindent
\textit{Mobiliseringsgraden inden er ret væsentligt. Hvad vi forventer af dem fordi, hvis det er en plejehjælpsbruger, der næsten ingen gangfunktion har inden, har vi ikke en forventning om at vi kan gøre det bedre. Så hvis de har været vant til at bruge lift nærmest inden,  så er det også den forventning vi har. Hvis de bor på plejehjem, kommer de også typisk hjem tidligere fordi der er fysioterapeuter koblet på plejehjemmene, så der får de den mobilisering og genoptræning på plejehjem i stedet for. } \\
\noindent


\textbf{Nu siger du det her med at komme hjem. Oplever I, når I skal sende patienter hjem, at der nogle gange opstår prop i forhold til, at patienterne skal udskrives og ældre patienter skal til plejehjem eller hvordan foregår det? } \\
\noindent
\textit{Der er sjældent, når de skal tilbage til plejehjemmet, hvis de bor på plejehjem. Problemet er tit, hvis de bor i et hjem, der simpelthen ikke er egnet til måske de hjælpemidler de skal bruge eller et hjem, der ikke kan rumme de nødvendige forandringer der er nu eller at nogen der har egentlig taget været lidt demente og de har egentlig taget kunnet klare sig, men så sker der forandringer, når de bliver indlagte og de bliver meget konfuse og meget demente. Det eskalerer nærmest deres sygdom, så de kan ikke komme hjem til sig selv. Der oplever vi tit en flaskehals fordi at så skal vi vente på en aflastningsplads til dem. Det kan godt være lidt svært at få til mange af dem.} \\
\noindent


\textbf{Så det er generelt ældre, der er problemet med at få hjem?} \\
\noindent
\textit{Ja, eller det er mere problemet, hvis de skal have en aflastningsplads. Det er tit der, der sker en flaskehals.}  \\
\noindent

\textbf{Så er der lige her til slut, er der andet du tænker, der er et problem på afdelingen i forhold til, at I måske står og mangler personale eller noget der forsinker jer i jeres arbejdsdage?} \\
\noindent
\textit{Jamen, selvfølgelig er der altid problemer nogle gange, når vi er for lidt personale, hvis der er sygdom eller noget andet, der gør at vi er for få hænder til at udføre arbejdet. Men det er nogengange den indstilling man har til det. Jeg tror, at jeg er meget positivt indstillet sådan generelt, så det er jo en udfordring, men ikke en begrænsning for at jeg kan komme igennem dagen og det stresser mig ikke nødvendigt, men det kan det jo godt gøre ved mange, hvis man konstant ligger med overbelægningsprocent, der er forhøjet. Vi ser jo også perioder, hvor der er flere der går hjem med stress pga. det. } \\ \noindent

\textbf{Så stress er en faktor på afdelingen? det er noget der påvirker jer?} \\
\noindent
\textit{ Ja.} \\
\noindent


\textbf{Nu hvor du på forhånd siger, at du vidste der ville være for mange patienter her i dag til morgen, vælger i så at tilkalde folk, altså ekstra personale på forhånd eller løber I bare hurtigere? } \\
\noindent
\textit{Vi er inde og vurdere. Vi er inde og kigge på, hvor mange er vi sat på til vagten i dag og der var egentlig taget sådan rimeligt antal personale sat på til at være i dag.
Hvis det havde været en dag, hvor der var for få personale sat på, så havde vi måske forsøgt at kalde på en. Men de havde også kaldt på en her, vi har en ekstra personale i dag.} \\
\noindent


\textbf{Okay, men er det lokalt fra afdelingen eller et vikarbureau? } \\
\noindent
\textit{ Det endte så med at være en over fra O2, fordi de så ikke havde så mange patienter, så der har vi lånt en herover. } \\
\noindent


\textbf{Så I har den der synergi, at i arbejder sammen med hinanden?} \\
\noindent
\textit{Vi forsøger.} \\
\noindent
\textbf{Ja, I forsøger.} \\
\noindent
\textit{Men altså, hvis de har rigtig mange patienter, så er vi også nogen gange ovre og hjælpe dem i både dagvagt og aftenvagter også for at forsøge at hjælpe hinanden i hvert fald.} \\
\noindent

\textbf{Det er i orden. Tak for interviewet.} \\
\noindent


\section{Interview med lægesekretær}

\textbf{Hvor længe har du arbejdet her?} \\
\noindent
\textit{Det har jeg i fire år.}

\textbf{Fire år?} \\
\noindent
\textit{Ja. Og så har jeg været i ortopædkirurgien, 7 år tidligere i min tid som lægesekretær.}


\textbf{Og må vi citere dig for interviewet?}\\
\noindent
\textit{Ja.}


\textbf{Jamen nu vil jeg spørge, hvordan planlægger I de her elektive patienter?}\\
\noindent
\textit{Vi har faste pladser. Vi har ikke så mange elektive under indlæggelse, men vi har rigtig mange elektive i dagkirurgi. I dagkirurgi har vi rigtig mange tider og  holder altid nogle enkelte ledige til, hvis der dukker noget akut op under indlæggelse. Det er kun de patienter, der er absolut nødvendigt skal opereres under indlæggelse, der kommer derind og det er ikke altid vi faktisk udnytter alle vores tider. I denne her uge har vi for eksempel ikke brugt vores pladser som vi så har været heldige med fordi der har været mange akutte patienter, så de har haft meget overbelægning eller mange patienter deroppe, så det har faktisk været heldigt i denne her uge. Men dem planlægger vi sådan, at vi som regel altid har plads til lidt ekstra. Vi booker ikke stramt op, kun i absolut nødstilfælde og det er, hvis der kommer en ekstra en på som man er nødt til at skal have plads til.}

\textbf{Er der en procentsats eller en del af sengene, der er afsat til akutte patienter?}\\
\noindent
\textit{Nej det tror jeg ikke. Ikke mig bekendt. Altså vi må operere om tirsdagen for vores vedkommende og der må vi sætte en til to patienter på afhængigt af, hvor stort et indgreb det er. Det er af hensyn til operationsgangen og deres kapacitet der. Det er egentligt ikke af hensyn til sengeafdelingen. Det er en del af vores elektiv. Det ved de, at der kan komme to, hvis det er to små operationer. Men en del af de patienter vi sender op på sengeafdelingen, selvom de er nødt til at skulle opereres under indlæggelse, så er det måske noget i forhold til anæstesi sikkerhed og opvågning. Mange af de patienter kan måske godt allerede gå hjem senere på dagen. Det er ikke nødvendigvis ensbetydende med, at de skal ligge deroppe i flere dage. Men der er også dem, der så kommer der op fordi de netop skal ligge der i nogle dage, fordi de skal træne efter operationen. Der er ikke som er 100 procent fastlagt.} 


\textbf{Når I indkalder de her elektive patienter, bliver det så gjort med kort eller lang varsel? Hvordan foregår det?}\\
\noindent
\textit{Jamen, hvis der sidder en patient henne i ambulatoriet i dag, der skal opereres under indlæggelse, så vil patienten få en tid med hjem i dag. Og hvis det ikke er noget, der haster og hvis det ikke er noget akut, så finder vi første ledige tid under de omstændigheder der nu er. Skal det være en bestemt læge, er det patienten, der siger jeg vil gerne vente til efter 15. December eller under de forhold, der nu er der. Er det sådan at patienten siger første ledige tid, jamen så vil de simpelthen få første ledige tid og det kan godt være i næste uge. Det kan også være om 14 dage, tre uger.}


\textbf{Hvis så man er nødt til at rykke den elektive patient under akutte patienter, hvordan gør man så?}\\
\noindent
\textit{Det er igen et spørgsmål om for vores område fordi vi ikke er så stramt booket op under indlæggelse. Så har vi plads til, at vi kan putte dem på måske en uge, måske 14 dage senere. Altså vi får dem på forholdsvis hurtigt og vi skal nok sørge for, at de får en tid. Vi kan også løbe ind i perioder, hvor vi har mange, der skal ligge under indlæggelse. Og så kan vi selvfølgelig godt løbe ind i kniberi med, hvad gør vi så her. Og så er det egentligt vores specialeansvarlige overlæge, der træder ind og siger, jamen så gør vi sådan her. Så er det ham, der ligesom træffer beslutningen om, hvordan håndterer vi det.} 


\textbf{Når I indkalder de her patienter, planlægger de elektive patienter, planlægger I så også efter for eksempel operationstype, hvor lang tid skal de ligge der?}\\
\noindent
\textit{Nej.}


\textbf{Er der noget du føler er relevant lige at tilføje.}
\\
\noindent
\textit{Nej, ikke andet end det her er jo kun for sengeafdelingen. Vores O6 kører jo ligesom et forløb for sig selv. Det er ambulant kirurgi. Det kører ligesom sit liv for sig selv, men der har vi også altid buffer, hvis der er nogen, der bliver aflyst. Vi prøver på at have en regel om, at den første tid bliver som regel ikke booket. Det er den vi holder tilbage, fordi det sjældent, at det er den første patient man aflyser, hvis man aflyser, så opererer man som regel de patienter, der er mødt ind i dagkirurgi. Så vælger man at aflyse dem der endnu ikke er mødt. Dem der først kommer senere på dagen. Så på den måde, hvis der er nogen, der bliver aflyst i dag og de så får en tid på mandag fra morgenstunden af, jamen så møder de ind, så bliver de opereret, så man undgår, at de bliver aflyst mere end én gang. Men hvis vi sætter dem på senere på dagen, så kan vi risikere, at de rent faktisk bliver aflyst en gang mere, hvis det skulle være. 7-9-13 og bank under bordet, det sker heldigvis ikke ret tit, men det kan jo ske på grund af sygdom.}


\textbf{Bare lige om jeg har forstået det korrekt, så snakker du noget om, at I ud fra, når I planlægger, så planlægger I om, altså ud fra om kirurgen også, om den er tilstede den dag det skal være, så det gør I ud fra erfaring og operationstype.}\\
\noindent
\textit{Det er ud fra, det er lægen selv, der har skrevet på. Lægen udfylder en operations tilmelding nu når de skriver patienter op til operation og derpå skriver han skal jeg selv operere denne her patient eller kan det være en anden der gør det. Patienten får også selv lov at vælge og sige jamen ønsker du, at det skal være den læge. Hvis ikke han har skrevet noget på, så får patienten selv lov at vælge. Jamen jeg vil gerne have, at det var den eller nej, jeg vil bare gerne have så hurtig en tid jeg kan få.}


\textbf{Okay, så det er de overvejelser I gør, når det er?}\\
\noindent
\textit{Ja, det er ligesom vores sygeplejersker, der sidder og har papirene og siger jamen vi skal have en tid, der passer ind med det her, men patienten vil bare gerne ind så hurtigt som muligt. Det kan også godt nogen gange være, at lægen skriver han gerne selv vil gøre det, men hvis der går et stykke tid og patienter siger jeg vil bare gerne have det overstået, så kan man måske godt sige okay, fint nok. Det er ikke noget, der er noget specielt kompliceret i, så det vil en anden også godt kunne gøre, det her.}