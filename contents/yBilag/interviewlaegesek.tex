\section{Sygeplejersker på sengeafsnit 01} \label{bilag01}
\textbf{Hvem er du?} \\
\noindent
\textit{Jeg hedder Mai Heilskov og jeg er sygeplejerske på O1, Aalborg sygehus syd. } \\
\noindent
\textbf{Og hvor længe har du været på afdelingen?}\\
\noindent
\textit{Snart 10 år} \\
\noindent
\textbf{Så vil vi høre om vi må citere dig for interviewet og optage?}\\
\noindent
\textit{ Ja, det må I gerne.}  \\
\noindent
\textbf{Her til at starte med så vil vi spørge om, hvor mange arbejdstimer har du om ugen?} \\
\noindent
\textit{34} \\
\noindent
\textbf{Så det er ikke fuldtids, men?} \\
\noindent
\textit{Ja, det svarer til at man har 3 dage fri på 8 uger eller sådan noget.} \\
\noindent
\textbf{Så du kører efter et 8 ugers vagtplan?} \\
\noindent
\textit{Al plan bliver kørt over de 8 uger, hvor det skal gå op i et antal timer.} \\
\noindent
\textbf{Okay, så det er 34 timers gennemsnit om ugen over 8 uger?} \\
\noindent
\textit{Ja.} \\
\noindent
\textbf{Så din vagtlængde det er 7,5 time? } \\
\noindent
\textit{8 eller 9 timers vagter.} \\
\noindent
\textbf{Okay, og det er dag og nat? Eller hvordan fungerer jeres skift?} \\
\noindent
\textit{Jeg kører dag-aften, men nogen kører dag-nat og nogle kører mere eller mindre primært nat eller aften.} \\
\noindent
\textbf{Så der er 3 vagtskift?} \\
\noindent
\textit{Der er 3 vagtskift.} \\
\noindent
\textbf{Okay, det passer med de 8 timer.} \\
\noindent
\textbf{Når du har her på en normal vagt, når du varetager patienter, hvor stor er en patient byrde? hvor mange patienter varetager du?} \\
\noindent
\textit{Hvor mange jeg har ansvaret for?} \\
\noindent
\textbf{Ja} \\
\noindent
\textit{Det er jo ikke primær pleje, så det er sådan lidt. Vi har tit gruppepleje, det betyder, at så er man jo 2 om et vis antal patienter. Men det varierer rigtig meget, hvor mange jeg har ansvaret for.} \\
\noindent
\textbf{Kan du sætte størrelse på, hvor mange I er sammen?} \\
\noindent
\textit{I dagvagt har vi måske typisk, er vi 2 om måske 6-8 patienter. sådan typisk alt efter om vi har overbelægning eller ej, men deromkring. I aftenvagten er vi jo 2 om 10-12 patienter.} \\
\noindent
\textbf{Nu siger du overbelæning, er det når i har?} \\
\noindent
\textit{Mere end de 20 patienter vi er normerede til.} \\
\noindent
\textbf{Okay, så det er når i har mere end de 20 normerede sengepladser. } \\
\noindent
\textbf{I forhold til dine pauser, hvordan forløber de? Kan I blive indkaldt i jeres pauser? eller har I påtvunget, at I skal holde pauser?} \\
\noindent
\textit{Pauserne, det er når det passer ind i forhold til arbejdsrytmen og nogen dage er der ikke pauser.} \\
\noindent
\textbf{Og man kan blive indkaldt fra sine pauser?} \\
\noindent
\textit{Ja, vi har jo telefon. Vi har sådan nogle dæktelefoner på os, så læger kan ringe til os og vi kan blive kaldt på med opkald fra gangen, fra patienter og sådan noget.}
\textbf{Så i er disponible?} \\
\noindent
\textit{Vi er disponible.} \\
\noindent
\textbf{Fra I møder til I går hjem?} \\
\noindent
\textit{Ja, det er vi.}  \\
\noindent
\textbf{Når I modtager patienter på afsnittet, de elektive patienter, skemalægges det ift. de indlægges på et bestemt tidspunkt eller?
Informant: De planlagte operationer?} \\
\noindent
\textit{Ja, de planlagte.} \\
\noindent
\textit{De kommer typisk om morgenen, fordi de planlagte operationer er jo i dagstid. så det kun, hvis det er den første på programmet, og de bor langvejsfra, så kan de komme dagen før nogen gange. så kommer de måske typisk kl 8 om aftenen.} \\
\noindent
\textbf{Så de bliver indskrevet måske et halvt døgn før indgrebet?} \\
\noindent
\textit{Ja.} \\
\noindent
\textit{Heroppe nogle gange og nogle gange på patienthotellet, det er sådan nyt, at der er nogen der går der ned. Og så kommer de så herop om morgenen, nogle gange kommer de her alligevel.} \\
\noindent
\textbf{Og de akutte patienter de indskrives døgnet rundt?} \\
\noindent
\textit{ Ja, det gør de.} \\
\noindent
\textbf{Er der et bestemt tidsrum I udskriver patienter på? Er der et tidspunkt på døgnet I ikke udskriver patienter på?} \\
\noindent
\textit{Langt de fleste udskrivelser foregår måske mellem 12 og 16, medmindre de er planlagt dagen før, så kan det være om formiddagen. Men det er jo typisk, fordi der først er stuegang kl 9, det er først der lægerne kommer, så 9, kvart over 9. Så før at man får kørt stuegang igennem og de praktiske ting er klar før personen kan komme hjem, så det typisk over middag før de kan. } \\
\noindent
\textbf{Hvor ofte vil du sige, at man vurderer patienternes tilstand, hvornår de kan sendes hjem?}  \\
\noindent
\textit{Altså det vurderer vi jo dagligt i forhold til, hvad vi tænker er realistisk i forhold til hvornår de skal hjem. og tager jo egentligt relativt hurtigt i forløbet af patienterne stilling til, hvad de tænker, og så sender vi jo, hvis de skal have hjælp, plejeforløbsplan afsted. sådan at de kan udskrives.} \\
\noindent
\textbf{Så det vurderes i starten, hvor lang tid kommer de til at ligge og så når det nærmer sig det tidsrum de skulle ligge, så vurderer man så igen?} \\
\noindent
\textit{Ja.} \\
\noindent
\textbf{Fint. Så er det om der er nogen tidspunkter I ikke indlægger patienter. Nu siger du, at akutte patienter kan komme døgnet rundt, men er der nogen tidspunkter I vagtskifter, at I venter?} \\
\noindent
\textit{Nej. De kommer, når det passer fra skadestuen af.} \\
\noindent
\textbf{De elektive patienter, der så er planlagte, hvordan foregår det, når der kommer en akut patient og tager vedkommendes sengeplads? Bliver de udsat?} \\
\noindent
\textit{Det er sjældent de er akut, fordi tit så er det forskellige stuer de bliver opereret på. og de planlagte elektive patienter kører jo på nogle andre stuer og kører rent dagtid. Så derfor så er det sjældent der er et overlap i forhold til de to. Det er mere nogen akutte patienter, der er derhjemme, der bliver kaldt ind til nogle operationer, hvor at der måske er noget der er endnu mere akut, der så kommer og tager deres plads.} \\
\noindent
\textbf{Okay.} \\
\noindent
\textit{Men elektive patienter de kører som regel på nogle andre stuer, så skal der være massiv overbelægning, så bliver de aflyst og så bliver de aflyst allerede måske hjemmefra.} \\
\noindent
\textbf{Så inden de ankommer, så får de en ny tid?} \\
\noindent
\textit{Ja.} \\
\noindent
\textbf{Det var også i forhold til jamen, hvor mange sengepladser har i forbeholdt til akutte i forhold til elektive, om der er en fast procentsats?} \\
\noindent
\textit{Ikke jeg er helt bekendt med i hvert fald.} \\
\noindent
\textbf{Okay. Hvad sker der så, hvis i ikke har flere sengepladser til rådighed? Du siger overbelægning, i ikke har plads til mere end 20 senge i har heroppe, hvad gør I så med patienterne, der kommer, de akutte?} \\
\noindent
\textit{Vi har jo lidt ekstra pladser. Vi er normeret til de 20 patienter, det er det vi har personale til, men vi har 22-24 antal fysiske pladser, hvor de kan være, så det fylder vi jo så op først. Men vi er jo to sengeafdelinger, så det hedder sig, at man fylder op til normeringen og det har vi antal pladser til og så kører den anden afdeling op til normeringen og de har plads til 25 ovre på den anden side af gangen på O2, så det er først når begge afsnit er fyldt helt op med normeret, så fylder vi jo ekstra og vi har begge steder et par ekstra fysiske pladser, hvor folk kan være. vi har bare ikke personale.} \\
\noindent
\textbf{Til at varetage?} \\
\noindent
\textit{Ja. Derefter så kommer de så på gangen, hvis det er, at vi ikke kan sørge for overflytning til andre steder på det tidspunkt.} \\
\noindent
\textbf{Så når du siger andre steder, så mener du?} \\
\noindent
\textit{Andre matrikler, altså vi har Farsø, og vi har Hjørring og vi har Frederikshavn.} \\
\noindent
\textbf{Okay, så i gør det ikke lokalt, i flytter dem ud på andre ortopædkirurgiske matrikler?} \\
\noindent
\textit{Okay, alt efter hvor specialet er og om der er plads de steder, men det jo primært mandag til fredag det er en mulighed. De fleste andre steder har ikke kirurgiske pladser så, hvis de akutte patienter, der kommer, hvor der er behov for operation. Det har de ikke i weekenden de andre steder, så er det jo her på Aalborg, men så forsøger vi så at flytte nogle af dem der måske er færdigbehandlet, der kan komme til videre genoptræning eller videre mobilisering til andre matrikler. } \\
\noindent
\textbf{Fint.} \\
\noindent
\textit{Så har vi også geriatriske pladser jo.} \\
\noindent
\textbf{En gang til?} \\
\noindent
\textit{Vi har sådan nogle geriatriske pladser, altså vi har jo 6 ortopædkirurgiske geriatriske pladser over på geriatriskafsnit, der har vi 6 pladser der, jeg kan ikke huske om det er 4 eller 6 pladser til ortopædkirurgiske patienter, så der har vi vores hoftepatienter der også kommer derover. Det aflaster jo så også os nogen gange.} \\
\noindent
\textbf{Det kan i også aflaste på.} \\
\noindent
\textit{Ja.} \\
\noindent
\textbf{Så i forhold til, nu siger du, at I får flere patienter end Ier normeret senge til. Og det i forhold til personalet som sagt?} \\
\noindent
\textit{Ja.} \\
\noindent
\textbf{Hvad gør i så? Nu har vi ud fra noget litteratur fundet ud af, at I tilkalder noget ekstra vagtpersonale fra vikarbureau eller hvordan foregår det? } \\
\noindent
\textit{Altså det er jo altid en vurderingssag, det jo ikke altid antallet af patienter, men også tyngden af de patienter der er fordi at, hvis det er overbelægning med 4 relativt ukomplicerede håndkirurgiske patienter, hvor de skal ind og have sat skinner i og så går de hjem og man forventer de går hjem samme dag, så er det ikke nødvendigvis man får en ekstra ind. Mens, hvis det nogen der er plejekrævende eller hvis det er rigtig mange håndpatienter også oveni sammen med de andre, så kommer der ekstra vikarer ind. }\\
\noindent
\textbf{Så det er i forhold til speciale eller hvor lang tid de skal?} \\
\noindent
\textit{ Hvad vi også forventer i forhold til, hvornår de skal hjem.} \\
\noindent
\textbf{Okay, så du vil sige generelt, hvis I har problemer på afdelingen når I har for mange patienter, så det personalemangel, der først begrænser jer før plads og udstyr eller?} \\
\noindent
\textit{De mangler jo altid udstyr, så der jo borde, hvis de er på gangen så har jo ingen borde og sådan nogle ting og de har jo ikke noget strøm og ingen skabe de kan låse ting indeni.} \\
\noindent
\textbf{ Nu snakker jeg også om i forhold til behandlingsudstyr eller genoptræningsudstyr.} \\
\noindent
\textit{Vi har det vi har, men mange de får jo hjælpemidler op i forhold til, hvis det er længere og det kommer nede fra hjælpemiddeldepotet og sådan nogle ting, så det er ikke nødvendigvis ikke, hvor mange patienter man har. } \\
\noindent
\textbf{Så nu kom du ind på tidligere, at I også får patienter fra det andet ortopædkirurgisk afsnit herinde, hvis i har en plads her og de mangler plads, men er der andet i gør, hvis i har sengepladser til rådighed? End at tage patienter fra den anden ortopædkirurgiske afdeling? } \\
\noindent
\textit{Vi skal nok få dem fyldt op. } \\
\noindent
\textbf{Okay, så i plejer at få dem fyldt op? } \\
\noindent
\textit{Ja, sådan relativt hurtigt i hvert fald, fordi så kommer de planlagte jo også. Altså jeg var her i går aftes og der var vi jo, vi er normerede til de 20, vi startede med at have 16 patienter eller 17, men da jeg gik hjem, havde vi fået patient nr. 20, så vi var op på det normerede faktisk til midnat, men jeg vidste der kom 3 patienter ind til morgen plus 2 der skulle ringes ind, så vi vidste jo, at vi ville jo komme op over det normerede.} \\
\noindent
\textbf{Så i vidste allerede, at I skulle begynde at aflaste på den anden afdeling og få udskrevet patienter?} \\
\noindent
\textit{Ja} \\
\noindent
\textbf{Okay. Nu er det i forhold til noget med, når I går ind og vurderer, hvor lang tid skal man lægge her om der noget bestemt I går ind og vurderer det ud fra, indlæggelsesvarigheden af patienterne?} \\
\noindent
\textit{ Ja det gør vi jo, men det er alt efter hvad speciale, hvad de fejler, hvad de har fået lavet.}  \\
\noindent
\textbf{Så det er ud fra diagnose og operation?} \\
\noindent
\textit{Ja og grundsygdom, altså grund diagnoser. Man kan jo godt ved nogen sige, at de er meget komplicerede. Man ved det kommer til at tage et længere indlæggelsesforløb.
Interviewer: Men der er ikke noget systematisk tjekliste i skal igennem eller er det bare vurderingssag?} \\
\noindent
\textit{Vi ved, at ca et normalt hoftepatient, der får et brud og der falder, det tager typisk omkring de 5 dage, så vi melder dem hjem til efter de 5 dage efter en operation og så kan man jo justere det bagefter jo.} \\
\noindent
\textbf{Så det er baseret på erfaring?} \\
\noindent
\textit{Ja.} \\
\noindent
\textbf{Er der nogle parametre, der vægter mere end andre? } \\
\noindent
\textit{I forhold til?} \\
\noindent
\textbf{I forhold til indlæggelsesvarigheden?} \\
\noindent
\textit{Mobiliseringsgraden inden er ret væsentligt. Hvad vi forventer af dem fordi, hvis det er en plejehjælpsbruger, der næsten ingen gangfunktion har inden, har vi ikke en forventning om at vi kan gøre det bedre. Så hvis de har været vant til at bruge lift nærmest inden,  så er det også den forventning vi har. Hvis de bor på plejehjem, kommer de også typisk hjem tidligere fordi der er fysioterapeuter koblet på plejehjemmene, så der får de den mobilisering og genoptræning på plejehjem i stedet for. } \\
\noindent
\textbf{Nu siger du det her med at komme hjem. Oplever I, når I skal sende patienter hjem, at der nogle gange opstår prop i forhold til, at patienterne skal udskrives og ældre patienter skal til plejehjem eller hvordan foregår det? } \\
\noindent
\textit{Der er sjældent, når de skal tilbage til plejehjemmet, hvis de bor på plejehjem. Problemet er tit, hvis de bor i et hjem, der simpelthen ikke er egnet til måske de hjælpemidler de skal bruge eller et hjem, der ikke kan rumme de nødvendige forandringer der er nu eller at nogen der har egentlig taget været lidt demente og de har egentlig taget kunnet klare sig, men så sker der forandringer, når de bliver indlagte og de bliver meget konfuse og meget demente. Det eskalerer nærmest deres sygdom, så de kan ikke komme hjem til sig selv. Der oplever vi tit en flaskehals fordi at så skal vi vente på en aflastningsplads til dem. Det kan godt være lidt svært at få til mange af dem.} \\
\noindent
\textbf{Så det er generelt ældre, der er problemet med at få hjem?} \\
\noindent
\textit{Ja, eller det er mere problemet, hvis de skal have en aflastningsplads. Det er tit der, der sker en flaskehals.}  \\
\noindent
\textbf{Så er der lige her til slut, er der andet du tænker, der er et problem på afdelingen i forhold til, at I måske står og mangler personale eller noget der forsinker jer i jeres arbejdsdage?} \\
\noindent
\textit{Jamen, selvfølgelig er der altid problemer nogle gange, når vi er for lidt personale, hvis der er sygdom eller noget andet, der gør at vi er for få hænder til at udføre arbejdet. Men det er nogengange den indstilling man har til det. Jeg tror, at jeg er meget positivt indstillet sådan generelt, så det er jo en udfordring, men ikke en begrænsning for at jeg kan komme igennem dagen og det stresser mig ikke nødvendigt, men det kan det jo godt gøre ved mange, hvis man konstant ligger med overbelægningsprocent, der er forhøjet. Vi ser jo også perioder, hvor der er flere der går hjem med stress pga. det. } \\ \noindent
\textbf{Så stress er en faktor på afdelingen? det er noget der påvirker jer?} \\
\noindent
\textit{ Ja.} \\
\noindent
\textbf{Nu hvor du på forhånd siger, at du vidste der ville være for mange patienter her i dag til morgen, vælger i så at tilkalde folk, altså ekstra personale på forhånd eller løber I bare hurtigere? } \\
\noindent
\textit{Vi er inde og vurdere. Vi er inde og kigge på, hvor mange er vi sat på til vagten i dag og der var egentlig taget sådan rimeligt antal personale sat på til at være i dag.
Hvis det havde været en dag, hvor der var for få personale sat på, så havde vi måske forsøgt at kalde på en. Men de havde også kaldt på en her, vi har en ekstra personale i dag.} \\
\noindent
\textbf{Okay, men er det lokalt fra afdelingen eller et vikarbureau? } \\
\noindent
\textit{ Det endte så med at være en over fra O2, fordi de så ikke havde så mange patienter, så der har vi lånt en herover. } \\
\noindent
\textbf{Så I har den der synergi, at i arbejder sammen med hinanden?} \\
\noindent
\textit{Vi forsøger.} \\
\noindent
\textbf{Ja, I forsøger.} \\
\noindent
\textit{Men altså, hvis de har rigtig mange patienter, så er vi også nogen gange ovre og hjælpe dem i både dagvagt og aftenvagter også for at forsøge at hjælpe hinanden i hvert fald.} \\
\noindent
\textbf{Det er i orden. Tak for interviewet.} \\


\subsection{Sygeplejerske på sengeafsnit 02} \label{bilagO2}
\textbf{Til at starte med vil vi så spørge, hvem er du?}\\
\noindent
\textit{Jeg hedder Lars og er almindelig sygeplejerske på gulvet.}\\
\noindent
\textbf{Hvor længe har du arbejdet på denne her afdeling?} \\
\noindent
\textit{$28$ år.}\\
\noindent
\textbf{Må vi citere dig for det her interview?}\\
\noindent
\textit{Ja det må I godt.}\\
\noindent
\textbf{Til at starte med at høre, hvor mange arbejdstimer har du gennemsnitligt om ugen?}\\
\noindent
\textit{Almindelig fuldtidsstilling, $37$ timer.}\\
\noindent
\textbf{Hvor lang tid er det fordelt over ift. om det er gennemsnitligt for en måned eller om det er gennemsnit hver uge?}\\
\noindent
\textit{Vi har en arbejdsplan, der hedder otte uger i alt, hvor jeg så veksler mellem dagvagter og aftenvagter.}\\
\noindent
\textbf{Har i så vagtskifte, når du siger dag og nat. Er der nogle bestemte tidspunkter de her vagtskifte finder sted?}\\
\noindent
\textit{Ja, på alle mine hverdage, dagvagter, arbejder jeg fra $7$ til $15$:$45$, og så møder jeg aftenvagten ind lidt tidligere. De møder ind $15$:$30$, så de har et kvarter til at sætte sig ind i de patienter de skal, og imens de gør det, skal jeg passe klokker.}\\
\noindent
\textbf{Så I får ligesom aflastet og sagt, at det er de her patienter I skal varetage og får fordelt viden.}\\
\noindent
\textit{Ja, men det er meningen, at de skal sætte sig og læse. Og jeg skal kun sige noget, hvis der er noget jeg lige skal af med, ellers skal de læse sig til det. Plus jeg skal tage klokkerne ude på gangen.}\\
\noindent
\textbf{Hvor mange patienter varetager du på en normal vagt? Har I et bestemt antal patienter, der er tildelt?}\\
\noindent
\textit{Nej, det er forskelligt, fordi de veksler meget i sværhedsgraden eller hvordan vi passer dem og selvfølgelig, hvad de fejler - to til fire til seks patienter.}\\
\textbf{Så det er to til seks på området.}\\
\noindent
\textit{Ja.}\\
\noindent
\textbf{Så kommer der noget ift. pauser. Nu har vi læst, at I har nogle pauser, hvis I har de her arbdejdsdage, om de er betalte pause, altså om I kan blive indkaldt I jeres pauser? eller hvordan det foregår?}\\
\noindent
\textit{Det ved jeg faktisk ikke, altså jeg bliver altid indkaldt, nu går vi med de her telefoner.  Så hvis lægen vil have fat i mig, så ringer han, fordi vi har skrevet os på tavler dernede med, at jeg har de patienter. Så har jeg mit navn og denne her telefon, så kan de ringe efter mig, hvis det er, eller fysioterapeuten, eller min kollega, hvis der er nogen. Så der er ikke sådan faste pauser. Man kan altid finde en. I hvert fald ikke mere, det kan godt være, at det har været sådan. Jeg tror også godt man må sige, man går til pause, men kutymen er, at man rejser sig, hvis der er brug for det. Plus det er også vores kloksystem, og det vil sige, jeg kan se, hvis jeg sidder ude og får kaffe, så ringer telefonen eller også, hvis jeg ikke har min telefon, så kan jeg se på en display, om det er min patient, der ringer både på stuen, men også ude på toilettet. Kutymen er, at hvis vi er et par stykker om de samme patienter. Èn af os tager klokken. Det er ikke meningen, at de andre skal tage den.}\\
\noindent
\textbf{Så patienterne er fordelt og tildelt mellem jer?}
\noindent
\textit{Ja.}
\noindent
\textbf{Så er det lidt ift., når I modtager de her patienter. Nu har vi snakket lidt med sekretariatet ift., hvordan I modatager de her patienter. Er det et bestemt tidsrum, du ser, at I modtager de her patienter? Nu snakker vi ift. de elektive patienter dem, der er planlagt om der er et bestemt tidsrum?}\\
\noindent
\textit{Det er der sådan set ikke helt, fordi vi får også afhængig af, hvad dage det er, patienter om eftermiddagen, der skal gøres klar til næste dag. Men altså indkaldte patienter møder om morgenen typisk kl $7$. Det er der også en sat af til at tage imod, så de bliver taget fra ude foran ved sekretæren, også kommer de ind på deres plads og så stiller man de spørgsmål, der er relevante og så sørger man for, at de er klædt om og parat til operationen, hvis de skal ned allerede ved 8-tiden eller lidt senere.}\\
\noindent
\textbf{Så i modtager dem om morgen og gør dem klar til i løbet af dagen?}\\
\noindent
\textit{Ja, det gør vi ved de fleste patienter.}\\
\noindent
\textbf{Så er det ift. de akutte patienter, det er døgnet rundt I modtager dem?}\\
\noindent
\textit{Ja, her på afdelingen.}\\
\noindent
\textbf{Ift. udskrivelse af patienter, foregår det et bestemt tidsrum på dagen, hvor man siger vi udskriver patienter fra kl. $10$ til kl. $14$, ellers udskriver vi ikke eller hvordan foregår det?}\\
\noindent
\textit{Det er også meget forskelligt, fordi dels, så kommer de fra kæmpe optagområder og nogen kommer endnu længere væk fra, så tit er det transporten, der nogengange bestemmer, hvornår vedkommende bliver hentet og alt afhængig af, hvem der henter dem. Hvis det er Falck og de skal hjem eller hvis de skal til et andet sygehus, så har de et tidsrum på to timer eller fire timer. Hvis du bestiller en taxa, så kan du næsten få et klokketidspunkt indenfor fem minutter. Og det synes jeg faktisk, at de er gode til at overholde.}\\ 
\noindent
\textbf{Så I udskriver løbende som der er behov for det?}\\
\noindent
\textit{Ja.}\\
\noindent
\textbf{Så ift., når I vurderer om patienter kan udskrives, hvor ofte gøres det? Er det ift., når der er stuegang eller er det flere gange i løbet af dagen eller hvordan vurderes det?}\\
\noindent
\textit{Nej, altså vi har jo et. Det er jo meget forskelligt for vores patienter, fordi vi har små børn og vi har unge mennesker og ældre mennesker og så har vi de rigtig gamle. De unge mennesker eller børn kan godt have en operation, hvor man skal have fat i hjemmesygeplejersken til at source dem, og det skal være planlagt i god tid. Også de andre, der aftaler man typisk over computeren om, hvordan man udskriver dem, og hvordan man sender dem hjem.}\\
\noindent
\textbf{Hvor ofte vurderer man så, hvornår de skal hjem og hvornår de er klar. Er det en dialog med patienten eller?}\\
\noindent
\textit{Ja, men typisk skal hjemmeplejeren også være blandet ind i det, og de skal komme til dem, så vil de helst have de ikke kommer for sent, fordi der er jo også forskel på deres dagvagt, hvor de er flest på arbejde. Og så kommer man over i en aftenvagt, hvor de ikke er ret mange på arbejde. Så det gælder om at få dem hjem i dagstiden. Det er meget vigtigt.}\\
\noindent
\textbf{Okay. Så ift. om der er nogle tidspunkter, hvor I ikke længere udskriver patienter og indlægger patienter f.eks. om natten?}\\
\noindent
\textit{Ikke indlægger, fordi vi er en akut afdeling. Der tager vi alt, hvad der kommer. Og udskriver, vi kan godt have en trafikulykke, hvor vi tror, at der er en hel masse, men hvor det hele bliver afviklet eller det hele bliver godkendt, at patienten ikke fejler noget. Og det kan typisk være en knallertulykke eller en bilulykke, hvor man kommer ind og kommer til observation. Det kan være om eftermiddagen eller det kan også være først på aftenen, hvor lægen kommer op og vurderer patienten og siger du fejler ikke noget og du må have lov til at komme hjem, men det er deres afgørelse.}\\
\noindent
\textbf{Nu siger du lægen? Er det læger, der vurderer patienterne, der kommer ind akut eller er det en kirurg?}\\
\noindent
\textit{Altså jeg har kun kirurger på min afdeling, og jeg må ikke som sygeplejerske sende nogen hjem på eget initiativ.}\\
\noindent
\textbf{Så de skal have været forbi?}\\
\noindent
\textit{En læge skal have godkendt og jeg skal i kontakt med en læge på en eller anden måde for at spørge om vedkommende må komme hjem, og det kan også være et barn med en brækket arm, der er opereret i løbet af formiddagen eller eftermiddagen, som hvis de ellers har det godt med smerter og ingen kvalme, så må man gerne have lov til at sende dem hjem om aftenen, hvis lægen siger god for det og det kan godt være pr. telefonen.}\\
\noindent
\textbf{Nu kommer vi ind på noget ift. sengepladser. Har I nogle sengepladser som I altid forbeholder til akutte patienter ift., når I fordeler sengene?}\\
\noindent
\textit{ Nej, det kan vi ikke mere, fordi vi har rimelig mange patienter til daglig, og vi er faktisk også begyndt og lade være med tænke på, at der ligger to eller tre mænd på en stue eller kvinder. Nu bruger vi simpelthen bare de pladser vi har, så vi ikke skal flytte rundt. En kvinde kan godt overnatte sammen med to mænd fordi vi har forhæng for.}\\
\noindent
\textbf{Så der er noget diskretion?}\\
\noindent
\textit{Ja. Og så er der nogle, der bliver flyttet om næste dag.}\\
\noindent
\textbf{Nu kommer vi nemlig ind på det her med, hvis der ikke er sengepladser nok på de her værelser, og du siger, at I godt kan finde på at placere dem bare inde på stuerne. Er I nødt til at have patienter liggende på gangen nogen gange? Nu det ikke natten over, nu det bare i løbet af dagen?}\\
\noindent
\textit{Ja, det har vi gjort tidligere, men brandvæsnet er efter os. Vi må ikke. For nogle år tilbage der var vi fuldstændig fyldt op. Også alle vores patienter har jo på en eller anden måde næsten brug for et hjælpemiddel af en slags. Det kan være krykker, de fylder ingenting, men gangstativ, kørestole, bækkenstole, hvad ved jeg. Alt fylder og det hele det kom herop til os, så det står heroppe til den enkelte patient med navn på. Det stod engang ude på gangen, det må det ikke for brandtilsynet. Så kom vi det ind på stuerne, så kom vi de ekstra patienter, du snakker om, de akutte, de blev stående ude på gangen. De må det ikke. Så kørte vi dem ind, så siger brandtilsynet, det er patienternes sikkerhed, så der skal vi køre patienterne ind på stuerne, hvis der er brand f.eks. og hjælpemidlerne må vi simpelthen bare placere i et andet rum. Det er ligegyldigt hvad, der må ikke være noget på gangene. Og det gjorde så, at vi fra, for et par år siden, hvor vi normalt har fire sengestue, sagde så vil vi kun have tre sengestuer for at have tre patienter på stuen, men så have et hjørne til alle hjælpemidlerne. Det kan vi sådan nogenlunde holde.}\\
\noindent
\textbf{Så I har valgt at sige nu tager vi det her udstyr og flytter det ind på stuen?}\\
\noindent
\textit{Ja, og hvis nummer fire patienter kommer, så får vedkommende pladsen og så skal vi finde et andet sted til udstyret.}\\
\noindent
\textbf{Jamen så er det lidt ift., hvis I så får for mange patienter, altså akutte patienter, og I ikke kan placere dem på de her stuer, da de er fyldt op. Hvad gør I så? Nu har vi læst, at der er trådt nogle regler i kraft med, at de bliver flyttet ned til andre stuer og andre afdelinger og om du kan bekræfte det?}\\
\noindent
\textit{Ja, det bliver så bare styret et andet sted fra. Det bliver styret fra AMA, Akut medicinsk modtagerafsnit. Der sidder nogle koordinatorer, som skal sørge for, at deres patienter der ovre fra eller fra modtagelsen, det medicinske regi, at der er plads til dem dvs., hvis de medicinske afsnit ikke har pladser nok, så kan vi risikere, at skal have en medicinsk patient eller to og det sørger den koordinator for.} \\
\noindent
\textbf{Så det er ikke jer, der styrer de patienter, og hvordan de flyttes?}\\
\noindent
\textit{Nej, vi skal melde tilbage. Jeg ved ikke om vi skal i dagligdagen, men i weekenderne, der skal vi melde tilbage. Jeg tror det er inden kl. 11 ellers så er det omkring kl. $9$ til koordinatoren og sige, at vi har to ledige pladser ellers forventer vi to ledige pladser lidt senere over middagen eller sådan noget, også har de råderum over dem i princippet.}\\
\noindent
\textbf{Ja, indtil I får en akut patient, der overtager.}\\
\noindent
\textit{Ja, så finder vi så ud af, hvad vi skal gøre. Det er lige trådt i kraft.}\\
\noindent
\textbf{Okay. Så, hvis I har nogle af de her ekstra patienter ift. hverdage, så bliver I sat på arbejde til at udfylde stuen eller udfylde afdelingen ift., hvor meget personale, der møder op eller hvordan fungerer det?}\\
\noindent
\textit{Nej, altså der jo planlagt, hvem der møder ind på den der otte ugers plan, så de skal bare møde ind, og så skal vi fordele patienterne, når vi kommer om morgenen. Og det har vi så gjort ved, at vi har typisk fire fem specialer, og de specialer har vi valgt os ind på af egen interesse, så jeg har, i en lang periode, så skal jeg passe rygpatienter primært. Men det kan godt være, at jeg skal passe børn, og det kan også være jeg skal passe noget andet eller noget tredje eller medicinske patienter. Det er forskelligt fra dag til dag. Men vi prøver at koncentrere os om dem vi har valgt os ind på.}\\
\noindent
\textbf{Ja, fordi det lidt ift., hvis der kommer for mange patienter og I ikke kan varetage dem, så har vi læst, at I har et vikarbureau I kontakter og får noget ekstra personale ind, og det kan du bekræfte, at det er sådan det fungerer?} \\
\noindent
\textit{Ja sådan set, hvis vi er i nød, så må vi ringe.}\\
\noindent
\textbf{Ja også ift. overarbejde om det er noget I bliver meget påvirket af altså, at I bliver nødt til at blive ekstra, når I for eksempel har de her vagtskifte, at I bliver nødt til at blive ekstra for at hjælpe det nu kommende vagthold?}\\
\noindent
\textit{Nej, det synes jeg ikke er noget problem. Det er meget sjældent.}\\
\noindent
\textbf{Okay, så det er sjældent det sker.}\\
\noindent
\textit{Ja, fordi så er det mig det hænger på eller de to andre, fordi vi er der jo til at lappe over. De resten er jo taget hjem, så der er kun tre at trække på, hvis der er nogle som ikke møder op eller vi har overset noget, eller der er et eller andet akut.}\\
\noindent
\textbf{Nu har vi har arbejdet med det her med, at I nogle gange har for mange patienter eller for lidt personale på arbejdet. Hvad er det, der først begrænser jer, er det, at I mangler plads på afdelingen eller er det, at I ikke har nok personale eller hvornår er det de her stress situationer opstår?}\\
\noindent
\textit{Altså det er sjældent vi har for mange patienter. Vi kan godt have mange patienter, men det behøver altså ikke være for mange, men det er nærmere, at vi ikke er nok personale til dem vi har af patienter, fordi mange af vores patienter kræver meget. Altså og kræver meget der mener jeg, at det tager lang tid at gøre ting færdige.}\\ 
\noindent
\textbf{Ja, altså passe og plejer?}\\
\noindent
\textit{Ja, passe og pleje og sørge for at tingene er i orden som det nu skal være, hvis det er helt rigtigt.}\\
\noindent
\textbf{Så det er ikke det fysiske, at det er pladsen heroppe, der mangler og at det er udstyret, men det er mere personalemangel nogen gange ift. opgavebyrden?}\\
\noindent
\textit{Ja, til opgavebyrden. Det er nok det rigtige ord. Det er fordi der mange, lige på min afdeling, at vi har nogle, hvor ting vitterligt tager lang tid og det bliver værre og værre, og der kommer flere og flere af den slags patienter ind fordi folk bliver ældre.}\\
\noindent
\textbf{Og det du siger er rygafdeling eller rygpatienter.}\\
\noindent
\textit{Ja, men vi har også mange med sår. Også har de ældre mennesker kommer typisk med andre skavanker, hvor det også tit er det vi skal tage os af, fordi det kan være sukkersyge eller dårlig blodomløb, hvor de kommer ind med et lille sår på en tå som vi skal tage os af.}\\
\noindent
\textbf{Så, når I har nogle elektive patienter og nogle akutte patienter, og når de her akutte patienter kommer og jeres arbejdsbyrde bliver større, løber I så bare hurtigere og gøre jeres arbejde hurtigere?}\\
\noindent
\textit{Ja.}\\
\noindent
\textbf{Udskyder I så de her elektive patienter og siger, at vi kan først operere jer i morgen eller hvordan fungerer det?}\\
\noindent
\textit{Ja, det kan sagtens tænkes. Men det er ikke os som sygeplejerske eller social og sundhedsassistent, der bestemmer det. Det er rent lægeligt administrativt.}\\
\noindent
\textbf{Okay, det er administrativt?}\\
\noindent
\textit{Ja. Man kan godt sige, at vi er spændt for, sådan rent teknisk eller administrativt, så kører det op på et lidt andet plan.}\\
\noindent
\textbf{Så er det ift., når I får de her patienter ind, vurderer I, hvor lang tid skal de ligge?}\\
\noindent
\textit{Nej, lad os sige kun rygpatienterne. Der vil jeg sige der har vi, hvis det er planlagt, så har vi, at det tager en uge f.eks., men nogle gange der går lidt længere tid, fordi der er ting der ikke lige går efter kalenderen, men ellers passer det nogenlunde. Men vi har mange sårpatienter, hvor det går langt over tiden fordi vi efterhånden er så specialiserede lægeligt og fordi vores afdeling er en sårafdeling så får vi alt, altså så får vi alle de patienter, der fejler noget med deres sår i en eller anden forbindelse.}\\
\noindent
\textbf{I forbindelse med vores projekt, går vi ind og kigger på om der er nogle parametre I går ind og vurderer. Nu siger du, vi har, hvis der er noget med ryggen, så ved I, at det tager en uge. Men er der noget andet f.eks. et skema I skal gå efter, hvor der er nogen ting I skal krydse af altså, hvor gammel er patienten er eller?}\\
\noindent
\textit{Nej, lige specielt rygsektoren har jo gjort sig god ved, at det er så systematiseret, at det er de samme kategorier end en skæv ryg på en ung pige. Der kommer næsten en hver $14$. dag, og de bliver jo opereret på samme måde og bliver passet på samme måde af fysioterapeuterne som kommer med det samme, så vi er hele tiden i gang med dem, så det forløb er planlagt til en uge.}\\
\noindent
\textbf{Okay, så det er vurdering ud fra erfaring simpelthen?}\\
\noindent
\textit{Ja, og ligesådan, vi har mange cancerpatienter, der kommer, hvor de pludseligt ikke kan mærke deres ben,  tværsnitssyndrom hedder det. De bliver også opereret sådan og sådan. Forløbet er sådan og sådan. De  skal op, de skal i gang og de skal holdes vedlige. Vi skal ikke presse dem, men vi skal sørge for, at de holder sig i gang.}\\
\noindent
\textbf{De skal være aktiverede?}\\
\noinden
\textit{Ja, hvorimod folk, der bliver amputeret, det er en anden kategori patienter, fordi de har tit dårligt blodomløb, sukkersyge, dårligt syn. Altså de er ikke så nemme at køre på den måde. Der skrider det hele.}\\
\noindent
\textbf{Så der er forskellige parametre indenfor hver kategori?}\\
\noindent
\textit{Ja meget.}\\
\noindent
\textbf{Og det er baseret på erfaring og ikke noget dokumenteret system?}\\
\noindent
\textit{Nej, det har jeg i hvert fald ikke styr på. Men rygsektoren det kører på de firkantede kasser og det kører bare.}\\
\noindent
\textbf{Så er det kun her til sidst om der er noget du har lyst til at tilføje eller noget du har lyst til at fortælle ift. jeres arbejde heroppe?}\\
\noindent
\textit{Nej, men det nye er, at man skriver til kommunen med udskrivelser og det gør man i så god tid som muligt, også kan man hele tiden rette på computeren, hvad man har skrevet eller hvad ens kollega har skrevet, så de hele tiden er opdateret.}\\
\noindent
\textbf{Så det er lidt det der nogengange begrænser jer heroppe, at det er svært at komme af med patienten bagefter eller?}\\
\noindent
\textit{Nej, hvis man er ude i god tid og får det skrevet ordentligt, og det ikke fordi, at man skal skrive så meget fordi det også blevet sådan et kasse system. Man skal bare huske det. Der er nogle gange vi glemmer det.}\\
\noindent
\textbf{Også, når der kommer akutte patienter?}\\
\noindent
\textit{Ja, men det er alligevel, at man skal skrive til kommunen, at man gør det og så nogle gange, hvis vi har lidt travlt, og man skal skrive det dagen før inden kl. $12$, tror jeg. Og hvis jeg først skal ind nu og skrive det nu, altså nu er kl. $13$. Se så skal jeg beholde patienten en dag mere.}\\
\noindent
\textbf{Så det er på den måde det foregår?}\\
\noindent
\textit{Ja, men der er så noget jeg ikke har helt styr på endnu, men der er noget, som hedder udskrivningsenheden. Jeg ved ikke om de sidder og kigger på, hvad jeg har skrevet, men når jeg så melder dem til i morgen eller i overmorgen, så ved jeg ikke om de sidder og kontrollerer og ringer til kommunen. Det har jeg ikke styr på.}\\
\noindent
\textbf{Når I flytter patienterne rundt, nu siger du, hvis I har en akut patient, så får de lov til at ligge på gangene først eller ryger de ind på stuen?}\\
\noindent
\textit{De ryger ind på stuen.}\\
\noindent
\textbf{Så prioriterer man så at få flyttet nogle af dem, der senere i deres helingsforløb eller elektive patienter ud eller?}\\
\noindent
\textit{Aldrig ude på gangen.}\\
\noindent
\textbf{Okay, det er ned på andre stuer eller?}\\
\noindent
\textit{Ja, så kan vi finde på det.}\\
\noindent
\textbf{Okay, så det er de elektive patienter og dem der er mindst kritiske akutte som I flytter ned på andre afdelinger?} \\
\noindent
\textit{Nej, ikke andre afdelinger. Det gør vi ikke så meget af, men vi kan godt finde på at flytte en patient fra vores afdeling over på vores naboafdeling, som også er en ortopædkirurgisk afdeling. I dagligdagen er vi delt op i to forskellige ting. Det hedder stadigvæk det samme og det er det samme. Der veksler vi hele tiden mellem patienterne, når det er de akutte, fordi om aftenen så kan de godt have travlt, hvor vi så tager deres patienter og næste dag, så er der faldet lidt ro over det, så hører de til derovre og så flytter vi dem derover. Men vi flytter dem ikke på en mave afdeling eller en medicinsk afdeling. Det er ikke det der foregår, men det vi snakkede om lige før med AMA, der har rådighed over hele sygehuset.}\\
\noindent
\textbf{Er der forskel på, hvordan man prioriterer de her patienter? Og hvem man flytter eller er det forskelligt?}\\
\noindent
\textit{Ja, der er det blevet sådan, at vi passer på børnene og vi passer på rygpatienter og vi passer på traumepatienter. Hvis der kommer nogle med hoftebrud, det er sådan blevet en folkesygdom. De hører til ovre på den anden afdeling, men dem passer vi også, hvis de ikke kan eller har tid til det, og har plads til dem. Men vi sender helst ikke børn over og vi må ikke sende rygpatienter over. Førhen hed det også knogle konstruktion, som også er meget specielt på vores afdeling. Dem vil vi heller ikke have andre steder, fordi det er sådan noget, hvor vi forlænger benene med noget apparatur og der er kun nogle få læger, som er her og de vil ikke have dem andre steder hen, fordi så skrider det hele og så går der noget galt med deres udstyr og varetagning. Det er ligesådan med rygpatienter. De vil altså helst have, at de er her, for de har oplevet, det er også det med erfaring. De har sendt folk ud af huset, hvor det andet sygehus ikke har varetaget det som de skulle eller har kørt deres eget system, så ødelægger det noget for patienten, også kan de måske starte forfra og det vil de ikke have mere.}\\ 
\noindent
\textbf{Så vil vi sige tak for interviewet.}\\
\noindent
\textit{Velbekomme.}\\



\section{Lægesekretær} \label{bilagsek}
\textbf{Vi kunne godt tænke os at høre om, hvem er du?}\\
\noindent
\textit{Jeg hedder Janni Ulstrup og jeg er lægesekretær her i ortopædkirurgisk afdeling i idrætsklinikken.} \\
\noindent
\textbf{Og hvor længe har du arbejdet her?} \\
\noindent
\textit{Det har jeg i fire år.} \\
\noindent
\textbf{Fire år?} \\
\noindent
\textit{Ja. Og så har jeg været i ortopædkirurgien, 7 år tidligere i min tid som lægesekretær.} \\
\noindent
\textbf{Og må vi citere dig for interviewet?}\\
\noindent
\textit{Ja.} \\
\noindent
\textbf{Jamen nu vil jeg spørge, hvordan planlægger I de her elektive patienter?}\\
\noindent
\textit{Vi har faste pladser. Vi har ikke så mange elektive under indlæggelse, men vi har rigtig mange elektive i dagkirurgi. I dagkirurgi har vi rigtig mange tider og  holder altid nogle enkelte ledige til, hvis der dukker noget akut op under indlæggelse. Det er kun de patienter, der er absolut nødvendigt skal opereres under indlæggelse, der kommer derind og det er ikke altid vi faktisk udnytter alle vores tider. I denne her uge har vi for eksempel ikke brugt vores pladser som vi så har været heldige med fordi der har været mange akutte patienter, så de har haft meget overbelægning eller mange patienter deroppe, så det har faktisk været heldigt i denne her uge. Men dem planlægger vi sådan, at vi som regel altid har plads til lidt ekstra. Vi booker ikke stramt op, kun i absolut nødstilfælde og det er, hvis der kommer en ekstra en på som man er nødt til at skal have plads til.}
\textbf{Er der en procentsats eller en del af sengene, der er afsat til akutte patienter?}\\
\noindent
\textit{Nej det tror jeg ikke. Ikke mig bekendt. Altså vi må operere om tirsdagen for vores vedkommende og der må vi sætte en til to patienter på afhængigt af, hvor stort et indgreb det er. Det er af hensyn til operationsgangen og deres kapacitet der. Det er egentligt ikke af hensyn til sengeafdelingen. Det er en del af vores elektiv. Det ved de, at der kan komme to, hvis det er to små operationer. Men en del af de patienter vi sender op på sengeafdelingen, selvom de er nødt til at skulle opereres under indlæggelse, så er det måske noget i forhold til anæstesi sikkerhed og opvågning. Mange af de patienter kan måske godt allerede gå hjem senere på dagen. Det er ikke nødvendigvis ensbetydende med, at de skal ligge deroppe i flere dage. Men der er også dem, der så kommer der op fordi de netop skal ligge der i nogle dage, fordi de skal træne efter operationen. Der er ikke som er 100 procent fastlagt.} 
\textbf{Når I indkalder de her elektive patienter, bliver det så gjort med kort eller lang varsel? Hvordan foregår det?}\\
\noindent
\textit{Jamen, hvis der sidder en patient henne i ambulatoriet i dag, der skal opereres under indlæggelse, så vil patienten få en tid med hjem i dag. Og hvis det ikke er noget, der haster og hvis det ikke er noget akut, så finder vi første ledige tid under de omstændigheder der nu er. Skal det være en bestemt læge, er det patienten, der siger jeg vil gerne vente til efter 15. December eller under de forhold, der nu er der. Er det sådan at patienten siger første ledige tid, jamen så vil de simpelthen få første ledige tid og det kan godt være i næste uge. Det kan også være om 14 dage, tre uger.}
\textbf{Hvis så man er nødt til at rykke den elektive patient under akutte patienter, hvordan gør man så?}\\
\noindent
\textit{Det er igen et spørgsmål om for vores område fordi vi ikke er så stramt booket op under indlæggelse. Så har vi plads til, at vi kan putte dem på måske en uge, måske 14 dage senere. Altså vi får dem på forholdsvis hurtigt og vi skal nok sørge for, at de får en tid. Vi kan også løbe ind i perioder, hvor vi har mange, der skal ligge under indlæggelse. Og så kan vi selvfølgelig godt løbe ind i kniberi med, hvad gør vi så her. Og så er det egentligt vores specialeansvarlige overlæge, der træder ind og siger, jamen så gør vi sådan her. Så er det ham, der ligesom træffer beslutningen om, hvordan håndterer vi det.} 
\textbf{Når I indkalder de her patienter, planlægger de elektive patienter, planlægger I så også efter for eksempel operationstype, hvor lang tid skal de ligge der?}\\
\noindent
\textit{Nej.} \\
\noindent
\textbf{Er der noget du føler er relevant lige at tilføje.} \\
\noindent
\textit{Nej, ikke andet end det her er jo kun for sengeafdelingen. Vores O6 kører jo ligesom et forløb for sig selv. Det er ambulant kirurgi. Det kører ligesom sit liv for sig selv, men der har vi også altid buffer, hvis der er nogen, der bliver aflyst. Vi prøver på at have en regel om, at den første tid bliver som regel ikke booket. Det er den vi holder tilbage, fordi det sjældent, at det er den første patient man aflyser, hvis man aflyser, så opererer man som regel de patienter, der er mødt ind i dagkirurgi. Så vælger man at aflyse dem der endnu ikke er mødt. Dem der først kommer senere på dagen. Så på den måde, hvis der er nogen, der bliver aflyst i dag og de så får en tid på mandag fra morgenstunden af, jamen så møder de ind, så bliver de opereret, så man undgår, at de bliver aflyst mere end én gang. Men hvis vi sætter dem på senere på dagen, så kan vi risikere, at de rent faktisk bliver aflyst en gang mere, hvis det skulle være. 7-9-13 og bank under bordet, det sker heldigvis ikke ret tit, men det kan jo ske på grund af sygdom.} \\
\noindent
\textbf{Bare lige om jeg har forstået det korrekt, så snakker du noget om, at I ud fra, når I planlægger, så planlægger I om, altså ud fra om kirurgen også, om den er tilstede den dag det skal være, så det gør I ud fra erfaring og operationstype.}\\
\noindent
\textit{Det er ud fra, det er lægen selv, der har skrevet på. Lægen udfylder en operations tilmelding nu når de skriver patienter op til operation og derpå skriver han skal jeg selv operere denne her patient eller kan det være en anden der gør det. Patienten får også selv lov at vælge og sige jamen ønsker du, at det skal være den læge. Hvis ikke han har skrevet noget på, så får patienten selv lov at vælge. Jamen jeg vil gerne have, at det var den eller nej, jeg vil bare gerne have så hurtig en tid jeg kan få.} \\
\noindent
\textbf{Okay, så det er de overvejelser I gør, når det er?}\\
\noindent
\textit{Ja, det er ligesom vores sygeplejersker, der sidder og har papirene og siger jamen vi skal have en tid, der passer ind med det her, men patienten vil bare gerne ind så hurtigt som muligt. Det kan også godt nogen gange være, at lægen skriver han gerne selv vil gøre det, men hvis der går et stykke tid og patienter siger jeg vil bare gerne have det overstået, så kan man måske godt sige okay, fint nok. Det er ikke noget, der er noget specielt kompliceret i, så det vil en anden også godt kunne gøre, det her.}

